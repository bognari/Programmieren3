\documentclass[
  10pt,                   % Schriftgröße
  DIV12,
  german,                 % für Umlaute, Silbentrennung etc.
  oneside,                % einseitiges Dokument
  %titlepage,              % es wird eine Titelseite verwendet
  parskip=half,           % Abstand zwischen Absätzen (halbe Zeile)
  headings=normal,        % Größe der Überschriften verkleinern
  captions=tableheading,  % Beschriftung von Tabellen unterhalb ausgeben
  %final                   % Status des Dokuments (final/draft)
]{scrartcl}      

\usepackage[utf8x]{inputenc}
\usepackage[T1]{fontenc}
\usepackage[ngerman]{babel}
\usepackage{stmaryrd}
\usepackage{amsfonts}
\usepackage{amssymb}
\usepackage{amsmath}
\usepackage{microtype}

\usepackage{listings}
\usepackage{color}
\usepackage{pxfonts}

\definecolor{mygreen}{RGB}{51,141,120}
\definecolor{myblue}{RGB}{0,128,180}
\definecolor{myviolet}{RGB}{118,0,118}

\lstset{ %
  language=Haskell,
  backgroundcolor=\color{white},         % choose the background color
  basicstyle=\ttfamily\footnotesize,     % size of fonts used for the code
  numbers=none,
  breaklines=true,                       % automatic line breaking only at whitespace
  captionpos=b,                          % sets the caption-position to bottom
  commentstyle=\color{mygreen},    % comment style
  escapeinside={\%*}{*)},                % if you want to add LaTeX within your code
  keywordstyle=\color{myblue}\bfseries, % keyword style
  stringstyle=\color{myviolet},    % string literal style
  frame=single,
  tabsize=2
}
\usepackage{tikz}
\usetikzlibrary{
  arrows,
  shapes.misc,
  shapes.arrows,
  chains,
  matrix,
  positioning,
  scopes,
  decorations.pathmorphing,
  shadows,
  backgrounds
}


\usepackage{float}
\usepackage{geometry}

\geometry{verbose,                     % zeigt die eingestellten Parameter beim
                                       % Latexlauf an
      paper=a4paper,                   % Papierformat
      top=25mm,                        % Rand oben
      left=25mm,                       % Rand links
      right=25mm,                      % Rand rechts
      bottom=40mm,                     % Rand unten
      pdftex                           % schreibt das Papierformat in die
                                       % Ausgabe damit Ausgabeprogramm
                                       % Papiergröße erkennt
}

\usepackage[%
  automark,
  headsepline,                %% Separation line below the header
  footsepline,               %% Separation line above the footer
  markuppercase
]{scrpage2}

\automark[subsection]{section}
\pagestyle{scrheadings}
\renewcommand*\thesubsection{\roman{subsection})}


%\lehead{\bfseries\pagemark}    %% Top left on even pages
%\lohead{\bfseries\headmark}    %% Top left on odd pages
%\rehead{\bfseries\headmark}    %% Top right on even pages
%\rohead{\bfseries\pagemark}    %% Top right on odd pages
\chead{Programmieren für Fortgeschrittene \\ Cheat Sheet}                       %% Top center

\begin{document}


\section{Kommentare}
\begin{itemize}
\item Zeilenkommentar: \lstinline|-- Text| 
\item Blockkommentar: \lstinline|{- Text -}|
\end{itemize}

\section{Konstanten}
\begin{minipage}[hbt]{6cm}
\centering
\begin{lstlisting}
pi :: Double
pi = 3.14
\end{lstlisting}
\end{minipage}
\hfill
\begin{minipage}[hbt]{8cm}
\centering
\scalebox{0.7}{\begin{tikzpicture}[
      nonterminal/.style={
         rectangle,
         minimum size=5.0mm,
         very thick,
         draw=blue!50!black!50,
         top color=white,
         bottom color=blue!50!black!20,
         font=\itshape\scriptsize,
         %text height=1.5ex,
         %text depth=.25ex
      },
      terminal/.style={
         rounded rectangle,
         minimum size=5.0mm,
         very thick,draw=black!50,
         top color=white,bottom color=black!20,
         font=\ttfamily\scriptsize,
         %text height=1.5ex,
         %text depth=.25ex
      },
      skip loop/.style={
         to path={-- ++(0,#1) -| (\tikztotarget)}
      },
      point/.style={coordinate},>=stealth',thick,draw=black!50,
      tip/.style={->,shorten >=0.007pt},every join/.style={rounded corners},
      hv path/.style={to path={-| (\tikztotarget)}},
      vh path/.style={to path={|- (\tikztotarget)}},
      %text height=1.5ex,text depth=.25ex
    ]
    \matrix[column sep=5.0mm, row sep=3.0mm] {
 &  & \node (n13) [nonterminal] {Bezeichner}; & \node (n14) [terminal] {::}; & \node (n15) [nonterminal] {Typ Erg}; & \node (n16) [terminal] {LF}; &  & \\
\node (n21) [circle, draw, inner sep=0pt, minimum size=0.75ex] {}; & \node (n22) [point] {}; &  &  &  &  & \node (n27) [point] {}; & \node (n28) [point] {};\\
 &  & \node (n33) [point] {}; &  &  &  &  & \\
 &  &  &  &  &  &  & \\
 &  &  &  & \node (n55) [nonterminal] {Wert}; &  &  & \\
\node (n61) [point] {}; & \node (n62) [nonterminal] {Bezeichner}; & \node (n63) [terminal] {=}; & \node (n64) [point] {}; &  & \node (n66) [point] {}; & \node (n67) [terminal] {LF}; & \node (n68) [circle, draw, inner sep=0pt, minimum size=0.75ex] {};\\
 &  &  &  & \node (n75) [nonterminal] {Ausdruck}; &  &  & \\
    };
  { [start chain]
       \chainin (n21);
       \chainin (n22)    [join];
  }
  { [start chain]
       \chainin (n22);
       \chainin (n13)    [join=by {vh path,tip}];
  }
  { [start chain]
       \chainin (n13);
       \chainin (n14)    [join=by tip];
  }
  { [start chain]
       \chainin (n14);
       \chainin (n15)    [join=by tip];
  }
  { [start chain]
       \chainin (n15);
       \chainin (n16)    [join=by tip];
  }
  { [start chain]
       \chainin (n16);
       \chainin (n27)    [join=by {hv path}];
  }
  { [start chain]
       \chainin (n22);
       \chainin (n33)    [join=by {vh path}];
  }
  { [start chain]
       \chainin (n33);
       \chainin (n27)    [join=by {hv path}];
  }
  { [start chain]
       \chainin (n27);
       \chainin (n28)    [join=by tip];
  }
  { [start chain]
       \chainin (n61);
       \chainin (n62)    [join=by tip];
  }
  { [start chain]
       \chainin (n62);
       \chainin (n63)    [join=by tip];
  }
  { [start chain]
       \chainin (n63);
       \chainin (n64)    [join];
  }
  { [start chain]
       \chainin (n64);
       \chainin (n55)    [join=by {vh path,tip}];
  }
  { [start chain]
       \chainin (n55);
       \chainin (n66)    [join=by {hv path}];
  }
  { [start chain]
       \chainin (n64);
       \chainin (n75)    [join=by {vh path,tip}];
  }
  { [start chain]
       \chainin (n75);
       \chainin (n66)    [join=by {hv path}];
  }
  { [start chain]
       \chainin (n66);
       \chainin (n67)    [join=by tip];
  }
  { [start chain]
       \chainin (n67);
       \chainin (n68)    [join=by tip];
  }
\end{tikzpicture}
}
\end{minipage}

\section{Funktionen}
\begin{minipage}[hbt]{6cm}
\centering
\begin{lstlisting}
plus :: Int -> Int -> Int
plus a b = a + b
\end{lstlisting}
\end{minipage}
\hfill
\begin{minipage}[hbt]{8cm}
\centering
\scalebox{0.6}{\begin{tikzpicture}[
      nonterminal/.style={
         rectangle,
         minimum size=5.0mm,
         very thick,
         draw=blue!50!black!50,
         top color=white,
         bottom color=blue!50!black!20,
         font=\itshape\scriptsize,
         %text height=1.5ex,
         %text depth=.25ex
      },
      terminal/.style={
         rounded rectangle,
         minimum size=5.0mm,
         very thick,draw=black!50,
         top color=white,bottom color=black!20,
         font=\ttfamily\scriptsize,
         %text height=1.5ex,
         %text depth=.25ex
      },
      skip loop/.style={
         to path={-- ++(0,#1) -| (\tikztotarget)}
      },
      point/.style={coordinate},>=stealth',thick,draw=black!50,
      tip/.style={->,shorten >=0.007pt},every join/.style={rounded corners},
      hv path/.style={to path={-| (\tikztotarget)}},
      vh path/.style={to path={|- (\tikztotarget)}},
      %text height=1.5ex,text depth=.25ex
    ]
    \matrix[column sep=5.0mm, row sep=3.0mm] {
 &  &  &  &  & \node (n16) [point] {}; &  &  &  &  &  &  & \\
 &  & \node (n23) [nonterminal] {Bezeichner}; & \node (n24) [terminal] {::}; & \node (n25) [point] {}; & \node (n26) [nonterminal] {Typ Para}; & \node (n27) [terminal] {->}; & \node (n28) [point] {}; & \node (n29) [nonterminal] {Typ Erg}; & \node (n210) [terminal] {LF}; &  &  & \\
\node (n31) [circle, draw, inner sep=0pt, minimum size=0.75ex] {}; & \node (n32) [point] {}; &  &  &  &  &  &  &  &  & \node (n311) [point] {}; & \node (n312) [point] {}; & \\
 &  & \node (n43) [point] {}; &  &  &  &  &  &  &  &  &  & \\
 &  &  &  &  &  &  &  &  &  &  &  & \\
 &  &  &  &  &  & \node (n67) [point] {}; &  &  &  &  &  & \\
 &  &  &  & \node (n75) [point] {}; &  &  &  & \node (n79) [nonterminal] {Wert}; &  &  &  & \\
\node (n81) [point] {}; & \node (n82) [point] {}; & \node (n83) [nonterminal] {Bezeichner}; & \node (n84) [point] {}; & \node (n85) [nonterminal] {Bezeichner Para}; & \node (n86) [point] {}; & \node (n87) [terminal] {=}; & \node (n88) [point] {}; &  & \node (n810) [point] {}; & \node (n811) [terminal] {LF}; & \node (n812) [point] {}; & \node (n813) [circle, draw, inner sep=0pt, minimum size=0.75ex] {};\\
 &  &  &  &  &  &  &  & \node (n99) [nonterminal] {Ausdruck}; &  &  &  & \\
    };
  { [start chain]
       \chainin (n31);
       \chainin (n32)    [join];
  }
  { [start chain]
       \chainin (n32);
       \chainin (n23)    [join=by {vh path,tip}];
  }
  { [start chain]
       \chainin (n23);
       \chainin (n24)    [join=by tip];
  }
  { [start chain]
       \chainin (n24);
       \chainin (n25)    [join];
  }
  { [start chain]
       \chainin (n25);
       \chainin (n26)    [join=by tip];
  }
  { [start chain]
       \chainin (n26);
       \chainin (n27)    [join=by tip];
  }
  { [start chain]
       \chainin (n27);
       \chainin (n28)    [join];
  }
  { [start chain]
       \chainin (n28);
       \chainin (n29)    [join=by tip];
  }
  { [start chain]
       \chainin (n29);
       \chainin (n210)    [join=by tip];
  }
  { [start chain]
       \chainin (n210);
       \chainin (n311)    [join=by {hv path}];
  }
  { [start chain]
       \chainin (n32);
       \chainin (n43)    [join=by {vh path}];
  }
  { [start chain]
       \chainin (n43);
       \chainin (n311)    [join=by {hv path}];
  }
  { [start chain]
       \chainin (n311);
       \chainin (n312)    [join=by tip];
  }
  { [start chain]
       \chainin (n81);
       \chainin (n82)    [join];
  }
  { [start chain]
       \chainin (n82);
       \chainin (n83)    [join=by tip];
  }
  { [start chain]
       \chainin (n83);
       \chainin (n84)    [join];
  }
  { [start chain]
       \chainin (n84);
       \chainin (n85)    [join=by tip];
  }
  { [start chain]
       \chainin (n85);
       \chainin (n86)    [join];
  }
  { [start chain]
       \chainin (n86);
       \chainin (n87)    [join=by tip];
  }
  { [start chain]
       \chainin (n87);
       \chainin (n88)    [join];
  }
  { [start chain]
       \chainin (n88);
       \chainin (n79)    [join=by {vh path,tip}];
  }
  { [start chain]
       \chainin (n79);
       \chainin (n810)    [join=by {hv path}];
  }
  { [start chain]
       \chainin (n88);
       \chainin (n99)    [join=by {vh path,tip}];
  }
  { [start chain]
       \chainin (n99);
       \chainin (n810)    [join=by {hv path}];
  }
  { [start chain]
       \chainin (n810);
       \chainin (n811)    [join=by tip];
  }
  { [start chain]
       \chainin (n811);
       \chainin (n812)    [join];
  }
  { [start chain]
       \chainin (n812);
       \chainin (n813)    [join=by tip];
  }
  { [start chain]
       \chainin (n28);
       \chainin (n16)    [join=by vh path];
       \chainin (n25)    [join=by {hv path,tip}];
  }
  { [start chain]
       \chainin (n86);
       \chainin (n75)    [join=by vh path];
       \chainin (n84)    [join=by {hv path,tip}];
  }
  { [start chain]
       \chainin (n812);
       \chainin (n67)    [join=by vh path];
       \chainin (n82)    [join=by {hv path,tip}];
  }
\end{tikzpicture}
}
\end{minipage}

\section{If-Then-Else}
\begin{minipage}[hbt]{6cm}
\centering
\begin{lstlisting}
ggT a b = if b == 0 
        then a 
        else ggT b (mod a b)
\end{lstlisting}
\end{minipage}
\hfill
\begin{minipage}[hbt]{8cm}
\centering
\scalebox{0.7}{\input{images/ifthenelse.tex}}
\end{minipage}

\section{Case Of}
\begin{minipage}[hbt]{6cm}
\centering
\begin{lstlisting}
ggT a b = case b of 
        0 -> a
        _ -> ggT b (mod a b)
\end{lstlisting}
\end{minipage}
\hfill
\begin{minipage}[hbt]{8cm}
\centering
\scalebox{0.8}{\begin{tikzpicture}[
      nonterminal/.style={
         rectangle,
         minimum size=5.0mm,
         very thick,
         draw=blue!50!black!50,
         top color=white,
         bottom color=blue!50!black!20,
         font=\itshape\scriptsize,
         %text height=1.5ex,
         %text depth=.25ex
      },
      terminal/.style={
         rounded rectangle,
         minimum size=5.0mm,
         very thick,draw=black!50,
         top color=white,bottom color=black!20,
         font=\ttfamily\scriptsize,
         %text height=1.5ex,
         %text depth=.25ex
      },
      skip loop/.style={
         to path={-- ++(0,#1) -| (\tikztotarget)}
      },
      point/.style={coordinate},>=stealth',thick,draw=black!50,
      tip/.style={->,shorten >=0.007pt},every join/.style={rounded corners},
      hv path/.style={to path={-| (\tikztotarget)}},
      vh path/.style={to path={|- (\tikztotarget)}},
      %text height=1.5ex,text depth=.25ex
    ]
    \matrix[column sep=5.0mm, row sep=3.0mm] {
\node (n11) [circle, draw, inner sep=0pt, minimum size=0.75ex] {}; & \node (n12) [terminal] {=}; & \node (n13) [terminal] {case}; & \node (n14) [nonterminal] {Variable}; & \node (n15) [terminal] {of}; & \node (n16) [terminal] {LF}; & \node (n17) [point] {}; & \\
 &  &  &  &  &  &  & \\
 &  &  & \node (n34) [point] {}; &  &  &  & \\
\node (n41) [point] {}; & \node (n42) [point] {}; & \node (n43) [nonterminal] {erwarteter Wert}; & \node (n44) [terminal] {->}; & \node (n45) [nonterminal] {Ausdruck}; & \node (n46) [terminal] {LF}; & \node (n47) [point] {}; & \node (n48) [point] {};\\
 &  &  &  &  &  &  & \\
 &  &  & \node (n64) [point] {}; &  &  &  & \\
\node (n71) [point] {}; & \node (n72) [point] {}; & \node (n73) [terminal] {\_}; & \node (n74) [terminal] {->}; & \node (n75) [nonterminal] {default Ausdruck}; & \node (n76) [point] {}; & \node (n77) [circle, draw, inner sep=0pt, minimum size=0.75ex] {}; & \\
    };
  { [start chain]
       \chainin (n11);
       \chainin (n12)    [join=by tip];
  }
  { [start chain]
       \chainin (n12);
       \chainin (n13)    [join=by tip];
  }
  { [start chain]
       \chainin (n13);
       \chainin (n14)    [join=by tip];
  }
  { [start chain]
       \chainin (n14);
       \chainin (n15)    [join=by tip];
  }
  { [start chain]
       \chainin (n15);
       \chainin (n16)    [join=by tip];
  }
  { [start chain]
       \chainin (n16);
       \chainin (n17)    [join=by tip];
  }
  { [start chain]
       \chainin (n41);
       \chainin (n42)    [join];
  }
  { [start chain]
       \chainin (n42);
       \chainin (n43)    [join=by tip];
  }
  { [start chain]
       \chainin (n43);
       \chainin (n44)    [join=by tip];
  }
  { [start chain]
       \chainin (n44);
       \chainin (n45)    [join=by tip];
  }
  { [start chain]
       \chainin (n45);
       \chainin (n46)    [join=by tip];
  }
  { [start chain]
       \chainin (n46);
       \chainin (n47)    [join];
  }
  { [start chain]
       \chainin (n47);
       \chainin (n48)    [join=by tip];
  }
  { [start chain]
       \chainin (n71);
       \chainin (n72)    [join];
  }
  { [start chain]
       \chainin (n72);
       \chainin (n73)    [join=by tip];
  }
  { [start chain]
       \chainin (n73);
       \chainin (n74)    [join=by tip];
  }
  { [start chain]
       \chainin (n74);
       \chainin (n75)    [join=by tip];
  }
  { [start chain]
       \chainin (n75);
       \chainin (n76)    [join];
  }
  { [start chain]
       \chainin (n76);
       \chainin (n77)    [join=by tip];
  }
  { [start chain]
       \chainin (n47);
       \chainin (n34)    [join=by vh path];
       \chainin (n42)    [join=by {hv path,tip}];
  }
  { [start chain]
       \chainin (n72);
       \chainin (n64)    [join=by vh path];
       \chainin (n76)    [join=by {hv path,tip}];
  }
\end{tikzpicture}
}
\end{minipage}

\section{Pattern Matching}
\begin{minipage}[hbt]{6cm}
\centering
\begin{lstlisting}
ggT a 0 = a
ggT a b = ggT b (mod a b)
\end{lstlisting}
\end{minipage}
\hfill
\begin{minipage}[hbt]{8cm}
\centering
\scalebox{0.8}{\begin{tikzpicture}[
      nonterminal/.style={
         rectangle,
         minimum size=5.0mm,
         very thick,
         draw=blue!50!black!50,
         top color=white,
         bottom color=blue!50!black!20,
         font=\itshape\scriptsize,
         %text height=1.5ex,
         %text depth=.25ex
      },
      terminal/.style={
         rounded rectangle,
         minimum size=5.0mm,
         very thick,draw=black!50,
         top color=white,bottom color=black!20,
         font=\ttfamily\scriptsize,
         %text height=1.5ex,
         %text depth=.25ex
      },
      skip loop/.style={
         to path={-- ++(0,#1) -| (\tikztotarget)}
      },
      point/.style={coordinate},>=stealth',thick,draw=black!50,
      tip/.style={->,shorten >=0.007pt},every join/.style={rounded corners},
      hv path/.style={to path={-| (\tikztotarget)}},
      vh path/.style={to path={|- (\tikztotarget)}},
      %text height=1.5ex,text depth=.25ex
    ]
    \matrix[column sep=5.0mm, row sep=3.0mm] {
 &  &  & \node (n14) [point] {}; &  &  &  &  & \\
\node (n21) [circle, draw, inner sep=0pt, minimum size=0.75ex] {}; & \node (n22) [nonterminal] {Funktion}; & \node (n23) [point] {}; & \node (n24) [nonterminal] {erwarteter Wert}; & \node (n25) [terminal] { }; & \node (n26) [point] {}; & \node (n27) [terminal] {=}; & \node (n28) [nonterminal] {Ausdruck}; & \node (n29) [circle, draw, inner sep=0pt, minimum size=0.75ex] {};\\
    };
  { [start chain]
       \chainin (n21);
       \chainin (n22)    [join=by tip];
  }
  { [start chain]
       \chainin (n22);
       \chainin (n23)    [join];
  }
  { [start chain]
       \chainin (n23);
       \chainin (n24)    [join=by tip];
  }
  { [start chain]
       \chainin (n24);
       \chainin (n25)    [join=by tip];
  }
  { [start chain]
       \chainin (n25);
       \chainin (n26)    [join];
  }
  { [start chain]
       \chainin (n26);
       \chainin (n27)    [join=by tip];
  }
  { [start chain]
       \chainin (n27);
       \chainin (n28)    [join=by tip];
  }
  { [start chain]
       \chainin (n28);
       \chainin (n29)    [join=by tip];
  }
  { [start chain]
       \chainin (n26);
       \chainin (n14)    [join=by vh path];
       \chainin (n23)    [join=by {hv path,tip}];
  }
\end{tikzpicture}
}
\end{minipage}

\section{Guards}
\begin{minipage}[hbt]{6cm}
\centering
\begin{lstlisting}
ggT a b | b == 0    = a
        | otherwise = 
            ggT b (mod a b) 
\end{lstlisting}
\end{minipage}
\hfill
\begin{minipage}[hbt]{8cm}
\centering
\scalebox{0.8}{\begin{tikzpicture}[
      nonterminal/.style={
         rectangle,
         minimum size=5.0mm,
         very thick,
         draw=blue!50!black!50,
         top color=white,
         bottom color=blue!50!black!20,
         font=\itshape\scriptsize,
         %text height=1.5ex,
         %text depth=.25ex
      },
      terminal/.style={
         rounded rectangle,
         minimum size=5.0mm,
         very thick,draw=black!50,
         top color=white,bottom color=black!20,
         font=\ttfamily\scriptsize,
         %text height=1.5ex,
         %text depth=.25ex
      },
      skip loop/.style={
         to path={-- ++(0,#1) -| (\tikztotarget)}
      },
      point/.style={coordinate},>=stealth',thick,draw=black!50,
      tip/.style={->,shorten >=0.007pt},every join/.style={rounded corners},
      hv path/.style={to path={-| (\tikztotarget)}},
      vh path/.style={to path={|- (\tikztotarget)}},
      %text height=1.5ex,text depth=.25ex
    ]
    \matrix[column sep=5.0mm, row sep=3.0mm] {
 &  &  &  & \node (n15) [point] {}; &  &  &  & \\
\node (n21) [circle, draw, inner sep=0pt, minimum size=0.75ex] {}; & \node (n22) [point] {}; & \node (n23) [terminal] {|}; & \node (n24) [nonterminal] {boolescher Term}; & \node (n25) [terminal] {=}; & \node (n26) [nonterminal] {Ausdruck}; & \node (n27) [terminal] {LF}; & \node (n28) [point] {}; & \node (n29) [point] {};\\
 &  &  &  &  &  &  &  & \\
\node (n41) [point] {}; & \node (n42) [terminal] {otherwise}; & \node (n43) [terminal] {=}; & \node (n44) [nonterminal] {default Ausdruck}; & \node (n45) [circle, draw, inner sep=0pt, minimum size=0.75ex] {}; &  &  &  & \\
    };
  { [start chain]
       \chainin (n21);
       \chainin (n22)    [join];
  }
  { [start chain]
       \chainin (n22);
       \chainin (n23)    [join=by tip];
  }
  { [start chain]
       \chainin (n23);
       \chainin (n24)    [join=by tip];
  }
  { [start chain]
       \chainin (n24);
       \chainin (n25)    [join=by tip];
  }
  { [start chain]
       \chainin (n25);
       \chainin (n26)    [join=by tip];
  }
  { [start chain]
       \chainin (n26);
       \chainin (n27)    [join=by tip];
  }
  { [start chain]
       \chainin (n27);
       \chainin (n28)    [join];
  }
  { [start chain]
       \chainin (n28);
       \chainin (n29)    [join=by tip];
  }
  { [start chain]
       \chainin (n41);
       \chainin (n42)    [join=by tip];
  }
  { [start chain]
       \chainin (n42);
       \chainin (n43)    [join=by tip];
  }
  { [start chain]
       \chainin (n43);
       \chainin (n44)    [join=by tip];
  }
  { [start chain]
       \chainin (n44);
       \chainin (n45)    [join=by tip];
  }
  { [start chain]
       \chainin (n28);
       \chainin (n15)    [join=by vh path];
       \chainin (n22)    [join=by {hv path,tip}];
  }
\end{tikzpicture}
}
\end{minipage}

\section{Module}
\begin{lstlisting}
module Wurf(weite, square) where
weite :: Double -> Double -> Double
weite v0 phi = ((square v0) / 9.81) * sin (2 * phi)
square :: Double -> Double
square x = x * x

module Foo where
import Wurf hiding (weite)
bar ... = ... (square a) ...
\end{lstlisting}
\section{Eigene Datentypen}
\begin{lstlisting}
data Point = Point{x :: Double, y :: Double}

data Shape = Circle{center :: Point, 
    radius :: Double}
  | Rectangle{point :: Point, width :: Double, 
    height :: Double}
  | Triangle{point1 :: Point, point2 :: Point, 
    point3 :: Point}
\end{lstlisting}
\section{Typparameter}
Keine Typparameter in Datentyp-Definitionen!
\begin{lstlisting}
data (Eq a, Ord a) => Pair a b = 
  PairConst {first :: a, second :: b} deriving(Show)
\end{lstlisting}

Besser:
\begin{lstlisting}
data Pair a b = PairConst a b deriving(Show)

instance Eq a => Eq (Pair a b) where
  (PairConst i1 _) == (PairConst i2 _) = i1 == i2

instance Ord a => Ord (Pair a b) where
   (PairConst i1 _) <= (PairConst i2 _) = i1 <= i2

bubbleSort :: (Ord a) => [(Pair a b)] -> [(Pair a b)]
bubbleSort [] = []
bubbleSort (x:xs) = step $ foldl go (x,[]) xs where
  go (y,acc) x = (min x y, max x y : acc)
  step (x,acc) = x : bubbleSort acc
\end{lstlisting}

\$ ist nichts anderes als: 
\begin{lstlisting}
(7 + 6) / (5 + 3) = (7 + 6) / $ 5 + 3
\end{lstlisting}
Die "`Klammerung"' geht bis zum Zeilenende.
\section{Listen}
%\begin{minipage}[hbt]{7cm}
%\centering
\begin{lstlisting}
primes = sieve [2..]
  where
    sieves (p:xs) = p:sieves [x|x<- xs, mod x p > 0]
\end{lstlisting}
%\end{minipage}
%\hfill
%\begin{minipage}[hbt]{7cm}
%\centering
\begin{center}
\scalebox{1}{\begin{tikzpicture}[
      nonterminal/.style={
         rectangle,
         minimum size=5.0mm,
         very thick,
         draw=blue!50!black!50,
         top color=white,
         bottom color=blue!50!black!20,
         font=\itshape\scriptsize,
         %text height=1.5ex,
         %text depth=.25ex
      },
      terminal/.style={
         rounded rectangle,
         minimum size=5.0mm,
         very thick,draw=black!50,
         top color=white,bottom color=black!20,
         font=\ttfamily\scriptsize,
         %text height=1.5ex,
         %text depth=.25ex
      },
      skip loop/.style={
         to path={-- ++(0,#1) -| (\tikztotarget)}
      },
      point/.style={coordinate},>=stealth',thick,draw=black!50,
      tip/.style={->,shorten >=0.007pt},every join/.style={rounded corners},
      hv path/.style={to path={-| (\tikztotarget)}},
      vh path/.style={to path={|- (\tikztotarget)}},
      %text height=1.5ex,text depth=.25ex
    ]
    \matrix[column sep=5.0mm, row sep=3.0mm] {
 &  &  &  &  &  &  &  & \node (n19) [point] {}; &  &  &  &  &  &  &  & \\
 &  &  &  &  &  &  & \node (n28) [point] {}; &  &  &  &  & \node (n213) [point] {}; &  &  &  & \\
\node (n31) [circle, draw, inner sep=0pt, minimum size=0.75ex] {}; & \node (n32) [terminal] {[}; & \node (n33) [nonterminal] {Ausdruck}; & \node (n34) [point] {}; & \node (n35) [terminal] {|}; & \node (n36) [point] {}; & \node (n37) [nonterminal] {Variable}; & \node (n38) [terminal] {<-}; & \node (n39) [nonterminal] {Quelle}; & \node (n310) [terminal] {,}; & \node (n311) [point] {}; & \node (n312) [point] {}; & \node (n313) [nonterminal] {Filter}; & \node (n314) [point] {}; & \node (n315) [point] {}; & \node (n316) [terminal] {]}; & \node (n317) [circle, draw, inner sep=0pt, minimum size=0.75ex] {};\\
    };
  { [start chain]
       \chainin (n31);
       \chainin (n32)    [join=by tip];
  }
  { [start chain]
       \chainin (n32);
       \chainin (n33)    [join=by tip];
  }
  { [start chain]
       \chainin (n33);
       \chainin (n34)    [join];
  }
  { [start chain]
       \chainin (n34);
       \chainin (n35)    [join=by tip];
  }
  { [start chain]
       \chainin (n35);
       \chainin (n36)    [join];
  }
  { [start chain]
       \chainin (n36);
       \chainin (n37)    [join=by tip];
  }
  { [start chain]
       \chainin (n37);
       \chainin (n38)    [join=by tip];
  }
  { [start chain]
       \chainin (n38);
       \chainin (n39)    [join=by tip];
  }
  { [start chain]
       \chainin (n39);
       \chainin (n310)    [join=by tip];
  }
  { [start chain]
       \chainin (n310);
       \chainin (n311)    [join];
  }
  { [start chain]
       \chainin (n311);
       \chainin (n312)    [join];
  }
  { [start chain]
       \chainin (n312);
       \chainin (n313)    [join=by tip];
  }
  { [start chain]
       \chainin (n313);
       \chainin (n314)    [join];
  }
  { [start chain]
       \chainin (n314);
       \chainin (n315)    [join];
  }
  { [start chain]
       \chainin (n315);
       \chainin (n316)    [join=by tip];
  }
  { [start chain]
       \chainin (n316);
       \chainin (n317)    [join=by tip];
  }
  { [start chain]
       \chainin (n311);
       \chainin (n28)    [join=by vh path];
       \chainin (n36)    [join=by {hv path,tip}];
  }
  { [start chain]
       \chainin (n312);
       \chainin (n213)    [join=by vh path];
       \chainin (n314)    [join=by {hv path,tip}];
  }
  { [start chain]
       \chainin (n34);
       \chainin (n19)    [join=by vh path];
       \chainin (n315)    [join=by {hv path,tip}];
  }
\end{tikzpicture}
}
\end{center}
%\end{minipage}

\section{Operationen}
\begin{table}[H]
	\centering
    \begin{tabular}{lll} %\hline
        Bezeichner & Typ	& Bedeutung\\ \hline
        \lstinline|(+)|        & \lstinline|a -> a -> a| & Addition          \\ 
        \lstinline|(-)|        & \lstinline|a -> a -> a| & Subtraktion       \\ 
        \lstinline|(*)|        & \lstinline|a -> a -> a| & Multiplikation \\ 
        \lstinline|negate|     & \lstinline|a -> a|      & Negation        \\ 
        \lstinline|abs|        & \lstinline|a -> a|      & Absolutbetrag     \\ 
        \lstinline|signum|     & \lstinline|a -> a|      & Vorzeichenbildung %\\ \hline
    \end{tabular}	
\caption{Für die Typen \lstinline|Int|, \lstinline|Integer|, \lstinline|Float| und \lstinline|Double|}
\end{table}
\begin{table}[H]
	\centering
    \begin{tabular}{lll} %\hline
        Bezeichner & Typ         & Bedeutung         \\ \hline
		\lstinline|succ| 	& \lstinline|a -> a| & Nachfolgerbildung \\
		\lstinline|pred| 	& \lstinline|a -> a| & Vorgängerbildung\\
		\lstinline|div| 	& \lstinline|a -> a -> a| & ganzzahlige Division, Ergebnis wird abgerundet\\
		\lstinline|mod| 	& \lstinline|a -> a -> a| & zur ganzzahligen Division \lstinline|div| gehörender Rest\\
		\lstinline|quot| 	& \lstinline|a -> a -> a| & ganzzahlige Division, Ergebnis wird Richtung 0 gerundet\\
		\lstinline|rem| 	& \lstinline|a -> a -> a| & zur ganzzahligen Division \lstinline|quot| gehörender Rest %\\ \hline
    \end{tabular}	
\caption{Für die Typen \lstinline|Int| und \lstinline|Integer|}
\end{table}
\begin{table}[H]
	\centering
    \begin{tabular}{lll} %\hline
        Bezeichner & Typ         & Bedeutung         \\ \hline
		\lstinline|(/)|	 	& \lstinline|a -> a -> a| & Division\\
		\lstinline|recip| 	& \lstinline|a -> a| & Kehrwertbildung\\
		\lstinline|(**)| 	& \lstinline|a -> a -> a| & Potenzieren\\
		\lstinline|sqrt| 	& \lstinline|a -> a| & Ziehen der Quadratwurzel %\\ \hline
    \end{tabular}
\caption{Für die Typen \lstinline|Float| und \lstinline|Double|}
\end{table}
\begin{table}[H]
	\centering
    \begin{tabular}{lll} %\hline
        Bezeichner & Typ         & Bedeutung         \\ \hline
		\lstinline!(||)!	& \lstinline|a -> a -> a| & Oder\\
		\lstinline|(&&)| 	& \lstinline|a -> a -> a| & Und\\
		\lstinline|not| 	& \lstinline|a -> a| & Verneinung %\\ \hline
    \end{tabular}
\caption{Für den Typ \lstinline|Bool|}
\end{table}
\newpage
\renewcommand*\thesubsection{\alph{subsection})}

\end{document}
