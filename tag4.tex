\subtitle{Tag vier - ein bisschen noch} 
\date{27.03.2014}

\begin{frame}[plain]
\titlepage
\end{frame}

\section{Lazy}
\begin{frame}[fragile]
\frametitle{Lazy}
\begin{block}{Betrachte folgende Funktion}
\begin{lstlisting}
rechne :: Double -> Double -> Double
rechne a b = if a > 10 
             then a + b
             else a
\end{lstlisting}
\end{block}
\only<1>{\begin{block}{Was erwartet ihr beim Aufruf von}
\lstinline|rechne 12 6|
\end{block}}
\only<2>{\begin{block}{und bei}
\lstinline|rechne 9 (10 / 0)|
\end{block}}
\end{frame}


\begin{frame}[fragile]
\frametitle{Lazy}
\begin{block}{Wir betrachten}
\begin{lstlisting}
prims :: [Integer]-> Int -> [Integer]
prims _ 0 		= []
prims (p:xs) i 	= (:) p $prims 
                [x|x<- xs, mod x p > 0] $i + 1
\end{lstlisting}
Terminiert die Funktion?
\end{block}
\pause 
\begin{block}{Terminiert die Funktion auch bei der Eingabe von}
\lstinline|primes [2..] 1|
\end{block}
\end{frame}

\frame{\frametitle{Lazy}
\begin{block}{\vspace*{-3ex}}
\begin{itemize}
  \item Haskell verwendet die Lazy-Evaluation für Ausdrücke 
  \item Lazy $\equiv$ Call-by-Need
  \item Dadurch sind Funktionen nicht strikt
\end{itemize}
\end{block}
}

\frame{\frametitle{Lazy - unendliche Listen}
\begin{block}{\vspace*{-3ex}}
\begin{itemize}
  \item Es werden vom Start an eine bestimmte Anzahl an Elemente erstellt
  \item Wenn weitere Elemente benötigt werden, werden diese neu Erstellt
  \item Wird immer nur ein Abschnitt benötigt wird der Start wieder gelöscht\\
  		stellt es euch als "`Ringpuffer"' vor
  \item Wenn jedoch alle Elemente benötigt werden $\to$ bis Speicher voll 
\end{itemize}
\end{block}
}

\frame{\frametitle{Lazy - Parameter und Ausdrücke}
\begin{block}{\vspace*{-3ex}}
\begin{itemize}
  \item Haskell verwendet Call-by-need
  \item Call-by-need ist Form es Call-by-name
\end{itemize}
\end{block}
}

\frame{\frametitle{Lazy - Call-by-name}
\begin{block}{\vspace*{-3ex}}
\begin{itemize}
  \item Ausdrücke werden nicht sofort ausgewertet sondern nur übergeben
  \item \lstinline|max (4 + 6) (10 / 0)|
  \only<2-5>{\item $\Rightarrow$ \lstinline|if (4 + 6) > (10 / 0) then (4 + 6) else (10 / 0)|}
  \only<3-5>{\item $\Rightarrow$ \lstinline|(4 + 6) > (10 / 0)|}
  \only<4-5>{\item $\Rightarrow$ \lstinline|10 > (10 / 0)|}
  \only<5>{\item $\Rightarrow \lightning$}
\end{itemize}
\end{block}
}


\frame{\frametitle{Lazy - Call-by-need}
\begin{block}{\vspace*{-3ex}}
\begin{itemize}
  \item Bei Call-by-need erweitert Call-by-name um Sharing
  \item Sharing: gleiche Ausdrücke werden nur einmal ausgewertet
\end{itemize}
\end{block}
}

\section{Funktionen höherer Ordnung (HOF)}
\subsection{Allgemeines zu HOF}
\frame{\frametitle{Funktionen höherer Ordnung (HOF)}
\begin{block}{\vspace*{-3ex}}
\begin{itemize}
	\item Funktionen können als Parameter nicht nur Ausdrücke sondern auch Funktionen erhalten
	\item Dieses wird im Funktionskopf angegeben
\end{itemize}
\begin{center}
\scalebox{0.6}{\begin{tikzpicture}[
      nonterminal/.style={
         rectangle,
         minimum size=5.0mm,
         very thick,
         draw=blue!50!black!50,
         top color=white,
         bottom color=blue!50!black!20,
         font=\itshape\scriptsize,
         %text height=1.5ex,
         %text depth=.25ex
      },
      terminal/.style={
         rounded rectangle,
         minimum size=5.0mm,
         very thick,draw=black!50,
         top color=white,bottom color=black!20,
         font=\ttfamily\scriptsize,
         %text height=1.5ex,
         %text depth=.25ex
      },
      skip loop/.style={
         to path={-- ++(0,#1) -| (\tikztotarget)}
      },
      point/.style={coordinate},>=stealth',thick,draw=black!50,
      tip/.style={->,shorten >=0.007pt},every join/.style={rounded corners},
      hv path/.style={to path={-| (\tikztotarget)}},
      vh path/.style={to path={|- (\tikztotarget)}},
      %text height=1.5ex,text depth=.25ex
    ]
    \matrix[column sep=5.0mm, row sep=3.0mm] {
 &  &  &  &  &  &  &  &  &  & \node (n111) [point] {}; &  &  &  &  &  &  &  &  & \\
 &  &  &  &  & \node (n26) [terminal] {->}; & \node (n27) [nonterminal] {Typ}; &  &  &  & \node (n211) [point] {}; &  &  &  &  & \node (n216) [terminal] {->}; & \node (n217) [nonterminal] {Typ}; &  &  & \\
\node (n31) [circle, draw, inner sep=0pt, minimum size=0.75ex] {}; & \node (n32) [nonterminal] {Funktionsname}; & \node (n33) [terminal] {::}; & \node (n34) [point] {}; & \node (n35) [point] {}; &  &  & \node (n38) [point] {}; & \node (n39) [terminal] {(}; & \node (n310) [point] {}; & \node (n311) [nonterminal] {HOF Typ}; & \node (n312) [terminal] {->}; & \node (n313) [point] {}; & \node (n314) [terminal] {)}; & \node (n315) [point] {}; &  &  & \node (n318) [point] {}; & \node (n319) [point] {}; & \node (n320) [circle, draw, inner sep=0pt, minimum size=0.75ex] {};\\
    };
  { [start chain]
       \chainin (n31);
       \chainin (n32)    [join=by tip];
  }
  { [start chain]
       \chainin (n32);
       \chainin (n33)    [join=by tip];
  }
  { [start chain]
       \chainin (n33);
       \chainin (n34)    [join];
  }
  { [start chain]
       \chainin (n34);
       \chainin (n35)    [join];
  }
  { [start chain]
       \chainin (n35);
       \chainin (n38)    [join];
  }
  { [start chain]
       \chainin (n26);
       \chainin (n27)    [join=by tip];
  }
  { [start chain]
       \chainin (n38);
       \chainin (n27)    [join=by {vh path,tip}];
  }
  { [start chain]
       \chainin (n26);
       \chainin (n35)    [join=by {hv path,tip}];
  }
  { [start chain]
       \chainin (n38);
       \chainin (n39)    [join=by tip];
  }
  { [start chain]
       \chainin (n39);
       \chainin (n310)    [join];
  }
  { [start chain]
       \chainin (n310);
       \chainin (n311)    [join=by tip];
  }
  { [start chain]
       \chainin (n311);
       \chainin (n312)    [join=by tip];
  }
  { [start chain]
       \chainin (n312);
       \chainin (n313)    [join];
  }
  { [start chain]
       \chainin (n313);
       \chainin (n314)    [join=by tip];
  }
  { [start chain]
       \chainin (n314);
       \chainin (n315)    [join];
  }
  { [start chain]
       \chainin (n315);
       \chainin (n318)    [join];
  }
  { [start chain]
       \chainin (n216);
       \chainin (n217)    [join=by tip];
  }
  { [start chain]
       \chainin (n318);
       \chainin (n217)    [join=by {vh path,tip}];
  }
  { [start chain]
       \chainin (n216);
       \chainin (n315)    [join=by {hv path,tip}];
  }
  { [start chain]
       \chainin (n318);
       \chainin (n319)    [join];
  }
  { [start chain]
       \chainin (n319);
       \chainin (n320)    [join=by tip];
  }
  { [start chain]
       \chainin (n313);
       \chainin (n211)    [join=by vh path];
       \chainin (n310)    [join=by {hv path,tip}];
  }
  { [start chain]
       \chainin (n319);
       \chainin (n111)    [join=by vh path];
       \chainin (n34)    [join=by {hv path,tip}];
  }
\end{tikzpicture}
}
\end{center}
\end{block}
}

\begin{frame}[fragile]
\frametitle{Funktionen höherer Ordnung (HOF)}
\begin{lstlisting}
filter :: (Int -> Bool) -> [Int] -> [Int]
filter do []     = 0
filter do (x:xs) | do x      = x : filter do xs
                 | otherwise = filter do xs
\end{lstlisting}
\end{frame}

\subsection{Funktionskomposition}
\begin{frame}[fragile]
\frametitle{Funktionskomposition}
\begin{block}{Wie vermeiden wir am besten "`Klammerungswirrwarr"'}
\lstinline|f (f (f (f (f (f (f (f (f x))))))))|
\end{block}
\pause
\begin{block}{Mit dem "`Punkt"'-Operator können wir Funktionen verbinden}
\lstinline|(f.f.f.f.f.f.f.f.f) x|
\end{block}
\pause
\begin{block}{Oder dem \lstinline|$|-Operator die Auswertungsreihenfolge verändern}
\lstinline|f $ f $ f $ f $ f $ f $ f $ f $ f x|
\end{block}
\end{frame}

\begin{frame}[fragile]
\frametitle{Funktionskomposition}
\begin{block}{Der "`Punkt"'-Operator ist definiert mit}
\begin{lstlisting}
(.) :: (b -> c) -> (a -> b) -> a -> c
(.) outerFunc innerFunc x = outerFunc (innerFunc x)
\end{lstlisting}
Das Resultat der inneren Funktion wird auf die äußere angewendet.
\end{block}
\end{frame}

\begin{frame}[fragile]
\frametitle{Funktionskomposition}
\begin{block}{Der \lstinline|$|-Operator ist definiert mit}
\begin{lstlisting}
($) :: (a -> b) -> a -> b
($) func x = func x
\end{lstlisting}
Die Funktion wird auf das Resultat von dem Ausdruck der "`rechts"' vom Operator steht angewandt.
\end{block}
\end{frame}

\section{Currying - für Fortgeschrittene}
\begin{frame}
\frametitle{Currying}
\begin{block}{\vspace*{-3ex}}
\begin{itemize}
  \item Currying bzw. Schönfinkeln ist das Zusammenfassen von Argumenten
  \item Wird in Sprachen und Kalkülen verwendet, in dem nur ein Argument erlaubt ist.\\
  		z.B. in der $\lambda$-Notation
  \item Die Form und Art des Zusammenfassens ist unterschiedlich 
\end{itemize}
\end{block}
\end{frame}

\begin{frame}
\frametitle{Currying in der Lambda-Notation}
\begin{block}{\vspace*{-3ex}}
\begin{itemize}
  \item $\lambda\;x\;y\;z\;.\;x\;y\;z$
  \item wird aufgespalten zu
  \item <2-7> $\lambda\;x\;.\;\lambda\;y\;.\;\lambda\;z\;.\;x\;y\;z$
  \item <3-7> wird ausgewertet mit den Argumenten $a\;b\;c$
  \item <4-7> $(\lambda\;x\;.\;\lambda\;y\;.\;\lambda\;z\;.\;x\;y\;z) a\;b\;c$
  \item <5-7> $(\lambda\;y\;.\;\lambda\;z\;.\;a\;y\;z) b\;c$
  \item <6-7> $(\lambda\;z\;.\;a\;b\;z) c$
  \item <7> $(a\;b\;c)$
\end{itemize}
\end{block}
\end{frame}

\begin{frame}
\frametitle{Currying in Haskell}
\begin{block}{\vspace*{-3ex}}
\begin{itemize}
  \item Auch wenn wir in Haskell Funktionen mehrere Argumente übergeben können
  \item Intern hat jede Funktion nur ein oder kein Argument!
\end{itemize}
\end{block}
\end{frame}

\begin{frame}
\frametitle{Currying in Haskell}
\begin{block}{Aufruf}
\lstinline|:t xor True|
\end{block}
\begin{block}{Ausgabe}
\lstinline|xor True :: Bool -> Bool|
\end{block}
\only<1>{\begin{block}{Aufruf}
\lstinline|(xor True) False|
\end{block}
\begin{block}{Ausgabe}
\lstinline|True|
\end{block}}
\only<2>{\begin{block}{Aufruf}
\lstinline|(xor True) False|
\end{block}
\begin{block}{Ausgabe}
\lstinline|False|
\end{block}}
\end{frame}

\begin{frame}[fragile]
\frametitle{Currying in Haskell}
\begin{block}{\vspace*{-3ex}}
\begin{itemize}
  \item Currying erleichtert das Arbeiten mit Funktionen höherer Ordnung
  \item Sehen wir uns folgendes Beispiel an
\end{itemize}
\end{block}
\begin{lstlisting}
map :: (a -> b) -> [a] -> [b]
map _ []     = []
map f (x:xs) = f x : map f xs
\end{lstlisting}
\begin{block}{Wie würdet ihr \lstinline|map| aufrufen um jedes Element einer Liste um 2 zu erhöhen?}
\only<2>{\lstinline|map (+ 2) [1..10]|}
\end{block}
\end{frame}

\begin{frame}
\frametitle{Currying in Haskell}
\begin{block}{\vspace*{-3ex}}
\begin{itemize}
  \item Soll Currying unterbunden werden, so muss die Anzahl der Argumente von Anfang an $\leq 1$ sein
  \item <1-3> \lstinline|f :: Int -> Int -> Int|
  \item <1-3> Hier für kommen Tupel ins Spiel
  \item <2-3> \lstinline|f' :: (Int, Int) -> Int|
  \item <2-3> Dieses "`Abändern"' ist jedoch nur bei eigenen Funktionen möglich
  \item <3> Funktionen können dies jedoch für uns übernehmen
\end{itemize}
\end{block}
\end{frame}

\begin{frame}[fragile]
\frametitle{Currying in Haskell}
\begin{block}{\lstinline|curry|}
\begin{lstlisting}
curry :: ((a, b) -> c) a -> b -> c
curry f x y = f (x, y)
\end{lstlisting}
\end{block}
\pause
\begin{block}{\lstinline|uncurry|}
\begin{lstlisting}
uncurry :: a -> b -> c -> ((a, b) -> c)
uncurry f t = f (fst t) (snd t)
\end{lstlisting}
\end{block}
\pause
\begin{block}{\vspace*{-3ex}}
\begin{itemize}
  \item \lstinline|fst t| gibt das erste Element aus \lstinline|t|
  \item \lstinline|snd t| gibt das zweite Element aus \lstinline|t|
\end{itemize}
\end{block}
\end{frame}

\section{Lambda Ausdrücke}
\begin{frame}
\frametitle{Anonyme Funktionen}
\begin{block}{\vspace*{-3ex}}
\begin{itemize}
  \item Haskell unterstützt anonyme Funktionen in Form von $\lambda$-Ausdrücken
  \item das $\lambda$-Symbol wird durch "`$\backslash$"' repräsentiert.\\ $\lambda x \leadsto$ \lstinline|\x|
  \item Aufbau: \\
  %\item {} $\backslash <Parameter_1> \ldots <Parameter_n>\; ->\; <Ausdruck>$
  \item durch das currying können $\lambda$-Ausdrücke mehrere Argumente besitzen
  \item es gelten alle bekannten Regeln für die $\lambda$-Notation
\end{itemize}
\begin{center}
\scalebox{0.8}{\begin{tikzpicture}[
      nonterminal/.style={
         rectangle,
         minimum size=5.0mm,
         very thick,
         draw=blue!50!black!50,
         top color=white,
         bottom color=blue!50!black!20,
         font=\itshape\scriptsize,
         %text height=1.5ex,
         %text depth=.25ex
      },
      terminal/.style={
         rounded rectangle,
         minimum size=5.0mm,
         very thick,draw=black!50,
         top color=white,bottom color=black!20,
         font=\ttfamily\scriptsize,
         %text height=1.5ex,
         %text depth=.25ex
      },
      skip loop/.style={
         to path={-- ++(0,#1) -| (\tikztotarget)}
      },
      point/.style={coordinate},>=stealth',thick,draw=black!50,
      tip/.style={->,shorten >=0.007pt},every join/.style={rounded corners},
      hv path/.style={to path={-| (\tikztotarget)}},
      vh path/.style={to path={|- (\tikztotarget)}},
      %text height=1.5ex,text depth=.25ex
    ]
    \matrix[column sep=5.0mm, row sep=3.0mm] {
 &  &  & \node (n14) [point] {}; &  &  &  &  & \\
\node (n21) [circle, draw, inner sep=0pt, minimum size=0.75ex] {}; & \node (n22) [terminal] {\textbackslash}; & \node (n23) [point] {}; & \node (n24) [nonterminal] {Parameter}; & \node (n25) [terminal] {.}; & \node (n26) [point] {}; & \node (n27) [terminal] {->}; & \node (n28) [nonterminal] {Ausdruck}; & \node (n29) [circle, draw, inner sep=0pt, minimum size=0.75ex] {};\\
    };
  { [start chain]
       \chainin (n21);
       \chainin (n22)    [join=by tip];
  }
  { [start chain]
       \chainin (n22);
       \chainin (n23)    [join];
  }
  { [start chain]
       \chainin (n23);
       \chainin (n24)    [join=by tip];
  }
  { [start chain]
       \chainin (n24);
       \chainin (n25)    [join=by tip];
  }
  { [start chain]
       \chainin (n25);
       \chainin (n26)    [join];
  }
  { [start chain]
       \chainin (n26);
       \chainin (n27)    [join=by tip];
  }
  { [start chain]
       \chainin (n27);
       \chainin (n28)    [join=by tip];
  }
  { [start chain]
       \chainin (n28);
       \chainin (n29)    [join=by tip];
  }
  { [start chain]
       \chainin (n26);
       \chainin (n14)    [join=by vh path];
       \chainin (n23)    [join=by {hv path,tip}];
  }
\end{tikzpicture}
}
\end{center}
\end{block}
\end{frame}

\begin{frame}[fragile]
\frametitle{Beispiele - Lambda-Ausdrücke}
\begin{lstlisting}
plus = \x -> \y -> x + y

istKleiner = \x -> \y ->  x < y

g = \x -> \y -> (\y -> \x -> (y,x)) y x
\end{lstlisting}
\end{frame}

\begin{frame}[fragile]
\frametitle{Beispiele - Lambda-Ausdrücke}
\begin{lstlisting}
-- y "Operator"
y f = f (y f)

fac = y (\f n -> if n > 0 then n * f (n - 1) else 1)
\end{lstlisting}
\only<2-3>{
\begin{exampleblock}{Aufruf}
\lstinline|fac 5|
\end{exampleblock}}
\only<3>{
\begin{exampleblock}{Ausgabe}
\lstinline|120|
\end{exampleblock}}
\end{frame}

%\begin{frame}
%\begin{center}
%\rotatebox{30}{\textcolor{red}{{\fontsize{100}{40} \selectfont Ende}}}
%\end{center}
%\end{frame}