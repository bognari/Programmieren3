%%This is a very basic article template.
%%There is just one section and two subsections.
\documentclass{article}

\usepackage[ngerman]{babel}
\usepackage[utf8x]{inputenc}

\usepackage{geometry}
\geometry{a4paper,left=24mm,right=18mm, top=2cm, bottom=3cm}

\usepackage{fancyhdr}
\pagestyle{fancy}

\headheight 35pt
\lhead{}
\rhead{}
\chead{Programmieren für Fortgeschrittene \\ Aufgabenblatt II \\}
\cfoot{Programmieren für Fortgeschrittene \\ Aufgabenblatt II \\}
\lfoot{\thepage}
\rfoot{\today}


%\title{Programmieren für Fortgeschrittene}
%\subtitle{Aufgaben Blatt 1}

%\author{Stephan Mielke}


%\date{21.03.2013}

\usepackage{tikz}
\usetikzlibrary{shapes,arrows,positioning}

\usepackage{amsmath}
\usepackage{amssymb}
\usepackage{amsthm}
\usepackage{MnSymbol}

\usepackage{float}

\usepackage{listings}
\usepackage{color}


\lstset{language=Haskell,
numbers=none,
keywordstyle=\color{green!40!black},
commentstyle=\color{magenta!40!black},
identifierstyle=\color{blue},
stringstyle=\color{orange},
otherkeywords={$, \{, \}, \[, \]},
frame=single,
tabsize=2
}


\begin{document}
\section{Mengen als Listen}
Mengen können in Haskell durch Listen repräsentiert werden. Allerdings unterscheiden
sich Mengen und Listen in zwei Punkten:
\begin{itemize}
  \item Die Elemente einer Liste besitzen eine Reihenfolge, die Elemente von Mengen
nicht.
\item Derselbe Wert darf in einer Liste mehrfach vorkommen, in einer Menge nicht.
\end{itemize}
Zur Darstellung von Mengen verwenden wir daher nur solche Listen, deren Elemente
streng monoton wachsen. Die Reihenfolge der Elemente ist damit festgelegt und eine
Dopplung von Werten ist nicht möglich.\\

Es sollen nun Mengenoperationen auf Basis von Listen implementiert werden. Geben Sie
für folgende Funktionen deren allgemeinste Typen an. Implementieren Sie anschließend
die Funktionen. Sie können davon ausgehen, dass als Argumente übergebene Mengendarstellungen
immer die oben genannte Eigenschaft erfüllen. Sie müssen aber sicher
stellen, dass Sie keine ungültigen Mengendarstellungen als Resultate generieren.
\subsection{}
Die Funktion $istElement$ entscheidet, ob eine Wert in einer
Menge enthalten ist und implementiert damit die Relation 2. Es ergibt sich z.B.
\begin{align*}
4 \;'istElement'\; [1, 3, 5, 7] & \leadsto False\\
5 \;'istElement'\; [1, 3, 5, 7] & \leadsto True
\end{align*}
\subsection{}
Die Funktion $istTeilmenge$ entscheidet, ob eine Menge Teilmenge einer anderen
Menge ist und implementiert damit die Relation $\subseteq$. Es ergibt sich z.B.
\begin{align*}
[1, 3] \;'istTeilmenge'\; [1, 2, 3, 4, 5] & \leadsto True\\
[1, 3, 6] \;'istTeilmenge'\; [1, 2, 3, 4, 5] & \leadsto False
\end{align*}
\subsection{}
Die Funktion $vereinigung$ bestimmt die Vereinigungsmenge zweier Mengen und
implementiert damit die Funktion $\cup$. Es ergibt sich z.B.
\begin{align*}
[1, 2, 4] \;'vereinigung'\; [3, 4, 5] \leadsto [1, 2, 3, 4, 5]
\end{align*}
\subsection{}
Die Funktion $schnitt$ bestimmt die Schnittmenge zweier Mengen und implementiert
damit die Funktion $\cap$. Es ergibt sich z.B.
\begin{align*}
[1, 2, 4] \;'schnitt'\; [3, 4, 5] \leadsto [4]
\end{align*}
\section{Funktionen als Listen}
Wir stellen Funktionen als Listen von Paaren dar. Die Liste zu einer Funktion $f$ enthält
für jedes $x$, für das $f (x)$ definiert ist, das Paar $(x, f(x))$. Die Paare sind innerhalb der Liste
aufsteigend nach den Funktionsargumenten sortiert. Kein Paar kommt mehrfach vor.
\subsection{}
Programmieren Sie eine Haskell-Funktion $resultat$, welche zu
einer als Liste dargestellten Funktion $f$ und einem Wert $x$ des Definitionsbereichs
den entsprechenden Funktionswert $f(x)$ liefert. Ist $f(x)$ nicht definiert, soll $resultat$
mit einer Fehlermeldung abbrechen. Es ergibt sich z.B.
\begin{align*}
resultat\; [(-2, 4), (-1, 1), (0, 0), (1, 1), (2, 4)] 2 \leadsto 4
\end{align*}
\subsection{}
Häufig ist es nicht erwünscht, dass Funktionen mit Fehlermeldungen abbrechen.
Stattdessen werden oft Werte des Datentyps $Maybe$ als Funktionsresultate verwendet.
$Maybe$ ist im Prelude wie folgt definiert:
\begin{lstlisting}
data Maybe wert = Nothing | Just wert
\end{lstlisting}
Der Wert $Nothing$ signalisiert einen Fehler, während ein Wert $Just\;x$ die erfolgreiche
Berechnung des Wertes x darstellt. Modifizieren Sie die Funktion $resultat$ zu einer
Funktion $evtlResultat$, welche nicht mit Fehlermeldungen abbricht, sondern $Maybe$-
Werte liefert. Es ergibt sich z.B.
\begin{align*}
evtlResultat\; [(-2, 4), (-1, 1), (0, 0), (1, 1), (2, 4)]\; 3 & \leadsto Nothing\\
evtlResultat\; [(-2, 4), (-1, 1), (0, 0), (1, 1), (2, 4)]\; 2 & \leadsto Just\;4
\end{align*}
\subsection{}
Programmieren Sie eine Haskell-Funktion $urbilder$, die zu einer gegebenen Funktion
$f$ und einem gegebenen Wert $y$ die Menge $\{x | f(x) = y\}$ berechnet. Diese Menge
soll wie in Aufgabe 1 beschrieben dargestellt werden. Es ergibt sich z.B.
\begin{align*}
urbilder\; [(-2, 4), (-1, 1), (0, 0), (1, 1), (2, 4)]\; 4 \leadsto [-2, 2] 
\end{align*}
\subsection{}
Programmieren sie eine Haskell-Funktion $echteArgumente$, welche zu einer gegebenen
Funktion $f$ die Menge aller $x$ liefert, für die es einen Funktionswert $f(x)$ gibt.
Verwenden Sie wieder die Mengendarstellung aus Aufgabe 1. Es ergibt sich z.B.
\begin{align*}
echteArgumente\; [(-2, 4), (-1, 1), (0, 0), (1, 1), (2, 4)] \leadsto [-2,-1, 0, 1, 2]
\end{align*}
\subsection{}
Wir betrachten nun Funktionen, deren Definitions- und Wertebereich identisch sind.
Ein Wert $x$ ist Fixpunkt einer solchen Funktion $f$ , wenn $f(x) = x$ gilt. Programmieren
Sie eine Haskell-Funktion $fixpunkte$, die zu einer gegebenen Funktion die Menge
aller Fixpunkte berechnet. Diese Menge soll wieder wie in Aufgabe 1 beschrieben
dargestellt werden. Es ergibt sich z.B.
\begin{align*}
fixpunkte \;[(-2, 4), (-1, 1), (0, 0), (1, 1), (2, 4)] \leadsto [0, 1]
\end{align*}
\subsection{}
Programmieren Sie eine Haskell-Funktion $funKomposition$, die aus zwei Funktionen
$f$ und $g$ deren Komposition $f \circ g$ berechnet. Es ergibt sich z.B.
\begin{align*}
[(1, 0), (2, 1)] \;'funKomposition' \;[(-2, 4), (-1, 1), (0, 0), (1, 1), (2, 4)] \leadsto [(-1, 0), (1, 0)]
\end{align*}

\section{Relationen als Listen}
Eine Relation ist eine Menge von Paaren. Wir wollen solche Mengen von Paaren in Haskell
als Listen von Paaren darstellen. Wir wollen allerdings keine besonderen Anforderungen
an diese Listen stellen. Paare können in einer Liste in beliebiger Reihenfolge stehen und
dürfen auch mehrfach auftreten.\\	

Die Funktionskomposition $\circ$ kann zu einer Relationskomposition verallgemeinert werden.
Es seien $R_1 \subseteq A \times B$ und $R_2 \subseteq B \times C$ zwei Relationen. Die Komposition $R_2 \circ R_1 \subseteq A \times C$
ist wie folgt definiert:
\begin{align*}
R_2 \circ R_1 = \left\{(a, c)\;| \exists\;b : (a, b) \in R_1 \wedge (b, c) \in R_2 \right\}
\end{align*}
Definieren Sie eine Funktion $relKomposition$, die die Relationskomposition umsetzt.
Verwenden Sie dafür eine List-Comprehension. Es ergibt sich z.B.
\begin{align*}
[(4,-2), (1,-1), (0, 0), (1, 1), (4, 2)] \;'relKomposition' \;[(0, 1), (1, 2)] \leadsto [(0,-1), (0, 1)]
\end{align*}
\section{Instanziierung von Typklassen}
Wir betrachten den im letzten Aufgabenblatt vorgestellten Typ $Ausdruck$ für
arithmetische Ausdrücke. Dieser ist wie folgt definiert:
\begin{lstlisting}
data Ausdruck = Konstante Integer
  | Variable String
  | Summe Ausdruck Ausdruck
  | Differenz Ausdruck Ausdruck
  | Produkt Ausdruck Ausdruck
\end{lstlisting}
Beachten Sie, dass wir diesmal die Klausel \textbf{deriving (Show)} weggelassen haben.
\subsection{}
Machen Sie $Ausdruck$ zu einer Instanz von $Eq$. Zwei Ausdrücke
sollen nur dann als gleich gelten, wenn sie strukturgleich sind. Die Ausdrücke
$x$ und $x + 0$ gelten z.B. als ungleich.
\subsection{}
Machen Sie $Ausdruck$ zu einer Instanz von $Show$. Es genügt dabei, die Methode $show$
zu implementieren, welche $Ausdr"ucke$ in Zeichenketten umwandelt. Ihre $show$-
Implementierung soll die gleichen Resultate liefern, wie die $show$-Implementierung,
welche durch $deriving(Show)$ automatisch erzeugt wird. Es ergibt sich z.B.
\begin{align*}
show\; (Differenz\; (Konstante\; 5) \;(Variable \;"x"))\\
\leadsto "Differenz\; (Konstante\; 5)\; (Variable\; \backslash "x\backslash")"
\end{align*}
Beachten Sie, dass Anführungsstriche (") in Zeichenkettenliteralen als $\backslash$" dargestellt
werden.
\section{eigene Klasse}
Wie betrachten folgende Klassendeklaration:
\begin{lstlisting}
class Spiegelbar wert where
	gespiegelt :: wert -> wert
\end{lstlisting}
Instanzen von $Spiegelbar$ sollen diejenigen Strukturen sein, die man spiegeln kann. Die
Funktion $gespiegelt$ liefert zu einem Wert die entsprechende gespiegelte Variante.
\subsection{}
Ein Beispiel für $spiegelbare$ Strukturen sind Listen. Die gespiegelte Liste zu einer
Liste $l$ enthält die Elemente von $l$ in umgekehrter Reihenfolge. Instanziieren Sie
$Spiegelbar$ entsprechend.
\subsection{} 
Der Typ blattmarkierter Binärbäume lässt sich folgendermaßen definieren:
\begin{lstlisting}
data Baum el = Blatt el | Verzweigung (Baum el) (Baum el)
\end{lstlisting}
Auch blattmarkierte Bäume lassen sich spiegeln, nämlich indem man für jeden
inneren Knoten dessen beide Unterbäume vertauscht. Schreiben Sie eine entsprechende
instance-Deklaration.

\section{Babylonisches Wurzelziehen}
Das Babylonische Wurzelziehen ist eine Methode zur Berechnung von Quadratwurzeln.
In verallgemeinerter Form kann es auch zum Ziehen anderer Wurzeln benutzt werden.
Es seien $x$ und $y$ reelle Zahlen mit $x > 0$ und $y > 0$. Wir definieren nun eine Folge $\langle z_i \rangle$
wie folgt:
\begin{itemize}
  \item Das erste Folgenglied $z_0$ ist eine beliebige positive reelle Zahl.
  \item Für jedes weitere Folgenglied $z_{n+1}$ gilt: \\
  \begin{align*}
  z_{n+1} = \frac{(x-1) \cdot z_n^x + y}{x \cdot z_n^{(x-1)}}
  \end{align*}
\end{itemize}
Die Folge $\langle z_i \rangle$ nähert sich $\sqrt[x]{y}$ an. Man berechnet nun solange Folgenglieder, bis der
Abstand der letzten beiden berechneten Glieder kleiner als eine vorgegebene Genauigkeitsschranke
$\epsilon$ ist, wobei $\epsilon$ eine positive reelle Zahl ist. Schreiben Sie eine Funktion
$wurzel$, die die Zahlen $\epsilon$, $x$, $y$ und $z_0$ in dieser Reihenfolge als Argumente erwartet und
daraus die entsprechende Wurzelnäherung berechnet.

\section{abstrakter Datentyp für Mengen}
Implementieren Sie einen abstrakten Datentyp $Menge$. Verwenden Sie intern die Darstellung
von Mengen als aufsteigend sortierte Listen aus Aufgabe 1.
Der abstrakte Datentyp soll folgende Operationen bereit stellen:

\begin{table}[H]
    \begin{tabular}{lp{5.5cm}p{6.5cm}} %\hline
        Bezeichner & Typ & Bedeutung         \\ \hline
		leer & $Menge\;a$ & die leere Menge \\
		einfuegen & $a \to Menge\;a \to Menge\;a$ & Einfügen eines Elements in eine Menge\\
		loeschen & $a \to Menge\;a \to Menge\;a$ & Löschen einesWerts aus einer Menge\\
		vereinigung & $Menge\;a \to Menge\;a \to Menge\;a$ & Vereinigung zweier Mengen\\
		schnitt & $Menge\;a \to Menge\;a \to Menge\;a$ & Schnitt zweier Mengen\\
		differenz & $Menge\;a \to Menge\;a \to Menge\;a$ & Differenz zweier Mengen\\
		istLeer & $Menge\;a \to Bool$ & Test, ob eine Menge leer ist\\
		istElement & $a \to Menge\;a \to Boola$ & Test, ob einWert Element einer Menge ist\\
		istTeilmenge & $Menge\;a \to Menge\;a \to Bool$ & Test, ob eine Menge Teilmenge einer \newline zweiten Menge ist\\
		istEchteTeilmenge & $Menge\;a \to Menge\;a \to Bool$ & Test, ob eine Menge echte Teilmenge einer \newline zweiten Menge ist\\
		minimalesElement & $Menge\;a \to Maybe\;a$ \newline $Menge\;a \to a$ & das minimale Element einer Menge\\
		maximalesElement & $Menge\;a \to Maybe\;a$ \newline $Menge\;a \to a$ & das maximale Element einer Menge
	\end{tabular}	
\end{table}
Die Funktion $loeschen$ soll die ursprüngliche Menge zurückgegeben, wenn der angegebene
Wert nicht Element dieser Menge ist. Die Funktionen $minimalesElement$ und
$maximalesElement$ sollen mit einer Fehlermeldung abbrechen oder den Typ $Maybe$ verwenden und $Nothing$ zurückgeben, wenn die gegebene Menge
leer ist.

\subsection{}
Legen Sie ein Modul $Menge$ für den abstrakten Datentyp
an. Fügen Sie in dieses Modul die Deklaration des Datentyps $Menge$ sowie die
Typsignaturen der Mengenoperationen ein. Legen Sie eine geeignete Exportliste
an. Es soll außerdem möglich sein, Mengen auf Gleichheit zu testen und in der
üblichen Notation $\{x_1, \ldots , x_n\}$ auszugeben. Sorgen Sie für entsprechende Typklassen Instanziierungen.

\subsection{}
Implementieren Sie die Mengenoperationen. Sie können dazu Ihre Lösung von
Aufgabe 1 verwenden.

\section{Funktionen höherer Ordnung}
\subsection{}
Programmieren Sie eine Haskell-Funktion 
\begin{lstlisting}
summe :: (Num zahl) => [zahl] -> zahl
\end{lstlisting}
welche die Summe der Elemente einer Liste von Zahlen berechnet. Verwenden Sie
dazu die vordefinierte Funktion $foldr$.\\
\begin{lstlisting}
foldr :: (a -> b -> b) -> b -> [a] -> b
\end{lstlisting}
$a$ und $b$ können gleiche Typen sein.\\
alles Weitere finden Sie unter: www.haskell.org/hoogle
\subsection{}
Programmieren Sie eine Haskell-Funktion 
\begin{lstlisting}
produkte :: (Num zahl) => [zahl] -> [zahl] -> [zahl]
\end{lstlisting}
welche zu zwei Listen $[x_1, x_2, \ldots , x_n]$ und $[y_1, y_2, \ldots, y_n]$ die Liste der Produkte
$[x_1 \cdot y_1, x_2 \cdot y_2, \ldots, x_n \cdot y_n]$
berechnet. Geben Sie zwei Lösungen an. Die erste Lösung soll die vordefinierten
Funktionen $zip$, $map$ und $uncurry$ verwenden, die zweite Lösung die vordefinierte
Funktion $zipWith$.
\begin{lstlisting}
zip :: [a] -> [b] -> [(a,b)]
map :: (a -> b) -> [a] -> [b]
uncurry :: (a -> b -> c) -> (a, b) -> c

zipWith :: (a -> b -> c) -> [a] -> [b] -> [c]
\end{lstlisting}
\subsection{}
Wir stellen einen Vektor als Liste seiner kartesischen Koordinaten dar:
\begin{lstlisting}
type Vektor koord = [koord]
\end{lstlisting}
Programmieren Sie eine Haskell-Funktion
\begin{lstlisting}
skalarProdukt :: (Num koord) => Vektor koord -> Vektor koord -> koord
\end{lstlisting}
welche zu zwei Vektoren $(x_1, \ldots , x_n)$ und $(y_1, \ldots ,y_n)$ das Skalarprodukt
$\sum^n_{i=1}{x_i \cdot y_i}$ berechnet. Verwenden Sie die in den Teilaufgaben 1 und 2 definierten Funktionen.
\subsection{}
Wir stellen eine Matrix als Liste ihrer Zeilen dar:
\begin{lstlisting}
type Matrix koord = [Vektor koord]
\end{lstlisting}
Programmieren Sie eine Haskell-Funktion
\begin{lstlisting}
lineareAbb :: (Num koord) => Matrix koord -> (Vektor koord -> Vektor koord)
\end{lstlisting}
die zu einer gegebenen $(n \times m)$-Matrix die durch sie dargestellte lineare Abbildung
ermittelt, welche m-dimensionale in n-dimensionale Vektoren überführt.

\section{Falten von Listen}
\subsection{}
Im Prelude sind die folgenden Funktionen definiert:
\begin{lstlisting}
(++)   :: [el] -> [el] -> [el]
concat :: [[el]] -> [el]
length :: [el] -> Int
filter :: (el -> Bool) -> [el] -> [el]
\end{lstlisting}
Geben Sie zu jeder dieser Funktionen einen äquivalenten Haskell-Ausdruck an,
der keine \textbf{let}-Ausdrücke enthält und höchstens die folgenden vordefinierten
Funktionen benutzt:
\begin{lstlisting}
succ  :: Int -> Int
(:)   :: el -> [el] -> [el]
foldr :: (el -> accu -> accu) -> accu -> [el] -> accu
\end{lstlisting}
\subsection{}
Geben Sie einen zu der Funktion $concat$ äquivalenten Haskell-Ausdruck an, der keine
\textbf{let}-Ausdrücke enthält und höchstens die folgenden vordefinierten Funktionen
benutzt:
\begin{lstlisting}
(:)   :: el -> [el] -> [el]
foldl :: (accu -> el -> accu) -> accu -> [el] -> accu
\end{lstlisting}
\section{Varianten von $map$ und $foldr$ für Bäume}
Die Listenfunktionen $map$ und $foldr$ sind im $Prelude$ vordefiniert. Varianten dieser
Funktionen gibt es auch für viele andere algebraische Datentypen, so z.B. auch für den
Typ $Baum$. Dieser wird wie folgt definiert:
\begin{lstlisting}
data Baum el = Blatt el | Verzweigung (Baum el) (Baum el)
\end{lstlisting}
Die Typsignaturen der $map$- und der $foldr$-Variante für Baum lauten folgendermaßen:
\begin{lstlisting}
baumMap  :: (el -> el') -> Baum el -> Baum el'
baumFold :: (el -> accu) -> (accu -> accu -> accu) -> Baum el -> accu
\end{lstlisting}
Einen Baum $baumMap \;fun \;baum$ erhält man, indem man in $baum$ jede Blattmarkierung $el$
durch $fun\;el$ ersetzt. Einen Baum $baumFold \;accuBlatt \;accuVerzweigung \;baum$ erhält man,
indem man in $baum$ jedes Vorkommen des Datenkonstruktors $Blatt$ durch $accuBlatt$ und
jedes Vorkommen des Datenkonstruktors $Verzweigung$ durch $accuVerzweigung$ ersetzt.
Zum Beispiel liefert
\begin{align*}
baumFold (\lambda el \to [el]) (++) ((Blatt \;2 \;'Verzweigung' \;Blatt \;3) 'Verzweigung' \;Blatt\;5)
\end{align*}
die Liste $([2] ++ [3]) ++ [5]$, also $[2, 3, 5]$. Allgemein stellt $baumFold \;(\lambda el \to [el]) \;(++)\; baum$
die Liste aller Blattmarkierungen von $baum$ dar.
\subsection{}
Implementieren Sie die beschriebenen Haskell-Funktionen $baumMap$ und $baumFold$.
\subsection{}
Programmieren Sie eine Haskell-Funktion 
\begin{lstlisting}
hoehe :: Baum el -> Int
\end{lstlisting}
welche die Höhe
eines Baums bestimmt. Die Höhe eines Baums, der nur aus einem Blatt besteht, sei
dabei $0$, nicht $1$. Verwenden Sie für Ihre Implementierung keine \textbf{explizite Rekursion},
sondern die Funktion $baumFold$.

\section{Sortieren mit variabler Ordnung}
Es soll eine Haskell-Funktion $mergeSort$ programmiert werden, welche nach verschiedenen
Kriterien sortieren kann. Dazu soll ihr neben der zu sortierenden Liste eine
Vergleichsfunktion als Argument übergeben werden. Die Vergleichsfunktion soll eine
irreflexive Ordnung darstellen.
\subsection{}
Welchen Typ hat eine $Vergleichsfunktion$ für einen Typ $el$?\\
Welchen Typ hat dann die Funktion $mergeSort$, die als zusätzliches Argument die
zu verwendende Vergleichsfunktion erhält?
\subsection{}
Implementieren Sie die Funktion $mergesort$, welche eine Vergleichsfunktion sowie die zu sortierende Liste als Argumente bekommt.\\
als Tipp die "`divide and conquer"' Implementierung.
\begin{lstlisting}
dc :: ([a] -> Bool) -> ([a] -> [b]) -> ([a] -> ([a],[a])) -> 
  (([b],[b]) -> [b]) -> [a] -> [b]
dc simple solve split combine q 
      | simple q 	= solve q
      | otherwise 	= 	combine (l1, l2)
        where 
          (p1, p2) = split q
          l1 = dc simple solve split combine p1
          l2 = dc simple solve split combine p2	
\end{lstlisting}
\subsection{}
Betrachten Sie den folgenden Datentyp zur Repräsentation von Büchern:
\begin{lstlisting}
data Buch = Buch {
    autor :: String,
    titel :: String,
    jahr :: Integer,
    verlag :: String
  } deriving (Show)
\end{lstlisting}
Programmieren Sie die Vergleichsfunktionen $autorVergleich$, $titelVergleich$ und
$jahrVergleich$, mit denen eine Liste von Büchern nach Autor, Titel bzw. Erscheinungsjahr
sortiert werden kann.
\subsection{}
Programmieren Sie eine Haskell-Funktion $und$ zur Komposition von Vergleichsfunktionen.
Es seien $vergleich_1$ und $vergleich_2$ zwei Vergleichsfunktionen. Dann soll
$vergleich_1\;'und'\;vergleich_2$ eine Vergleichsfunktion sein, mittels der zuerst nach dem
durch $vergleich_1$ dargestellten Kriterium und bei äquivalenten Einträgen nach dem
durch $vergleich_2$ dargestellten Kriterium sortiert wird. So soll z.B.
\begin{align*}
mergeSort (autorVergleich \;'und' \;titelVergleich) \;<[Liste]>
\end{align*}
Bücher zunächst nach Autor und bei gleichem Autor nach Titel sortieren.
\end{document}
