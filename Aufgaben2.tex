\documentclass[
  10pt,                   % Schriftgröße
  DIV12,
  german,                 % für Umlaute, Silbentrennung etc.
  oneside,                % einseitiges Dokument
  %titlepage,              % es wird eine Titelseite verwendet
  parskip=half,           % Abstand zwischen Absätzen (halbe Zeile)
  headings=normal,        % Größe der Überschriften verkleinern
  captions=tableheading,  % Beschriftung von Tabellen unterhalb ausgeben
  %final                   % Status des Dokuments (final/draft)
]{scrartcl}      

\usepackage[utf8x]{inputenc}
\usepackage[T1]{fontenc}
\usepackage[ngerman]{babel}
\usepackage{stmaryrd}
\usepackage{amsfonts}
\usepackage{amssymb}
\usepackage{amsmath}
\usepackage{microtype}

\usepackage{listings}
\usepackage{color}
\usepackage{pxfonts}

\definecolor{mygreen}{RGB}{51,141,120}
\definecolor{myblue}{RGB}{0,128,180}
\definecolor{myviolet}{RGB}{118,0,118}

\lstset{ %
  language=Haskell,
  backgroundcolor=\color{white},         % choose the background color
  basicstyle=\ttfamily\footnotesize,     % size of fonts used for the code
  numbers=none,
  breaklines=true,                       % automatic line breaking only at whitespace
  captionpos=b,                          % sets the caption-position to bottom
  commentstyle=\color{mygreen},    % comment style
  escapeinside={\%*}{*)},                % if you want to add LaTeX within your code
  keywordstyle=\color{myblue}\bfseries, % keyword style
  stringstyle=\color{myviolet},    % string literal style
  frame=single,
  tabsize=2
}
\usepackage{tikz}
\usetikzlibrary{
  arrows,
  shapes.misc,
  shapes.arrows,
  chains,
  matrix,
  positioning,
  scopes,
  decorations.pathmorphing,
  shadows,
  backgrounds
}


\usepackage{float}
\usepackage{geometry}

\geometry{verbose,                     % zeigt die eingestellten Parameter beim
                                       % Latexlauf an
      paper=a4paper,                   % Papierformat
      top=25mm,                        % Rand oben
      left=25mm,                       % Rand links
      right=25mm,                      % Rand rechts
      bottom=40mm,                     % Rand unten
      pdftex                           % schreibt das Papierformat in die
                                       % Ausgabe damit Ausgabeprogramm
                                       % Papiergröße erkennt
}

\usepackage[%
  automark,
  headsepline,                %% Separation line below the header
  footsepline,               %% Separation line above the footer
  markuppercase
]{scrpage2}

\automark[subsection]{section}
\pagestyle{scrheadings}
\renewcommand*\thesubsection{\roman{subsection})}


%\lehead{\bfseries\pagemark}    %% Top left on even pages
%\lohead{\bfseries\headmark}    %% Top left on odd pages
%\rehead{\bfseries\headmark}    %% Top right on even pages
%\rohead{\bfseries\pagemark}    %% Top right on odd pages
\chead{Programmieren für Fortgeschrittene \\ Aufgabenblatt 2} 

\begin{document}
\section{Rekursion über Listen}
Die folgenden Teilaufgaben sollen ohne Verwendung vordefinierter Bibliotheksfunktionen gelöst werden. Berücksichtigen Sie, dass die zu programmierenden Funktionen nicht für alle Eingaben sinnvoll definiert sind. In diesen Fällen sollen entsprechende Fehlermeldungen ausgegeben werden.
\subsection{}	
Programmieren Sie eine Haskell-Funktion \lstinline|elementAnIdx :: Integer -> [el] -> el|, die zu einem Index \lstinline|i| und einer Liste das i-te Element dieser Liste bestimmt. Das erste Element einer nichtleeren Liste habe dabei den Index $0$. Es ergibt sich zum Beispiel:\\
\lstinline|elementAnIdx 2 [3, 4, 5, 6] = 5|
\subsection{}
Programmieren Sie eine Haskell-Funktion \lstinline|snoc :: [el] -> el -> [el]|, die an eine Liste ein weiteres Element anhängt. Es ergibt sich zum Beispiel:\\
\lstinline|snoc [1, 2, 3] 5 = [1, 2, 3, 5]|
\subsection{}
Programmieren Sie eine Haskell-Funktion \lstinline|rueckwaerts :: [el] -> [el]|, welche eine Liste umdreht, die Elemente der Liste also in umgekehrter Reihenfolge zurück gibt.
\subsection{}
Programmieren Sie eine Haskell-Funktion \lstinline|praefix :: Integer -> [el] -> [el]|, die für eine gegebene Zahl $n$ und eine Liste $l$ die Liste der ersten $n$ Elemente von $l$ zurück gibt.
\subsection{}
Programmieren Sie eine Haskell-Funktion \lstinline|suffix :: Integer -> [el] -> [el]|, die für eine Zahl $n$ und eine Liste $l$ die Liste der letzten $n$ Elemente von $l$ zurückgibt.
\newpage
\section{Mengen als Listen}
Mengen können in Haskell durch Listen repräsentiert werden. Allerdings unterscheiden
sich Mengen und Listen in zwei Punkten:
\begin{itemize}
  \item Die Elemente einer Liste besitzen eine Reihenfolge, die Elemente von Mengen
nicht.
\item Derselbe Wert darf in einer Liste mehrfach vorkommen, in einer Menge nicht.
\end{itemize}
Zur Darstellung von Mengen verwenden wir daher nur solche Listen, deren Elemente
streng monoton wachsen. Die Reihenfolge der Elemente ist damit festgelegt und eine
Dopplung von Werten ist nicht möglich.\\

Es sollen nun Mengenoperationen auf Basis von Listen implementiert werden. Geben Sie
für folgende Funktionen deren allgemeinste Typen an. Implementieren Sie anschließend
die Funktionen. Sie können davon ausgehen, dass als Argumente übergebene Mengendarstellungen
immer die oben genannte Eigenschaft erfüllen. Sie müssen aber sicherstellen, dass Sie keine ungültigen Mengendarstellungen als Resultate generieren.
\subsection{}
Die Funktion \lstinline|istElement| entscheidet, ob eine Wert in einer
Menge enthalten ist und implementiert damit die Relation $\in$. Es ergibt sich z.B.
\begin{center}
\begin{tabular}{ll}
\lstinline|4 `istElement` [1, 3, 5, 7]| & $\leadsto$ \lstinline|False|\\
\lstinline|5 `istElement` [1, 3, 5, 7]| & $\leadsto$ \lstinline|True|
\end{tabular}
\end{center}
\subsection{}
Die Funktion \lstinline|istTeilmenge| entscheidet, ob eine Menge Teilmenge einer anderen Menge ist und implementiert damit die Relation $\subseteq$. Es ergibt sich z.B.
\begin{center}
\begin{tabular}{ll}
\lstinline|[1, 3] `istTeilmenge` [1, 2, 3, 4, 5]| & $\leadsto$ \lstinline|True|\\
\lstinline|[1, 3, 6] `istTeilmenge` [1, 2, 3, 4, 5]| & $\leadsto$ \lstinline|False|
\end{tabular}
\end{center}
\subsection{}
Die Funktion \lstinline|vereinigung| bestimmt die Vereinigungsmenge zweier Mengen und
implementiert damit die Funktion $\cup$. Es ergibt sich z.B.
\begin{center}
\lstinline|[1, 2, 4] `vereinigung` [3, 4, 5]|  $\leadsto$ \lstinline|[1, 2, 3, 4, 5]|
\end{center}
\subsection{}
Die Funktion \lstinline|schnitt| bestimmt die Schnittmenge zweier Mengen und implementiert
damit die Funktion $\cap$. Es ergibt sich z.B.
\begin{center}
\lstinline|[1, 2, 4] `schnitt` [3, 4, 5]| $\leadsto$ \lstinline|[4]|
\end{center}
\newpage
\section{Mengen mit Currying}
Anders als in der letzten Aufgabe wollen wir nun Mengen nicht als Listen sondern 
durch ihre Eigenschaften definieren.
Gegeben seinen die Typdefinition \lstinline|Set| sowie die Funktion \lstinline|myelem|:
\begin{lstlisting}
type Set = (Integer -> Bool)

myElem :: Set -> Integer  -> Bool
myElem    s      e         = s e
\end{lstlisting}
Der Typ \lstinline|Set| ist eine Funktion von Integer nach Bool, die entscheidet ob ein Element in der
Menge liegt oder nicht. Die Funktion \lstinline|myElem| werdet den Typ \lstinline|Set| aus. 

Denken Sie für die folgenden Teilaufgaben an Lambda-Ausdrücke und das Currying!

Damit Sie ihre Menge Ausgeben können, wird Ihnen die folgende Funktion \lstinline|myPrint| gestellt:
\begin{lstlisting}
myPrint :: Set -> Integer -> Integer -> [Integer]
myPrint s min max | min > max    = []
                  | myElem s min = min : (myPrint s (min + 1) max)
                  | otherwise    = myPrint s (min + 1) max
\end{lstlisting}
Ihr wird eine Menge sowie ein Start- und Endwert übergeben und gibt eine Liste von 
Integer-Werten zurück.

\subsection{Einelementige Mengen}
Schreiben Sie die Funktion \lstinline|mySingleSet :: Integer    -> Set|, welche einen 
Integerwert erhält und aus diesem eine Menge erstellt. 

\subsection{Mengen durch Eigenschaften}
Schreiben Sie die Funktion \lstinline|myBoolSet :: (Integer -> Bool) -> Set|, welche eine 
die Eigenschaften einer Menge als Lambda-Ausdrücke erhält und eine Menge zurück gibt.

\subsection{Vereinigung}
Schreiben Sie die Funktion \lstinline|myUnion :: Set -> Set -> Set|, welche zwei Mengen
logisch vereinigt. Denken Sie an die mathematische Definition der Vereinigung.

\subsection{Hinzufügen}
Schreiben Sie die Funktion \lstinline|myAdd :: Set -> Integer -> Set|, welche einen Integer-Wert 
zu einer Menge hinzufügt.

\subsection{Durchschnitt}
Schreiben Sie die Funktion \lstinline|myIntersect :: Set -> Set -> Set|, welche zwei Mengen
logisch schneidet. Denken Sie an die mathematische Definition des Durchschnitts.

\subsection{Differenz}
Schreiben Sie die Funktion \lstinline|myDiff :: Set -> Set -> Set|, welche zwei Mengen
erhält und die zweite von der ersten Menge abzieht. Denken Sie an die mathematische Definition der Differenz.

\subsection{Mehrelementige Mengen (nicht so einfach)}
Schreiben Sie die Funktion \lstinline|myListSet :: [Integer] -> Set|, welche eine Liste von Integer-Werten erhält und 
aus dieser eine Funktion erstellt. 

\subsection{Quantoren}
Da wir nun Mengen besitzen können wir nun auch Beweise auf ihnen führen.
Schreiben Sie dazu die Funktion \lstinline|myForAll :: Set -> (Integer -> Bool) -> Integer -> Integer -> Bool|, welche eine Menge
auf eine Eigenschaft prüft und nur dann wahr ausgibt, wenn alle Elemente diese Eigenschaft besitzen also den $\forall$ Quantor.
Die letzten beiden Integer-werte sind das Intervall. Die können die Funktion in Anlehnung an die Funktion
\lstinline|myprint| erstellen.

Nach dem Sie den $\forall$ Quantor erstellt haben, erstellen Sie nun die Funktion \lstinline|myExists :: Set -> (Integer -> Bool) -> Integer -> Integer -> Bool| 
die den $\exists$ Quantor repräsentiert. Denken Sie an die mathematischen Zusammenhang zwischen den beiden Quantoren.

\subsection{Filter}
Schreiben Sie die Funktion \lstinline|myFilter :: Set -> (Integer -> Bool) -> Set|, welche ein positiv Filter für eine Menge
realisiert. Somit wenn ein Element in der Menge ist und den Filter erfüllt, dann ist dies auch in der zurück gegebenen Menge.

\subsection{Teilmenge}
Schreiben Sie die Funktion \lstinline|mySubset :: Set -> Set -> Bool|, welche bestimmt ob die erste Menge eine Teilmenge der zweiten ist.
Die Intervallgrenzen können Sie beliebig festlegen.

\subsection{Äquivalenz}
Schreiben Sie die Funktion \lstinline|myEquals :: Set -> Set -> Bool|, welche bestimmt ob zwei Mengen identisch sind.
\end{document}
