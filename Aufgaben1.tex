\documentclass[
  10pt,                   % Schriftgröße
  DIV12,
  german,                 % für Umlaute, Silbentrennung etc.
  oneside,                % einseitiges Dokument
  %titlepage,              % es wird eine Titelseite verwendet
  parskip=half,           % Abstand zwischen Absätzen (halbe Zeile)
  headings=normal,        % Größe der Überschriften verkleinern
  captions=tableheading,  % Beschriftung von Tabellen unterhalb ausgeben
  %final                   % Status des Dokuments (final/draft)
]{scrartcl}      


\usepackage[utf8x]{inputenc}
\usepackage[ngerman]{babel}

\usepackage{geometry}
\geometry{verbose,                     % zeigt die eingestellten Parameter beim
                                       % Latexlauf an
      paper=a4paper,                   % Papierformat
      top=25mm,                        % Rand oben
      left=25mm,                       % Rand links
      right=25mm,                      % Rand rechts
      bottom=40mm,                     % Rand unten
      pdftex                           % schreibt das Papierformat in die
                                       % Ausgabe damit Ausgabeprogramm
                                       % Papiergröße erkennt
}

%\usepackage[headsepline, footsepline, ilines]{scrpage2}
%\chead{}

\usepackage[%
  automark,
  headsepline,                %% Separation line below the header
  footsepline,               %% Separation line above the footer
  markuppercase
]{scrpage2}

%\lehead{\bfseries\pagemark}    %% Top left on even pages
%\lohead{\bfseries\headmark}    %% Top left on odd pages
%\rehead{\bfseries\headmark}    %% Top right on even pages
%\rohead{\bfseries\pagemark}    %% Top right on odd pages
\chead{Programmieren für Fortgeschrittene \\ Aufgabenblatt I}                       %% Top center

\automark[subsection]{section}
\pagestyle{scrheadings}

\usepackage{amsfonts}
\usepackage{amssymb}
\usepackage{amsmath}

\usepackage{microtype}
\usepackage{float}

\usepackage{mathptmx}
\usepackage[scaled=.90]{helvet}
\usepackage{courier}

\usepackage{tikz}
\usetikzlibrary{
  arrows,
  shapes.misc,
  shapes.arrows,
  chains,
  matrix,
  positioning,
  scopes,
  decorations.pathmorphing,
  shadows,
  backgrounds
}

\usepackage{listings}
\usepackage{color}

\lstset{
	language=Haskell,
	numbers=none,
	keywordstyle=\color{green!40!black},
	commentstyle=\color{magenta!40!black},
	identifierstyle=\color{blue},
	stringstyle=\color{orange},
	%otherkeywords={$, \{, \}, \[, \]},
	frame=single,
	tabsize=2
}

\begin{document}
\renewcommand*\thesubsection{\roman{subsection})}

\section*{Hinweise}
\subsection{Kommentare}
\begin{itemize}
\item Zeilenkommentar: \lstinline|-- Text| 
\item Blockkommentar: \lstinline|{- Text -}|
\end{itemize}

\subsection{Konstanten}
\begin{minipage}[hbt]{6cm}
\centering
\begin{lstlisting}
pi :: Double
pi = 3.14
\end{lstlisting}
\end{minipage}
\hfill
\begin{minipage}[hbt]{8cm}
\centering
\scalebox{0.7}{\begin{tikzpicture}[
      nonterminal/.style={
         rectangle,
         minimum size=5.0mm,
         very thick,
         draw=blue!50!black!50,
         top color=white,
         bottom color=blue!50!black!20,
         font=\itshape\scriptsize,
         %text height=1.5ex,
         %text depth=.25ex
      },
      terminal/.style={
         rounded rectangle,
         minimum size=5.0mm,
         very thick,draw=black!50,
         top color=white,bottom color=black!20,
         font=\ttfamily\scriptsize,
         %text height=1.5ex,
         %text depth=.25ex
      },
      skip loop/.style={
         to path={-- ++(0,#1) -| (\tikztotarget)}
      },
      point/.style={coordinate},>=stealth',thick,draw=black!50,
      tip/.style={->,shorten >=0.007pt},every join/.style={rounded corners},
      hv path/.style={to path={-| (\tikztotarget)}},
      vh path/.style={to path={|- (\tikztotarget)}},
      %text height=1.5ex,text depth=.25ex
    ]
    \matrix[column sep=5.0mm, row sep=3.0mm] {
 &  & \node (n13) [nonterminal] {Bezeichner}; & \node (n14) [terminal] {::}; & \node (n15) [nonterminal] {Typ Erg}; & \node (n16) [terminal] {LF}; &  & \\
\node (n21) [circle, draw, inner sep=0pt, minimum size=0.75ex] {}; & \node (n22) [point] {}; &  &  &  &  & \node (n27) [point] {}; & \node (n28) [point] {};\\
 &  & \node (n33) [point] {}; &  &  &  &  & \\
 &  &  &  &  &  &  & \\
 &  &  &  & \node (n55) [nonterminal] {Wert}; &  &  & \\
\node (n61) [point] {}; & \node (n62) [nonterminal] {Bezeichner}; & \node (n63) [terminal] {=}; & \node (n64) [point] {}; &  & \node (n66) [point] {}; & \node (n67) [terminal] {LF}; & \node (n68) [circle, draw, inner sep=0pt, minimum size=0.75ex] {};\\
 &  &  &  & \node (n75) [nonterminal] {Ausdruck}; &  &  & \\
    };
  { [start chain]
       \chainin (n21);
       \chainin (n22)    [join];
  }
  { [start chain]
       \chainin (n22);
       \chainin (n13)    [join=by {vh path,tip}];
  }
  { [start chain]
       \chainin (n13);
       \chainin (n14)    [join=by tip];
  }
  { [start chain]
       \chainin (n14);
       \chainin (n15)    [join=by tip];
  }
  { [start chain]
       \chainin (n15);
       \chainin (n16)    [join=by tip];
  }
  { [start chain]
       \chainin (n16);
       \chainin (n27)    [join=by {hv path}];
  }
  { [start chain]
       \chainin (n22);
       \chainin (n33)    [join=by {vh path}];
  }
  { [start chain]
       \chainin (n33);
       \chainin (n27)    [join=by {hv path}];
  }
  { [start chain]
       \chainin (n27);
       \chainin (n28)    [join=by tip];
  }
  { [start chain]
       \chainin (n61);
       \chainin (n62)    [join=by tip];
  }
  { [start chain]
       \chainin (n62);
       \chainin (n63)    [join=by tip];
  }
  { [start chain]
       \chainin (n63);
       \chainin (n64)    [join];
  }
  { [start chain]
       \chainin (n64);
       \chainin (n55)    [join=by {vh path,tip}];
  }
  { [start chain]
       \chainin (n55);
       \chainin (n66)    [join=by {hv path}];
  }
  { [start chain]
       \chainin (n64);
       \chainin (n75)    [join=by {vh path,tip}];
  }
  { [start chain]
       \chainin (n75);
       \chainin (n66)    [join=by {hv path}];
  }
  { [start chain]
       \chainin (n66);
       \chainin (n67)    [join=by tip];
  }
  { [start chain]
       \chainin (n67);
       \chainin (n68)    [join=by tip];
  }
\end{tikzpicture}
}
\end{minipage}

\subsection{Funktionen}
\begin{minipage}[hbt]{6cm}
\centering
\begin{lstlisting}
plus :: Int -> Int -> Int
plus a b = a + b
\end{lstlisting}
\end{minipage}
\hfill
\begin{minipage}[hbt]{8cm}
\centering
\scalebox{0.6}{\begin{tikzpicture}[
      nonterminal/.style={
         rectangle,
         minimum size=5.0mm,
         very thick,
         draw=blue!50!black!50,
         top color=white,
         bottom color=blue!50!black!20,
         font=\itshape\scriptsize,
         %text height=1.5ex,
         %text depth=.25ex
      },
      terminal/.style={
         rounded rectangle,
         minimum size=5.0mm,
         very thick,draw=black!50,
         top color=white,bottom color=black!20,
         font=\ttfamily\scriptsize,
         %text height=1.5ex,
         %text depth=.25ex
      },
      skip loop/.style={
         to path={-- ++(0,#1) -| (\tikztotarget)}
      },
      point/.style={coordinate},>=stealth',thick,draw=black!50,
      tip/.style={->,shorten >=0.007pt},every join/.style={rounded corners},
      hv path/.style={to path={-| (\tikztotarget)}},
      vh path/.style={to path={|- (\tikztotarget)}},
      %text height=1.5ex,text depth=.25ex
    ]
    \matrix[column sep=5.0mm, row sep=3.0mm] {
 &  &  &  &  & \node (n16) [point] {}; &  &  &  &  &  &  & \\
 &  & \node (n23) [nonterminal] {Bezeichner}; & \node (n24) [terminal] {::}; & \node (n25) [point] {}; & \node (n26) [nonterminal] {Typ Para}; & \node (n27) [terminal] {->}; & \node (n28) [point] {}; & \node (n29) [nonterminal] {Typ Erg}; & \node (n210) [terminal] {LF}; &  &  & \\
\node (n31) [circle, draw, inner sep=0pt, minimum size=0.75ex] {}; & \node (n32) [point] {}; &  &  &  &  &  &  &  &  & \node (n311) [point] {}; & \node (n312) [point] {}; & \\
 &  & \node (n43) [point] {}; &  &  &  &  &  &  &  &  &  & \\
 &  &  &  &  &  &  &  &  &  &  &  & \\
 &  &  &  &  &  & \node (n67) [point] {}; &  &  &  &  &  & \\
 &  &  &  & \node (n75) [point] {}; &  &  &  & \node (n79) [nonterminal] {Wert}; &  &  &  & \\
\node (n81) [point] {}; & \node (n82) [point] {}; & \node (n83) [nonterminal] {Bezeichner}; & \node (n84) [point] {}; & \node (n85) [nonterminal] {Bezeichner Para}; & \node (n86) [point] {}; & \node (n87) [terminal] {=}; & \node (n88) [point] {}; &  & \node (n810) [point] {}; & \node (n811) [terminal] {LF}; & \node (n812) [point] {}; & \node (n813) [circle, draw, inner sep=0pt, minimum size=0.75ex] {};\\
 &  &  &  &  &  &  &  & \node (n99) [nonterminal] {Ausdruck}; &  &  &  & \\
    };
  { [start chain]
       \chainin (n31);
       \chainin (n32)    [join];
  }
  { [start chain]
       \chainin (n32);
       \chainin (n23)    [join=by {vh path,tip}];
  }
  { [start chain]
       \chainin (n23);
       \chainin (n24)    [join=by tip];
  }
  { [start chain]
       \chainin (n24);
       \chainin (n25)    [join];
  }
  { [start chain]
       \chainin (n25);
       \chainin (n26)    [join=by tip];
  }
  { [start chain]
       \chainin (n26);
       \chainin (n27)    [join=by tip];
  }
  { [start chain]
       \chainin (n27);
       \chainin (n28)    [join];
  }
  { [start chain]
       \chainin (n28);
       \chainin (n29)    [join=by tip];
  }
  { [start chain]
       \chainin (n29);
       \chainin (n210)    [join=by tip];
  }
  { [start chain]
       \chainin (n210);
       \chainin (n311)    [join=by {hv path}];
  }
  { [start chain]
       \chainin (n32);
       \chainin (n43)    [join=by {vh path}];
  }
  { [start chain]
       \chainin (n43);
       \chainin (n311)    [join=by {hv path}];
  }
  { [start chain]
       \chainin (n311);
       \chainin (n312)    [join=by tip];
  }
  { [start chain]
       \chainin (n81);
       \chainin (n82)    [join];
  }
  { [start chain]
       \chainin (n82);
       \chainin (n83)    [join=by tip];
  }
  { [start chain]
       \chainin (n83);
       \chainin (n84)    [join];
  }
  { [start chain]
       \chainin (n84);
       \chainin (n85)    [join=by tip];
  }
  { [start chain]
       \chainin (n85);
       \chainin (n86)    [join];
  }
  { [start chain]
       \chainin (n86);
       \chainin (n87)    [join=by tip];
  }
  { [start chain]
       \chainin (n87);
       \chainin (n88)    [join];
  }
  { [start chain]
       \chainin (n88);
       \chainin (n79)    [join=by {vh path,tip}];
  }
  { [start chain]
       \chainin (n79);
       \chainin (n810)    [join=by {hv path}];
  }
  { [start chain]
       \chainin (n88);
       \chainin (n99)    [join=by {vh path,tip}];
  }
  { [start chain]
       \chainin (n99);
       \chainin (n810)    [join=by {hv path}];
  }
  { [start chain]
       \chainin (n810);
       \chainin (n811)    [join=by tip];
  }
  { [start chain]
       \chainin (n811);
       \chainin (n812)    [join];
  }
  { [start chain]
       \chainin (n812);
       \chainin (n813)    [join=by tip];
  }
  { [start chain]
       \chainin (n28);
       \chainin (n16)    [join=by vh path];
       \chainin (n25)    [join=by {hv path,tip}];
  }
  { [start chain]
       \chainin (n86);
       \chainin (n75)    [join=by vh path];
       \chainin (n84)    [join=by {hv path,tip}];
  }
  { [start chain]
       \chainin (n812);
       \chainin (n67)    [join=by vh path];
       \chainin (n82)    [join=by {hv path,tip}];
  }
\end{tikzpicture}
}
\end{minipage}

\subsection{If-Then-Else}
\begin{minipage}[hbt]{6cm}
\centering
\begin{lstlisting}
ggT a b = if b == 0 
        then a 
        else ggT b (mod a b)
\end{lstlisting}
\end{minipage}
\hfill
\begin{minipage}[hbt]{8cm}
\centering
\scalebox{0.7}{\input{images/ifthenelse.tex}}
\end{minipage}

\subsection{Case Of}
\begin{minipage}[hbt]{6cm}
\centering
\begin{lstlisting}
ggT a b = case b of 
        0 -> a
        _ -> ggT b (mod a b)
\end{lstlisting}
\end{minipage}
\hfill
\begin{minipage}[hbt]{8cm}
\centering
\scalebox{0.8}{\begin{tikzpicture}[
      nonterminal/.style={
         rectangle,
         minimum size=5.0mm,
         very thick,
         draw=blue!50!black!50,
         top color=white,
         bottom color=blue!50!black!20,
         font=\itshape\scriptsize,
         %text height=1.5ex,
         %text depth=.25ex
      },
      terminal/.style={
         rounded rectangle,
         minimum size=5.0mm,
         very thick,draw=black!50,
         top color=white,bottom color=black!20,
         font=\ttfamily\scriptsize,
         %text height=1.5ex,
         %text depth=.25ex
      },
      skip loop/.style={
         to path={-- ++(0,#1) -| (\tikztotarget)}
      },
      point/.style={coordinate},>=stealth',thick,draw=black!50,
      tip/.style={->,shorten >=0.007pt},every join/.style={rounded corners},
      hv path/.style={to path={-| (\tikztotarget)}},
      vh path/.style={to path={|- (\tikztotarget)}},
      %text height=1.5ex,text depth=.25ex
    ]
    \matrix[column sep=5.0mm, row sep=3.0mm] {
\node (n11) [circle, draw, inner sep=0pt, minimum size=0.75ex] {}; & \node (n12) [terminal] {=}; & \node (n13) [terminal] {case}; & \node (n14) [nonterminal] {Variable}; & \node (n15) [terminal] {of}; & \node (n16) [terminal] {LF}; & \node (n17) [point] {}; & \\
 &  &  &  &  &  &  & \\
 &  &  & \node (n34) [point] {}; &  &  &  & \\
\node (n41) [point] {}; & \node (n42) [point] {}; & \node (n43) [nonterminal] {erwarteter Wert}; & \node (n44) [terminal] {->}; & \node (n45) [nonterminal] {Ausdruck}; & \node (n46) [terminal] {LF}; & \node (n47) [point] {}; & \node (n48) [point] {};\\
 &  &  &  &  &  &  & \\
 &  &  & \node (n64) [point] {}; &  &  &  & \\
\node (n71) [point] {}; & \node (n72) [point] {}; & \node (n73) [terminal] {\_}; & \node (n74) [terminal] {->}; & \node (n75) [nonterminal] {default Ausdruck}; & \node (n76) [point] {}; & \node (n77) [circle, draw, inner sep=0pt, minimum size=0.75ex] {}; & \\
    };
  { [start chain]
       \chainin (n11);
       \chainin (n12)    [join=by tip];
  }
  { [start chain]
       \chainin (n12);
       \chainin (n13)    [join=by tip];
  }
  { [start chain]
       \chainin (n13);
       \chainin (n14)    [join=by tip];
  }
  { [start chain]
       \chainin (n14);
       \chainin (n15)    [join=by tip];
  }
  { [start chain]
       \chainin (n15);
       \chainin (n16)    [join=by tip];
  }
  { [start chain]
       \chainin (n16);
       \chainin (n17)    [join=by tip];
  }
  { [start chain]
       \chainin (n41);
       \chainin (n42)    [join];
  }
  { [start chain]
       \chainin (n42);
       \chainin (n43)    [join=by tip];
  }
  { [start chain]
       \chainin (n43);
       \chainin (n44)    [join=by tip];
  }
  { [start chain]
       \chainin (n44);
       \chainin (n45)    [join=by tip];
  }
  { [start chain]
       \chainin (n45);
       \chainin (n46)    [join=by tip];
  }
  { [start chain]
       \chainin (n46);
       \chainin (n47)    [join];
  }
  { [start chain]
       \chainin (n47);
       \chainin (n48)    [join=by tip];
  }
  { [start chain]
       \chainin (n71);
       \chainin (n72)    [join];
  }
  { [start chain]
       \chainin (n72);
       \chainin (n73)    [join=by tip];
  }
  { [start chain]
       \chainin (n73);
       \chainin (n74)    [join=by tip];
  }
  { [start chain]
       \chainin (n74);
       \chainin (n75)    [join=by tip];
  }
  { [start chain]
       \chainin (n75);
       \chainin (n76)    [join];
  }
  { [start chain]
       \chainin (n76);
       \chainin (n77)    [join=by tip];
  }
  { [start chain]
       \chainin (n47);
       \chainin (n34)    [join=by vh path];
       \chainin (n42)    [join=by {hv path,tip}];
  }
  { [start chain]
       \chainin (n72);
       \chainin (n64)    [join=by vh path];
       \chainin (n76)    [join=by {hv path,tip}];
  }
\end{tikzpicture}
}
\end{minipage}

\subsection{Pattern Matching}
\begin{minipage}[hbt]{6cm}
\centering
\begin{lstlisting}
ggT a 0 = a
ggT a b = ggT b (mod a b)
\end{lstlisting}
\end{minipage}
\hfill
\begin{minipage}[hbt]{8cm}
\centering
\scalebox{0.8}{\begin{tikzpicture}[
      nonterminal/.style={
         rectangle,
         minimum size=5.0mm,
         very thick,
         draw=blue!50!black!50,
         top color=white,
         bottom color=blue!50!black!20,
         font=\itshape\scriptsize,
         %text height=1.5ex,
         %text depth=.25ex
      },
      terminal/.style={
         rounded rectangle,
         minimum size=5.0mm,
         very thick,draw=black!50,
         top color=white,bottom color=black!20,
         font=\ttfamily\scriptsize,
         %text height=1.5ex,
         %text depth=.25ex
      },
      skip loop/.style={
         to path={-- ++(0,#1) -| (\tikztotarget)}
      },
      point/.style={coordinate},>=stealth',thick,draw=black!50,
      tip/.style={->,shorten >=0.007pt},every join/.style={rounded corners},
      hv path/.style={to path={-| (\tikztotarget)}},
      vh path/.style={to path={|- (\tikztotarget)}},
      %text height=1.5ex,text depth=.25ex
    ]
    \matrix[column sep=5.0mm, row sep=3.0mm] {
 &  &  & \node (n14) [point] {}; &  &  &  &  & \\
\node (n21) [circle, draw, inner sep=0pt, minimum size=0.75ex] {}; & \node (n22) [nonterminal] {Funktion}; & \node (n23) [point] {}; & \node (n24) [nonterminal] {erwarteter Wert}; & \node (n25) [terminal] { }; & \node (n26) [point] {}; & \node (n27) [terminal] {=}; & \node (n28) [nonterminal] {Ausdruck}; & \node (n29) [circle, draw, inner sep=0pt, minimum size=0.75ex] {};\\
    };
  { [start chain]
       \chainin (n21);
       \chainin (n22)    [join=by tip];
  }
  { [start chain]
       \chainin (n22);
       \chainin (n23)    [join];
  }
  { [start chain]
       \chainin (n23);
       \chainin (n24)    [join=by tip];
  }
  { [start chain]
       \chainin (n24);
       \chainin (n25)    [join=by tip];
  }
  { [start chain]
       \chainin (n25);
       \chainin (n26)    [join];
  }
  { [start chain]
       \chainin (n26);
       \chainin (n27)    [join=by tip];
  }
  { [start chain]
       \chainin (n27);
       \chainin (n28)    [join=by tip];
  }
  { [start chain]
       \chainin (n28);
       \chainin (n29)    [join=by tip];
  }
  { [start chain]
       \chainin (n26);
       \chainin (n14)    [join=by vh path];
       \chainin (n23)    [join=by {hv path,tip}];
  }
\end{tikzpicture}
}
\end{minipage}

\subsection{Guards}
\begin{minipage}[hbt]{6cm}
\centering
\begin{lstlisting}
ggT a b | b == 0    = a
        | otherwise = 
            ggT b (mod a b) 
\end{lstlisting}
\end{minipage}
\hfill
\begin{minipage}[hbt]{8cm}
\centering
\scalebox{0.8}{\begin{tikzpicture}[
      nonterminal/.style={
         rectangle,
         minimum size=5.0mm,
         very thick,
         draw=blue!50!black!50,
         top color=white,
         bottom color=blue!50!black!20,
         font=\itshape\scriptsize,
         %text height=1.5ex,
         %text depth=.25ex
      },
      terminal/.style={
         rounded rectangle,
         minimum size=5.0mm,
         very thick,draw=black!50,
         top color=white,bottom color=black!20,
         font=\ttfamily\scriptsize,
         %text height=1.5ex,
         %text depth=.25ex
      },
      skip loop/.style={
         to path={-- ++(0,#1) -| (\tikztotarget)}
      },
      point/.style={coordinate},>=stealth',thick,draw=black!50,
      tip/.style={->,shorten >=0.007pt},every join/.style={rounded corners},
      hv path/.style={to path={-| (\tikztotarget)}},
      vh path/.style={to path={|- (\tikztotarget)}},
      %text height=1.5ex,text depth=.25ex
    ]
    \matrix[column sep=5.0mm, row sep=3.0mm] {
 &  &  &  & \node (n15) [point] {}; &  &  &  & \\
\node (n21) [circle, draw, inner sep=0pt, minimum size=0.75ex] {}; & \node (n22) [point] {}; & \node (n23) [terminal] {|}; & \node (n24) [nonterminal] {boolescher Term}; & \node (n25) [terminal] {=}; & \node (n26) [nonterminal] {Ausdruck}; & \node (n27) [terminal] {LF}; & \node (n28) [point] {}; & \node (n29) [point] {};\\
 &  &  &  &  &  &  &  & \\
\node (n41) [point] {}; & \node (n42) [terminal] {otherwise}; & \node (n43) [terminal] {=}; & \node (n44) [nonterminal] {default Ausdruck}; & \node (n45) [circle, draw, inner sep=0pt, minimum size=0.75ex] {}; &  &  &  & \\
    };
  { [start chain]
       \chainin (n21);
       \chainin (n22)    [join];
  }
  { [start chain]
       \chainin (n22);
       \chainin (n23)    [join=by tip];
  }
  { [start chain]
       \chainin (n23);
       \chainin (n24)    [join=by tip];
  }
  { [start chain]
       \chainin (n24);
       \chainin (n25)    [join=by tip];
  }
  { [start chain]
       \chainin (n25);
       \chainin (n26)    [join=by tip];
  }
  { [start chain]
       \chainin (n26);
       \chainin (n27)    [join=by tip];
  }
  { [start chain]
       \chainin (n27);
       \chainin (n28)    [join];
  }
  { [start chain]
       \chainin (n28);
       \chainin (n29)    [join=by tip];
  }
  { [start chain]
       \chainin (n41);
       \chainin (n42)    [join=by tip];
  }
  { [start chain]
       \chainin (n42);
       \chainin (n43)    [join=by tip];
  }
  { [start chain]
       \chainin (n43);
       \chainin (n44)    [join=by tip];
  }
  { [start chain]
       \chainin (n44);
       \chainin (n45)    [join=by tip];
  }
  { [start chain]
       \chainin (n28);
       \chainin (n15)    [join=by vh path];
       \chainin (n22)    [join=by {hv path,tip}];
  }
\end{tikzpicture}
}
\end{minipage}

\subsection{Module}
\begin{lstlisting}
module Wurf(weite, square) where
weite :: Double -> Double -> Double
weite v0 phi = ((square v0) / 9.81) * sin (2 * phi)
square :: Double -> Double
square x = x * x

module Foo where
import Wurf hiding (weite)
bar ... = ... (square a) ...
\end{lstlisting}
\subsection{Eigene Datentypen}
\begin{lstlisting}
data Point = Point{x :: Double, y :: Double}

data Shape = Circle{center :: Point, 
    radius :: Double}
  | Rectangle{point :: Point, width :: Double, 
    height :: Double}
  | Triangle{point1 :: Point, point2 :: Point, 
    point3 :: Point}
\end{lstlisting}
\subsection{Typparameter}
Keine Typparameter in Datentyp Definitionen!
\begin{lstlisting}
data (Eq a, Ord a) => Pair a b = 
  PairConst {first :: a, second :: b} deriving(Show)
\end{lstlisting}

besser:
\begin{lstlisting}
data Pair a b = PairConst a b deriving(Show)

instance Eq a => Eq (Pair a b) where
  (PairConst i1 _) == (PairConst i2 _) = i1 == i2

instance Ord a => Ord (Pair a b) where
   (PairConst i1 _) <= (PairConst i2 _) = i1 <= i2

bubbleSort :: (Ord a) => [(Pair a b)] -> [(Pair a b)]
bubbleSort [] = []
bubbleSort (x:xs) = step $ foldl go (x,[]) xs where
  go (y,acc) x = (min x y, max x y : acc)
  step (x,acc) = x : bubbleSort acc
\end{lstlisting}

\$ ist nichts anderes als: 
\begin{lstlisting}
(7 + 6) / (5 + 3) = (7 + 6) / $ 5 + 3
\end{lstlisting}
Die "`Klammerung"' geht bis zum Zeilenende.
\subsection{Listen}
%\begin{minipage}[hbt]{7cm}
%\centering
\begin{lstlisting}
primes = sieve [2..]
  where
    sieves (p:xs) = p:sieves [x|x<- xs, mod x p > 0]
\end{lstlisting}
%\end{minipage}
%\hfill
%\begin{minipage}[hbt]{7cm}
%\centering
\begin{center}
\scalebox{1}{\begin{tikzpicture}[
      nonterminal/.style={
         rectangle,
         minimum size=5.0mm,
         very thick,
         draw=blue!50!black!50,
         top color=white,
         bottom color=blue!50!black!20,
         font=\itshape\scriptsize,
         %text height=1.5ex,
         %text depth=.25ex
      },
      terminal/.style={
         rounded rectangle,
         minimum size=5.0mm,
         very thick,draw=black!50,
         top color=white,bottom color=black!20,
         font=\ttfamily\scriptsize,
         %text height=1.5ex,
         %text depth=.25ex
      },
      skip loop/.style={
         to path={-- ++(0,#1) -| (\tikztotarget)}
      },
      point/.style={coordinate},>=stealth',thick,draw=black!50,
      tip/.style={->,shorten >=0.007pt},every join/.style={rounded corners},
      hv path/.style={to path={-| (\tikztotarget)}},
      vh path/.style={to path={|- (\tikztotarget)}},
      %text height=1.5ex,text depth=.25ex
    ]
    \matrix[column sep=5.0mm, row sep=3.0mm] {
 &  &  &  &  &  &  &  & \node (n19) [point] {}; &  &  &  &  &  &  &  & \\
 &  &  &  &  &  &  & \node (n28) [point] {}; &  &  &  &  & \node (n213) [point] {}; &  &  &  & \\
\node (n31) [circle, draw, inner sep=0pt, minimum size=0.75ex] {}; & \node (n32) [terminal] {[}; & \node (n33) [nonterminal] {Ausdruck}; & \node (n34) [point] {}; & \node (n35) [terminal] {|}; & \node (n36) [point] {}; & \node (n37) [nonterminal] {Variable}; & \node (n38) [terminal] {<-}; & \node (n39) [nonterminal] {Quelle}; & \node (n310) [terminal] {,}; & \node (n311) [point] {}; & \node (n312) [point] {}; & \node (n313) [nonterminal] {Filter}; & \node (n314) [point] {}; & \node (n315) [point] {}; & \node (n316) [terminal] {]}; & \node (n317) [circle, draw, inner sep=0pt, minimum size=0.75ex] {};\\
    };
  { [start chain]
       \chainin (n31);
       \chainin (n32)    [join=by tip];
  }
  { [start chain]
       \chainin (n32);
       \chainin (n33)    [join=by tip];
  }
  { [start chain]
       \chainin (n33);
       \chainin (n34)    [join];
  }
  { [start chain]
       \chainin (n34);
       \chainin (n35)    [join=by tip];
  }
  { [start chain]
       \chainin (n35);
       \chainin (n36)    [join];
  }
  { [start chain]
       \chainin (n36);
       \chainin (n37)    [join=by tip];
  }
  { [start chain]
       \chainin (n37);
       \chainin (n38)    [join=by tip];
  }
  { [start chain]
       \chainin (n38);
       \chainin (n39)    [join=by tip];
  }
  { [start chain]
       \chainin (n39);
       \chainin (n310)    [join=by tip];
  }
  { [start chain]
       \chainin (n310);
       \chainin (n311)    [join];
  }
  { [start chain]
       \chainin (n311);
       \chainin (n312)    [join];
  }
  { [start chain]
       \chainin (n312);
       \chainin (n313)    [join=by tip];
  }
  { [start chain]
       \chainin (n313);
       \chainin (n314)    [join];
  }
  { [start chain]
       \chainin (n314);
       \chainin (n315)    [join];
  }
  { [start chain]
       \chainin (n315);
       \chainin (n316)    [join=by tip];
  }
  { [start chain]
       \chainin (n316);
       \chainin (n317)    [join=by tip];
  }
  { [start chain]
       \chainin (n311);
       \chainin (n28)    [join=by vh path];
       \chainin (n36)    [join=by {hv path,tip}];
  }
  { [start chain]
       \chainin (n312);
       \chainin (n213)    [join=by vh path];
       \chainin (n314)    [join=by {hv path,tip}];
  }
  { [start chain]
       \chainin (n34);
       \chainin (n19)    [join=by vh path];
       \chainin (n315)    [join=by {hv path,tip}];
  }
\end{tikzpicture}
}
\end{center}
%\end{minipage}

\subsection{Operationen}
\begin{table}[H]
	\centering
    \begin{tabular}{lll} %\hline
        Bezeichner & Typ	& Bedeutung \\ \hline
        \lstinline|(+)|        & \lstinline|a -> a -> a| & Addition          \\ 
        \lstinline|(-)|        & \lstinline|a -> a -> a| & Subtraktion       \\ 
        \lstinline|(*)|        & \lstinline|a -> a -> a| & Multiplikation    \\ 
        \lstinline|negate|     & \lstinline|a -> a|      & Negation        \\ 
        \lstinline|abs|        & \lstinline|a -> a|      & Absolutbetrag     \\ 
        \lstinline|signum|     & \lstinline|a -> a|      & Vorzeichenbildung %\\ \hline
    \end{tabular}	
\caption{Für die Typen \lstinline|Int|, \lstinline|Integer|, \lstinline|Float| und \lstinline|Double|}
\end{table}
\begin{table}[H]
	\centering
    \begin{tabular}{lll} %\hline
        Bezeichner & Typ         & Bedeutung         \\ \hline
		\lstinline|succ| 	& \lstinline|a -> a| & Nachfolgerbildung \\
		\lstinline|pred| 	& \lstinline|a -> a| & Vorgängerbildung\\
		\lstinline|div| 	& \lstinline|a -> a -> a| & ganzzahlige Division, Ergebnis wird abgerundet\\
		\lstinline|mod| 	& \lstinline|a -> a -> a| & zur ganzzahligen Division \lstinline|div| gehörender Rest\\
		\lstinline|quot| 	& \lstinline|a -> a -> a| & ganzzahlige Division, Ergebnis wird Richtung 0 gerundet\\
		\lstinline|rem| 	& \lstinline|a -> a -> a| & zur ganzzahligen Division \lstinline|quot| gehörender Rest %\\ \hline
    \end{tabular}	
\caption{Für die Typen \lstinline|Int| und \lstinline|Integer|}
\end{table}
\begin{table}[H]
	\centering
    \begin{tabular}{lll} %\hline
        Bezeichner & Typ         & Bedeutung         \\ \hline
		\lstinline|(/)|	 	& \lstinline|a -> a -> a| & Division\\
		\lstinline|recip| 	& \lstinline|a -> a| & Kehrwertbildung\\
		\lstinline|(**)| 	& \lstinline|a -> a -> a| & Potenzieren\\
		\lstinline|sqrt| 	& \lstinline|a -> a| & Ziehen der Quadratwurzel %\\ \hline
    \end{tabular}
\caption{Für die Typen \lstinline|Float| und \lstinline|Double|}
\end{table}
\begin{table}[H]
	\centering
    \begin{tabular}{lll} %\hline
        Bezeichner & Typ         & Bedeutung         \\ \hline
		\lstinline!(||)!	& \lstinline|a -> a -> a| & Oder\\
		\lstinline|(&&)| 	& \lstinline|a -> a -> a| & Und\\
		\lstinline|not| 	& \lstinline|a -> a| & Verneinung %\\ \hline
    \end{tabular}
\caption{Für den Typ \lstinline|Bool|}
\end{table}
\newpage
\renewcommand*\thesubsection{\alph{subsection})}
\section{Einfache Haskell-Funktion}
Im Folgenden soll eine Funktion programmiert werden, die die
Wegstrecke $s(t)$ berechnet, welche im freien Fall unter Einfluss der Erdbeschleunigung $g = 9.81$ nach einer Zeit $t$ zurück gelegt wurde. Verwenden Sie die Formel 
\begin{displaymath}
s(t) = \frac{1}{2} \cdot gt^2.
\end{displaymath}
Die Wegstrecke soll in Metern und die Zeit in Sekunden angegeben werden.
\subsection{}
Mit welche Datentypen sollten die physikalischen Größen Weg, Zeit und Beschleunigung repräsentiert werden? Geben Sie den daraus resultierenden Typ einer Funktion \lstinline|fallstrecke| an, die $s(t)$ aus $t$ berechnet.
\subsection{}
Schreiben Sie die Typsignatur \lstinline|fallstrecke :: Typ| sowie die Definition von \lstinline|fallstrecke| in eine Datei. Laden Sie diese in GHCi und testen Sie Ihre Funktion. Zum Beispiel sollte sich \lstinline|fallstrecke 1 = 4.905| ergeben.
\subsection{}
Verändern Sie Ihr Programm zunächst so, dass $g$ als globale Variable (Konstante) deklariert ist. Ersetzen Sie dann die globale Deklaration durch lokale Deklarationen, zunächst mit \lstinline|where| und anschließend mit \lstinline|let|.
\subsection{}
Die Eingabe eines negativen Wertes soll zu einem undefinierten Funktionswert
und zu einer Fehlermeldung führen. Dies erreichen Sie mit Hilfe der Funktion
\lstinline|error|, z.B. durch \lstinline|error "negative Zeit"|. Realisieren Sie dieses Verhalten einmal mit \textbf{If-Then-Else Ausdrücken} und einmal mit \textbf{Guards}.
\section{Einfache Haskell-Funktionen}
Schreiben Sie Haskell-Funktionen zur Berechnung von Volumen, Oberflächeninhalt und Radius von Kugeln. Beachten Sie, dass einige dieser Funktionen als Komposition der übrigen aufgefasst werden können. Verwenden Sie die vordefinierte Konstante \lstinline|pi::Double|.
Hilfreiche Funktionen über reellen Zahlen finden Sie in den Hinweisen dieses Aufgabenblattes.
\subsection{}
Programmieren Sie die folgenden Funktionen:\\
\begin{itemize}
\item \lstinline|radiusNachOberflaeche :: Double -> Double|
\item \lstinline|oberflaecheNachRadius :: Double -> Double|
\item \lstinline|radiusNachVolumen :: Double -> Double|
\item \lstinline|volumenNachRadius :: Double -> Double|
\item \lstinline|oberflaecheNachVolumen :: Double -> Double|
\item \lstinline|volumenNachOberflaeche :: Double -> Double|
\end{itemize}
\subsection{}
Da es keine Kugeln mit negativem Volumen, Oberflächeninhalt oder Radius
gibt, sollen die Funktionen für negative Eingaben undefiniert bleiben und mit
einer Fehlermeldung abbrechen. Erweitern Sie dazu die Funktionen um sinnvolle
Fallunterscheidungen und Fehlermeldungen.

\section{Bbedingte Ausdrücke}
Schreiben Sie eine Funktion vom Typ \lstinline|Integer -> Bool|, die zu einer gegebenen Jahreszahl prüft, ob sie im Gregorianischen Kalender ein Schaltjahr ist oder nicht.

Pseudocode:
\begin{lstlisting}
if year is >= 1582 then 
  if year is divisible by 400 then
    is_leap_year
  else if year is divisible by 100 then
    not_leap_year
  else if year is divisible by 4 then
    is_leap_year
  else
    not_leap_year
else
  not_leap_year
\end{lstlisting}
\subsection{}
Verwenden Sie hierfür nur If-Then-Else-Ausdrücke.
\subsection{}
Verwenden Sie hierfür nur Guards
\subsection{}
Verwenden Sie hierfür weder If-Then-Else-Ausdrücke noch Guards, sondern die
booleschen Operatoren.
\section{Lokale Deklarationen}
\subsection{}
Zur Bestimmung der Nullstellen von quadratischen Gleichungen
der Form $ax^2 + bx + c$ gibt es neben der bekannten „p-q-Formel“ 
(mit $p = \frac{b}{a}$ und $q = \frac{c}{a}$) auch die so genannte „a-b-c-Formel“
\begin{displaymath}
x_{1, 2} = \frac{-b \pm \sqrt{b^2 - 4\;ac}}{2a}
\end{displaymath}
Setzen Sie diese Berechnung in Haskell zunächst mit \textbf{let-in} und dann mit \textbf{where} um.
Dabei sollen $x_1$ und $x_2$ als 2-Tupel zurückgegeben werden.
\subsection{}
Seit der Veröffentlichung der Gaußschen Wochentagsformel haben verschiedene Mathematiker und Astronomen eine ganze Reihe vergleichbarer Formeln aufgestellt. Diese benötigen meist vorausberechnete Tabellen und verwenden komplizierte Sonderregeln, z.B. für Schaltjahre. Eine im Vergleich dazu sehr elegante Formel wurde 1972 von J. D. Robertson vorgestellt. Diese wird im Folgenden beschrieben. Gegeben sei ein beliebiges Datum $(T, M, J)$ im Gregorianischen Kalender, wobei $T$ den Tag des Monats beginnend mit 1, $M$ den Monat des Jahres beginnend mit 1 und $J$ das Jahr bezeichnet. Dann lässt sich der Wochentag $W$ mit 0 $\equiv$ Sonntag, 1 $\equiv$ Montag, . . . , 6 $\equiv$ Samstag folgendermaßen berechnen:
\begin{displaymath}
W = \left(D  + T + 77 + E + \left[\frac{B}{400}\right] - 2 \left[\frac{B}{100}\right]\right) \;mod\;7
\end{displaymath}
Dabei gelte folgendes:
\begin{align*}
A &= M + 10\\
B &= \left[\frac{M - 14}{12}\right] + J\\
C &= A - 12 \left[\frac{A}{13}\right]\\
D &= \left[\frac{13 \cdot C - 1}{5}\right]\\
E &= \left[\frac{5(B \; mod \; 100)}{4}\right]\\
[x] &= \begin{cases}\lfloor x \rfloor& x \geq 0 \\ \lceil x \rceil & x < 0 \end{cases} \\
\lfloor x \rfloor &= max\{ z \in \mathbb{Z} | z \leq x\}\\
\lceil x \rceil &= min\{ z \in \mathbb{Z} | z \geq x\}
\end{align*}
Schreiben Sie eine Haskell-Funktion \lstinline|robertson :: Integer -> Integer -> Integer -> Integer|, die gemäß der obigen Formel den Wochentag berechnet. Sie können sich mit der Hilfsfunktion $wochentag$ den Wochentag als Text anzeigenlassen. Sie erhalten z.B. folgende Ausgaben:
\begin{itemize}
\item \lstinline|wochentag 19 6 1623| $\to$ \lstinline|"Montag"|
\item \lstinline|wochentag 19 1 2038| $\to$ \lstinline|"Dienstag"|
\item \lstinline|wochentag 30 4 1777| $\to$ \lstinline|"Mittwoch"|
\end{itemize}
\section{Logische Operatoren, Infix-Notation}
Die booleschen Funktionen und sowie nicht seien wie folgt definiert:

\begin{lstlisting}
und :: Bool -> Bool -> Bool
und = (&&)

nicht :: Bool -> Bool
nicht = not
\end{lstlisting}

Programmieren Sie die zweistelligen booleschen Funktionen \lstinline|oder|, \lstinline|darausFolgt|, \lstinline|genauDannWenn| und \lstinline|entwederOder| unter ausschließlicher Verwendung von \lstinline|und|
und \lstinline|nicht|. Testen Sie Ihre Funktionen.

\section{Rekursion über den natürlichen Zahlen}
In den folgenden Teilaufgaben dürfen keine anderen arithmetischen Operationen als die beiden vordefinierten Funktionen \lstinline|succ| und \lstinline|pred| verwendet werden.
\subsection{}
Programmieren Sie eine Funktion \lstinline|plus :: Integer -> Integer -> Integer| welche Summen natürlicher Zahlen berechnet. Nutzen Sie aus, dass \lstinline|a + b = succ(pred a + b)| und \lstinline|0 + a = a| gilt.
\subsection{}
Programmieren Sie eine Funktion \lstinline|mal :: Integer -> Integer -> Integer| welche Produkte natürlicher Zahlen berechnet. Nutzen Sie aus, dass \lstinline|a * b = b + (pred a * b)| und \lstinline|0 * a = 0| bzw. \lstinline|1 * a = a| gilt.
\subsection{}
Programmieren Sie eine Funktion \lstinline|hoch :: Integer -> Integer -> Integer| welche Potenzen natürlicher Zahlen berechnet. 
\subsection{}
Die Reihe \lstinline|plus|, \lstinline|mal|, \lstinline|hoch| kann beliebig fortgesetzt werden. Programmieren Sie eine Funktion \\\lstinline|meta :: Integer -> Integer -> Integer -> Integer| sodass \lstinline|meta 0 a b| der Summe \lstinline|plus a b|, \lstinline|meta 1 a b| dem Produkt \lstinline|mal a b|, \lstinline|meta 2 a b| der Potenz \lstinline|hoch a b| usw. entspricht. Was berechnet \lstinline|meta 3 a b|?
\newpage
\section{Datentypen}
\subsection{}
Geben Sie für die folgenden Typdeklarationen die Namen und Typen der automatisch generierten Konstruktoren und Selektoren an:
\begin{lstlisting}
data Ampelfarbe = Rot | Gelb | Gruen
data Menge = Stueck Integer | Kilogramm Double | Liter Double
data HausOderPostfach = Haus {
    strasse :: String,
    hausNummer :: Integer
  }
  | Postfach {
    postfachNummer :: Integer
  }
data Adresse = Adresse {
    hausOderPostfach :: HausOderPostfach,
    ort :: String,
    plz :: Integer
  }
\end{lstlisting}
\subsection{}
Geben Sie für den Datentyp Menge eine äquivalente Deklaration an, bei der automatisch Selektorfunktionen generiert werden. Wählen Sie sinnvolle Namen für die Selektorfunktionen.
\subsection{}
Geben Sie für die Datentypen HausOderPostfach und Adresse äquivalente Deklarationen an, bei denen nicht automatisch Selektoren generiert werden.
\subsection{}
Programmieren Sie aufbauend auf den Typdeklarationen aus Teilaufgabe 1 Selektorfunktionen mit den folgenden Signaturen:\\
\begin{lstlisting}
strasseVonAdresse :: Adresse -> String
postfachNummerVonAdresse :: Adresse -> Integer
\end{lstlisting}
\section{Pattern-Matching}
Gegeben sind die folgenden Typdeklarationen:
\begin{lstlisting}
data SI = SI String Integer
data SIListe = Leer | NichtLeer SI SIListe
data VielleichtSI = Nicht | Doch SI
\end{lstlisting}
\subsection{}
Bestimmen Sie die Typen der Variablen in den folgenden Mustern. Begründen Sie
Ihre Antworten kurz.
\begin{lstlisting}
loesche :: SI -> SIListe -> SIListe
loesche a Leer = error "nicht da"
loesche (SI b c) (NichtLeer (SI d e) f ) 
               | b == d && c == e = f
               | otherwise        = NichtLeer (SI d e) (loesche (SI b c) f )

ersetze :: SI -> SIListe -> SIListe
ersetze g Leer = Leer
ersetze h@(SI i _) (NichtLeer j@(SI k _) l) = 
               if i == k then NichtLeer h (ersetze h l)
                         else NichtLeer j (ersetze h l)
                                                        
findeAnIdx :: Integer -> SIListe -> VielleichtSI
findeAnIdx _ Leer                        = Nicht
findeAnIdx m (NichtLeer n o) | m == 0    = Doch n
                             | otherwise = findeAnIdx ( pred m) o
\end{lstlisting}
Hinweis:\\
Das $@$ bindet den "`zerlegten"' Datentyp an den vorstehenden Bezeichner, damit dieser nicht wieder zusammen gesetzt werden muss.
\subsection{}
Gegeben sind folgende Definitionen:
\begin{lstlisting}
liste1 = NichtLeer (SI "Bert" 7)
         (NichtLeer (SI "Bianca" 9)
         (NichtLeer (SI "Bert" 7) Leer))
liste2 = NichtLeer (SI "Bert" 8)
         (NichtLeer (SI "Robert" 7) Leer)
liste3 = NichtLeer (SI "Robert" 8)
         (NichtLeer (SI "Robert" 7) Leer)
\end{lstlisting}
Geben Sie die Resultate folgender Funktionsanwendungen an. Begründen Sie Ihre
Antworten kurz.\\
a) 
\begin{lstlisting}
loesche (SI "Bert" 7) liste1
\end{lstlisting}
b)
\begin{lstlisting}
ersetze (SI "Robert" 9) liste2
\end{lstlisting}
c)
\begin{lstlisting}
ersetze (SI "Robert" 9) liste3
\end{lstlisting}
d)
\begin{lstlisting}
findeAnIdx 1 liste3
\end{lstlisting}
\section{Arithmetische Ausdrücke}
Wir betrachten einen rekursiven Datentyp \lstinline|Ausdruck|, dessen Werte arithmetische Ausdrücke darstellen. Als Operationen sind Addition, Subtraktion und Multiplikation zugelassen, atomare Einheiten können Konstanten und Variablen sein. Die Deklaration
von \lstinline|Ausdruck| ist wie folgt:
\begin{lstlisting}
data Ausdruck = Konstante Integer
  | Variable String
  | Summe Ausdruck Ausdruck
  | Differenz Ausdruck Ausdruck
  | Produkt Ausdruck Ausdruck
  deriving (Show) -- ermoeglicht Ausgabe der Werte im GHCi
\end{lstlisting}
Der Ausdruck $x + (3 - x) \cdot y$ wird beispielsweise repräsentiert durch:
\begin{lstlisting}
Summe (Variable "x") (Produkt (Differenz (Konstante 3) (Variable "x")) 
  (Variable "y"))
\end{lstlisting}
\subsection{}
Geben Sie die Repräsentationen folgender arithmetischer Ausdrücke an:\\
a) $-x$\\
b) $3 \cdot x + y$\\
c) $3 \cdot (x + y)$\\
d) $3 + x \cdot y$
\subsection{}
Geben Sie die arithmetischen Ausdrücke an, die folgendermaßen repräsentiert
werden:\\
a) \begin{lstlisting}
Produkt (Differenz (Variable "x") (Konstante 0)) (Variable "y")
\end{lstlisting}
b) \begin{lstlisting}
Differenz (Variable "x") (Produkt (Konstante 0) (Variable "y"))
\end{lstlisting}
c) \begin{lstlisting}
Summe (Produkt (Konstante 3) (Variable "z")) (Konstante 5)
\end{lstlisting}
\subsection{}
Programmieren Sie eine Haskell-Funktion \lstinline|ausdruckNachString::Ausdruck -> String|
welche arithmetische Ausdrücke in Zeichenfolgen konvertiert. Die ausgegebenen
Zeichenfolgen sollen die arithmetischen Ausdrücke in Haskell-Syntax darstellen. Es ist z.B. folgendes möglich:
\begin{lstlisting}
ausdruckNachString (Differenz (Konstante 3) (Variable "x")) = "3 - x"
\end{lstlisting}
Hinweis:
Zum Aneinanderhängen von Zeichenfolgen kann der Operator \lstinline|++| verwendet werden.
\begin{lstlisting}
(++) :: String -> String -> String
\end{lstlisting}
Ganzzahlen können durch Anwenden der Funktion \lstinline|show| in
ihre Zifferndarstellung konvertiert werden. Es ergibt sich beispielsweise
\begin{lstlisting}
show 325 = "325"
\end{lstlisting}
\section{Rekursion über Listen}
Die folgenden Teilaufgaben sollen ohne Verwendung vordefinierter Bibliotheksfunktionen gelöst werden. Berücksichtigen Sie, dass die zu programmierenden Funktionen nicht für alle Eingaben sinnvoll definiert sind. In diesen Fällen sollen entsprechende Fehlermeldungen ausgegeben werden.
\subsection{}	
Programmieren Sie eine Haskell-Funktion \lstinline|elementAnIdx :: Integer -> [el] -> el|, die zu einem Index \lstinline|i| und einer Liste das i-te Element dieser Liste bestimmt. Das erste Element einer nichtleeren Liste habe dabei den Index $0$. Es ergibt sich zum Beispiel:\\
\lstinline|elementAnIdx 2 [3, 4, 5, 6] = 5|
\subsection{}
Programmieren Sie eine Haskell-Funktion \lstinline|snoc :: [el] -> el -> [el]|, die an eine Liste ein weiteres Element anhängt. Es ergibt sich zum Beispiel:\\
\lstinline|snoc [1, 2, 3] 5 = [1, 2, 3, 5]|
\subsection{}
Programmieren Sie eine Haskell-Funktion \lstinline|rueckwaerts :: [el] -> [el]|, welche eine Liste umdreht, die Elemente der Liste also in umgekehrter Reihenfolge zurück gibt.
\subsection{}
Programmieren Sie eine Haskell-Funktion \lstinline|praefix :: Integer -> [el] -> [el]|, die für eine gegebene Zahl $n$ und eine Liste $l$ die Liste der ersten $n$ Elemente von $l$ zurück gibt.
\subsection{}
Programmieren Sie eine Haskell-Funktion \lstinline|suffix :: Integer -> [el] -> [el]|, die für eine Zahl $n$ und eine Liste $l$ die Liste der letzten $n$ Elemente von $l$ zurückgibt.
\end{document}
