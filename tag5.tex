\subtitle{Tag vier - ein bisschen noch} 
\date{28.03.2014}

\begin{frame}[plain]
\titlepage
\end{frame}

\section{Das Array}
\begin{frame}
\frametitle{Das Array}
\begin{block}{\vspace*{-3ex}}
\begin{itemize}
	\item In Haskell existieren nicht nur Listen zur Speicherung und Verarbeitung von Daten sondern auch zwei Array Formen
	\item Arrays in Haskell besitzen immer eine feste Größe die bei der Erstellung angegeben wird
\end{itemize}
\end{block}
\end{frame}

\begin{frame}
\frametitle{Statische Arrays}
\begin{block}{\vspace*{-3ex}}
\begin{itemize}
	\item Es muss das Modul \lstinline|Data.Array.IArray| importiert werden
	\item Die einzelnen Elemente eines Arrays werden mit dem \lstinline|!| Operator angesprochen\\ z.B. \lstinline|a!5| gibt das Element mit dem Index $5$ aus dem Array $a$ wieder 
\end{itemize}
\only<2>{
\begin{center}
\scalebox{0.8}{\begin{tikzpicture}[
      nonterminal/.style={
         rectangle,
         minimum size=5.0mm,
         very thick,
         draw=blue!50!black!50,
         top color=white,
         bottom color=blue!50!black!20,
         font=\itshape\scriptsize,
         %text height=1.5ex,
         %text depth=.25ex
      },
      terminal/.style={
         rounded rectangle,
         minimum size=5.0mm,
         very thick,draw=black!50,
         top color=white,bottom color=black!20,
         font=\ttfamily\scriptsize,
         %text height=1.5ex,
         %text depth=.25ex
      },
      skip loop/.style={
         to path={-- ++(0,#1) -| (\tikztotarget)}
      },
      point/.style={coordinate},>=stealth',thick,draw=black!50,
      tip/.style={->,shorten >=0.007pt},every join/.style={rounded corners},
      hv path/.style={to path={-| (\tikztotarget)}},
      vh path/.style={to path={|- (\tikztotarget)}},
      %text height=1.5ex,text depth=.25ex
    ]
    \matrix[column sep=5.0mm, row sep=3.0mm] {
\node (n11) [circle, draw, inner sep=0pt, minimum size=0.75ex] {}; & \node (n12) [terminal] {listArray}; & \node (n13) [terminal] {(}; & \node (n14) [nonterminal] {Start}; & \node (n15) [terminal] {,}; & \node (n16) [nonterminal] {Ende}; & \node (n17) [terminal] {)}; & \node (n18) [terminal] {[}; & \node (n19) [nonterminal] {values}; & \node (n110) [terminal] {]}; & \node (n111) [circle, draw, inner sep=0pt, minimum size=0.75ex] {};\\
    };
  { [start chain]
       \chainin (n11);
       \chainin (n12)    [join=by tip];
  }
  { [start chain]
       \chainin (n12);
       \chainin (n13)    [join=by tip];
  }
  { [start chain]
       \chainin (n13);
       \chainin (n14)    [join=by tip];
  }
  { [start chain]
       \chainin (n14);
       \chainin (n15)    [join=by tip];
  }
  { [start chain]
       \chainin (n15);
       \chainin (n16)    [join=by tip];
  }
  { [start chain]
       \chainin (n16);
       \chainin (n17)    [join=by tip];
  }
  { [start chain]
       \chainin (n17);
       \chainin (n18)    [join=by tip];
  }
  { [start chain]
       \chainin (n18);
       \chainin (n19)    [join=by tip];
  }
  { [start chain]
       \chainin (n19);
       \chainin (n110)    [join=by tip];
  }
  { [start chain]
       \chainin (n110);
       \chainin (n111)    [join=by tip];
  }
\end{tikzpicture}
}
\end{center}
Als Array Index kann jeder Datentyp verwendet werden, welcher die Typklasse \lstinline|Ix| implementiert
}
\only<3>{
\begin{center}
\scalebox{0.7}{\begin{tikzpicture}[
      nonterminal/.style={
         rectangle,
         minimum size=5.0mm,
         very thick,
         draw=blue!50!black!50,
         top color=white,
         bottom color=blue!50!black!20,
         font=\itshape\scriptsize,
         %text height=1.5ex,
         %text depth=.25ex
      },
      terminal/.style={
         rounded rectangle,
         minimum size=5.0mm,
         very thick,draw=black!50,
         top color=white,bottom color=black!20,
         font=\ttfamily\scriptsize,
         %text height=1.5ex,
         %text depth=.25ex
      },
      skip loop/.style={
         to path={-- ++(0,#1) -| (\tikztotarget)}
      },
      point/.style={coordinate},>=stealth',thick,draw=black!50,
      tip/.style={->,shorten >=0.007pt},every join/.style={rounded corners},
      hv path/.style={to path={-| (\tikztotarget)}},
      vh path/.style={to path={|- (\tikztotarget)}},
      %text height=1.5ex,text depth=.25ex
    ]
    \matrix[column sep=5.0mm, row sep=3.0mm] {
\node (n11) [circle, draw, inner sep=0pt, minimum size=0.75ex] {}; & \node (n12) [terminal] {array}; & \node (n13) [terminal] {(}; & \node (n14) [nonterminal] {Start}; & \node (n15) [terminal] {,}; & \node (n16) [nonterminal] {Ende}; & \node (n17) [terminal] {)}; & \node (n18) [terminal] {[}; & \node (n19) [terminal] {(}; & \node (n110) [nonterminal] {key}; & \node (n111) [terminal] {,}; & \node (n112) [nonterminal] {value}; & \node (n113) [terminal] {)}; & \node (n114) [terminal] {]}; & \node (n115) [circle, draw, inner sep=0pt, minimum size=0.75ex] {};\\
    };
  { [start chain]
       \chainin (n11);
       \chainin (n12)    [join=by tip];
  }
  { [start chain]
       \chainin (n12);
       \chainin (n13)    [join=by tip];
  }
  { [start chain]
       \chainin (n13);
       \chainin (n14)    [join=by tip];
  }
  { [start chain]
       \chainin (n14);
       \chainin (n15)    [join=by tip];
  }
  { [start chain]
       \chainin (n15);
       \chainin (n16)    [join=by tip];
  }
  { [start chain]
       \chainin (n16);
       \chainin (n17)    [join=by tip];
  }
  { [start chain]
       \chainin (n17);
       \chainin (n18)    [join=by tip];
  }
  { [start chain]
       \chainin (n18);
       \chainin (n19)    [join=by tip];
  }
  { [start chain]
       \chainin (n19);
       \chainin (n110)    [join=by tip];
  }
  { [start chain]
       \chainin (n110);
       \chainin (n111)    [join=by tip];
  }
  { [start chain]
       \chainin (n111);
       \chainin (n112)    [join=by tip];
  }
  { [start chain]
       \chainin (n112);
       \chainin (n113)    [join=by tip];
  }
  { [start chain]
       \chainin (n113);
       \chainin (n114)    [join=by tip];
  }
  { [start chain]
       \chainin (n114);
       \chainin (n115)    [join=by tip];
  }
\end{tikzpicture}
}
\end{center}
In diesem Fall wird dem Array eine Liste mit Tupeln übergeben bei dem das erste Element der "`Primarykey"' ist (wie eine Map)
}
\end{block}
\end{frame}

\begin{frame}[fragile]
\frametitle{Statische Arrays}
\begin{lstlisting}
listArray :: (Ix i, IArray a e) => (i, i) 
    -> [e] -> a i e
myArray1 = (listArray ('a','e') [10..15]) 
    :: Array Char Int
\end{lstlisting}	
\pause
\begin{lstlisting}
array :: (Ix i, IArray a e) => (i, i) 
    -> [(i, e)] -> a i e
myArray2 = (array (1,5) [(k,k*2)| 
    k <- [1..5]):: Array Int Int
\end{lstlisting}
\begin{alertblock}{Achtung}
Die Anzahl der Listen Elemente und der Platz müssen nicht übereinstimmen, solange das Array nicht ausgegeben (\lstinline|show a|) wird. 
\end{alertblock}
\end{frame}

\begin{frame}[fragile]
\frametitle{Statische Arrays}
\begin{lstlisting}
accumArray :: (IX i, IArray a e) => (e -> e' -> e) 
   -> e -> (i,i) -> [(i, e')] -> a i e

myArray3 = (accumArray (+) 0 (0,4) [(i `mod` 5
   , 1) | i <- [1..123]]) :: Array Int Int
\end{lstlisting}	
\pause
\begin{block}{\vspace*{-3ex}}
\lstinline|array (0,4) [(0,24),(1,25),(2,25),(3,25),(4,24)]|
\end{block}
\end{frame}

\begin{frame}
\frametitle{Wichtige Array Funktionen}
\begin{block}{\vspace*{-3ex}}
\begin{itemize}
\item \lstinline|amap| ist die Array-Form der $map$ Funktion für Listen
\item \lstinline|elems| wandelt das Array in eine Liste um (nur die Werte)
\item \lstinline|assocs| wandelt das Array in eine Liste von Tupeln der Form $(k, v)$ um
\item Der \textbackslash\textbackslash$\;$Operator (update) ändert in einem Array die gegebenen Wertpaare.
\end{itemize}
\end{block}
\end{frame}

\begin{frame}
\frametitle{Statische vs. dynamische Arrays}
\begin{block}{\vspace*{-3ex}}
\begin{itemize}
\item Bei einem Update mit dem \textbackslash\textbackslash$\;$Operator (update) wird bei statischen das gesamte Array kopiert und die Änderungen vorgenommen
\item Somit dauert es bei statischen länger als bei dynamischen
\end{itemize}
\end{block}
\end{frame}

\begin{frame}
\frametitle{Dynamische Arrays}
\begin{block}{\vspace*{-3ex}}
\begin{itemize}
\item Import von \lstinline|Data.Array.Diff|
\item Funktionen heißen gleich nur Typ ist \lstinline|DiffArray| statt \lstinline|Array|
\item Besitzen zwar eine Konstante Zeit beim Update
\item Aber erhöhte Zugriffszeit beim Lesen
\item Durch geschickte Array Konstruktion kann jedoch fast vollständig auf Updates verzichtet werden
\end{itemize}
\end{block}
\end{frame}

\begin{frame}[fragile]
\frametitle{Haskell API}
\begin{block}{\vspace*{-3ex}}
für weitere Datentypen und deren Funktionen siehe:\\
haskell.org/hoogle
\end{block}
\end{frame}

\section{Monaden}
\begin{frame}
\frametitle{Monaten}
\begin{block}{\vspace*{-3ex}}
\begin{itemize}
\item Monaden sind ein mathematisches Konzept aus der Kategorientheorie
\item Werden eingesetzt um Funktionen miteinander zu kombinieren 
\item Ist in Haskell eine polymorphe Datenstruktur mit speziellen Funktionen
\item Das Prinzip ist:
\begin{itemize}
\item Sequenzialisierung gemäß des Continuation-style Programming \\
	der Kontrollfluss kehrt nicht zum Aufrufer zurück sondern geht zur Nachfolgefunktion 
\item Darstellung und Transformation eines versteckten Zustands (Hiding)
\item Sicherung von Single-Threadedness dadurch, weil keine dagegen verstoßende Funktion benutzt werden kann
\end{itemize}
\end{itemize}
\end{block}
\end{frame}

\begin{frame}[fragile]
\frametitle{Monaden - Klasse}
\begin{lstlisting}
class  Monad m  where
    -- verbinden zweiter Funktionen
    -- Ergebnis ist Argument der zweiten Funktion
    (>>=)   :: forall a b. m a -> (a -> m b) -> m b
    -- verbindet zwei Funktionen aber verwirft jedes 
    -- Ergebnis (wie in Imperativen Sprachen)
    (>>)    :: forall a b. m a -> m b -> m b
    m >> k  = m >>= \_ -> k
    -- fuegt einen Wert in den Monaden Typ ein
    return  :: a -> m a
    -- gibt eine Fehlernachricht zurueck
    fail    :: String -> m a
    fail    = error
\end{lstlisting}	
\end{frame}

\begin{frame}[fragile]
\frametitle{Monaden - Klasse}
\begin{lstlisting}
add :: Maybe Int -> Maybe Int -> Maybe Int
add mA mB = case mA of
    Nothing -> Nothing
    Just a    -> case mB of
                 Nothing -> Nothing
                 Just b  -> Just (a + b)
\end{lstlisting}	
\pause
\begin{lstlisting}
add' :: Maybe Int -> Maybe Int -> Maybe Int
add' mA mB = mA >>= (\a ->
              mB >>= (\b ->
                return (a + b)))
\end{lstlisting}	
\end{frame}

\begin{frame}
\frametitle{Do-Notation}
\begin{block}{\vspace*{-3ex}}
\begin{itemize}
\item Mit der do-Notation werden Monaden (\lstinline|>>=|) zusammengefasst \\ pro Zeile
\item Somit ist es syntaktischer Zucker
\item Für die Verwendung der do-Notation sind 4 Regeln zu beachten
\end{itemize}
\end{block}
\end{frame}

\begin{frame}[fragile]
\frametitle{Do-Notation - Regel 1}
\begin{block}{\vspace*{-3ex}}
\begin{itemize}
\item Einzelne Anweisungen benötigen keine Umformung.
\item Das \lstinline|do| wird einfach weggelassen.
\end{itemize}
\end{block}
\begin{lstlisting}
do 
  e
\end{lstlisting}	
\pause
\begin{lstlisting}
e
\end{lstlisting}	
\end{frame}

\begin{frame}[fragile]
\frametitle{Do-Notation - Regel 2}
\begin{block}{\vspace*{-3ex}}
\begin{itemize}
\item Wird der Rückgabewert nicht benötigt
\item Dann wird die Anweisung nach vorne gezogen
\end{itemize}
\end{block}
\begin{lstlisting}
do 
  e
  <Anweisung>
\end{lstlisting}	
\pause
\begin{lstlisting}
e >>= \_ -> do
  <Anweisungen>
\end{lstlisting}	
\end{frame}

\begin{frame}[fragile]
\frametitle{Do-Notation - Regel 3}
\begin{block}{\vspace*{-3ex}}
\begin{itemize}
\item Wird der Rückgabewert mit Pattern-Matching ausgewertet
\item Dann muss eine Hilfsfunktion dies übernehmen
\end{itemize}
\end{block}
\begin{lstlisting}
do 
  pattern <- e
  <Anweisungen>
\end{lstlisting}	
\pause
\begin{lstlisting}
let ok pattern = do 
    <Anweisungen>
  ok _ = fail "Fehler" 
in e >>= ok
\end{lstlisting}	
\end{frame}

\begin{frame}[fragile]
\frametitle{Do-Notation - Regel 4}
\begin{block}{\vspace*{-3ex}}
\begin{itemize}
\item Wird ein Wert mit \lstinline|let| gespeichert, kann dies vor das \lstinline|do| gezogen werden
\item Das \lstinline|in| ist im do-Block optional
\end{itemize}
\end{block}
\begin{lstlisting}
do 
  let <Deklaration>
  in <Anweisungen>
\end{lstlisting}	
\pause
\begin{lstlisting}
let <Deklaration>
in do
      <Anweisungen>
\end{lstlisting}	
\end{frame}

\begin{frame}[fragile]
\frametitle{Do-Notation - If-Then-Else}
\begin{block}{Wir erwarten}
\begin{lstlisting}
f = do 
  if <irgendwas> then
    <Anweisungen>
  else 
    <Anweisungen>
\end{lstlisting}	
\end{block}
\vspace*{-3ex}
\begin{block}{Aber!}
\begin{lstlisting}
f = 
  if <irgendwas> then do
    <Anweisungen>
  else do
    <Anweisungen>
\end{lstlisting}	
\end{block}
\end{frame}

\begin{frame}[fragile]
\frametitle{Do-Notation - Beispiel}
\begin{lstlisting}
add' :: Maybe Int -> Maybe Int -> Maybe Int
add' mA mB = mA >>= (\a ->
               mB >>= (\b ->
                 return (a + b)))
\end{lstlisting}	
\pause
\begin{lstlisting}
add :: Maybe Int -> Maybe Int -> Maybe Int
add mA mB = do
  a <- mA
  b <- mB
  return (a + b)
\end{lstlisting}
\end{frame}

\begin{frame}
\frametitle{Vordefinierte Monaden}
\begin{block}{\vspace*{-3ex}}
\begin{itemize}
\item Writer - für Debug / Logging / Tracing
\item Reader - zum Lesen von gemeinsamen Zuständen (global)
\item State - Verknüpfung von Writer und Reader für gemeinsame Zustände
\\ für zustandsbasierte Rechnungen
\end{itemize}
\end{block}
\end{frame}

\section{IO}

\begin{frame}[fragile]
\frametitle{Hangman}
\begin{lstlisting}
import Data.Char
import System.IO

w = "Lambda"
maxI = 5

hangman :: String -> Int -> IO ()
hangman cs i | i > maxI = putStrLn "\nverloren"
             | all (`elem` cs) (map toLower w) = 
               putStrLn "\ngewonnen"
\end{lstlisting}	
\end{frame}

\begin{frame}[fragile]
\frametitle{Hangman}
\begin{lstlisting}
hangman cs i = 
  do
    putStrLn " "
    printWord cs
    putStrLn "\nWelcher Buchstabe?"
    c <- getChar >>= (return.toLower)
    if (c `elem` (map toLower w)) then
      hangman (c:cs) i 
    else do
      putStrLn $ "\n" ++ (show (i+1)) ++ " falsch\n"
      hangman cs (i+1)
\end{lstlisting}	
\end{frame}

\begin{frame}[fragile]
\frametitle{Hangman}
\begin{lstlisting}
printWord :: String -> IO ()
printWord cs = mapM_ pC w 
  where
    pC x  | toLower x `elem` cs = putChar x
          | otherwise = putChar '_'

          
main = do
  hSetBuffering stdin NoBuffering
  hangman " " 0
\end{lstlisting}	
\end{frame}

\begin{frame}
\frametitle{Das O in IO}
\begin{block}{\vspace*{-3ex}}
\begin{itemize}
  \item \lstinline|print :: Show a => a -> IO()| \\ gibt jeden Datentyp der \lstinline|Show| implementiert aus
  \item \lstinline|putChar :: Char -> IO()| \\ gibt ein \lstinline|Char| aus
  \item \lstinline|putStr :: String -> IO()|\\ gibt einen \lstinline|String| aus \\
  \lstinline|putStr = sequence_map putChar|
\end{itemize}
\end{block}
\end{frame}

\begin{frame}
\frametitle{Das O in IO}
\begin{block}{\vspace*{-3ex}}
\begin{itemize}
  \item \lstinline|writeFile :: FilePath -> String -> IO()| 
  \item \lstinline|type Filepath = String|
  \item Schreibt den \lstinline|String| mittels Textstrom in eine Datei
\end{itemize}
\end{block}
\end{frame}

\begin{frame}
\frametitle{Das I in IO}
\begin{block}{\vspace*{-3ex}}
\begin{itemize}
  \item \lstinline|readLn :: Read a => IO a|\\ liest jeden Datentyp der \lstinline|Read| implementiert ein
  \item \lstinline|getChar :: IO Char|\\ liest ein Char ein
  \item \lstinline|getLine :: IO String|\\ liest die ganze Zeile ein als String
  \item Die Pufferung der Eingaben ist über \lstinline|hSetBuffering| einstellbar\\ für Windows bekommt es der GHC trotzdem nicht hin :(
\end{itemize}
\end{block}
\end{frame}

\begin{frame}
\frametitle{Das I in IO}
\begin{block}{\vspace*{-3ex}}
\begin{itemize}
  \item \lstinline|readFile :: FilePath -> IO String|
  \item \lstinline|type Filepath = String|
  \item Liest die Datei als String ein
\end{itemize}
\end{block}
\end{frame}

\begin{frame}[fragile]
\frametitle{Wortsuche in einer Datei}
\begin{lstlisting}
isInfix :: String -> String -> Maybe String
isInfix [] _ = Nothing
isInfix t w | w == take (length w) t = Just t
            | otherwise = isInfix (tail t) w
\end{lstlisting}	
\end{frame}

\begin{frame}[fragile]
\frametitle{Wortsuche in einer Datei}
\begin{lstlisting}
main = do
  putStrLn "Dateipfad: "
  filepath <- getLine
  putStrLn "gesuchtes Wort: "
  w <- getLine
  c <- readFile filepath
  case (isInfix c w) of
    Nothing -> putStrLn "nicht enthalten"
    Just s -> putStrLn $
        "an Stelle " ++ (take 100 s)
\end{lstlisting}	
\end{frame}

\section*{Zusammenfassung}

\begin{frame}
\frametitle{Zusammenfassung}
\begin{block}{\vspace*{-3ex}}
Ab jetzt seid ihr auf dem Wissenstand, auf dem ich bin.

% The following outlook is optional.
\vskip0pt plus.5fill
\begin{itemize}
  \item Nun seid ihr dran
  \begin{itemize}
    \item Stellt Fragen
    \item Schlagt auf www.haskell.org/hoogle nach 
    \item Oder vergesst alles schnell wieder
  \end{itemize}
\end{itemize}
\end{block}
\end{frame}
