\documentclass[fleqn,11pt,aspectratio=43]{beamer}
\usepackage[utf8x]{inputenc}
\usepackage[T1]{fontenc}
\usepackage[ngerman]{babel}
\usepackage{stmaryrd}
\usepackage{amsfonts}
\usepackage{amssymb}
\usepackage{amsmath}
\usepackage{microtype}

\usepackage{listings}
\usepackage{color}
\usepackage{pxfonts}

\definecolor{mygreen}{RGB}{51,141,120}
\definecolor{myblue}{RGB}{0,128,180}
\definecolor{myviolet}{RGB}{118,0,118}

\lstset{ %
  language=Haskell,
  backgroundcolor=\color{white},         % choose the background color
  basicstyle=\ttfamily\footnotesize,     % size of fonts used for the code
  numbers=none,
  breaklines=true,                       % automatic line breaking only at whitespace
  captionpos=b,                          % sets the caption-position to bottom
  commentstyle=\color{mygreen},    % comment style
  escapeinside={\%*}{*)},                % if you want to add LaTeX within your code
  keywordstyle=\color{myblue}\bfseries, % keyword style
  stringstyle=\color{myviolet},    % string literal style
  frame=single,
  tabsize=2
}
\usepackage{tikz}
\usetikzlibrary{
  arrows,
  shapes.misc,
  shapes.arrows,
  chains,
  matrix,
  positioning,
  scopes,
  decorations.pathmorphing,
  shadows,
  backgrounds
}


\usetheme[%
  %cmyk,%<cmyk/rgbprint>,          Auswahl des Farbmodells
  orange,%<blue/orange/green/violet> Auswahl des Sekundärfarbklangs
  %dark,%<light,medium>        Auswahl der Helligkeit
  %colorhead,%    Farbig hinterlegte Kopfleiste
  %colorfoot,%    Farbig hinterlegt Fußleiste auf Titelseite
  %colorblocks,%   Blöcke Farbig hinterlegen
  %nopagenum,%    Keine Seitennumer in Fußzeile
  %nodate,%       Kein Datum in Fußleiste
  %tocinheader,%   Inhaltsverzeichnis in Kopfleiste
  %tinytocinheader,% kleines Kopfleisten-Inhaltsverzeichnis
  %widetoc,%      breites Kopfleisten-Inhaltsverzeichnis
  %narrowtoc,%    schmales Kopfleisten-Inhaltsverzeichnis
  %nosubsectionsinheader,%  Keine subsections im Kopfleisten-Inhaltsverzeichnis
  %nologoinfoot,% Kein Logo im Fußbereich darstellen
  ]{tubs}

\titlegraphic[scaled]{\includegraphics{img/titlepicture.jpg}}
\logo{\includegraphics{img/logo_mit_text.pdf}}

\AtBeginSection[]{
  \begin{frame}[noframenumbering] 
  		\scriptsize
  		\frametitle{Überblick}  
  		\tableofcontents[currentsection, hideallsubsections]
  \end{frame}
}

\AtBeginSubsection[]{
  \begin{frame}[noframenumbering]
    	\scriptsize 
  		\frametitle{\insertsectionhead - \insertsubsectionhead} 
  		\tableofcontents[ 
  			currentsubsection, 
  		    sectionstyle=show/hide, 
  		   	subsectionstyle=show/shaded/hide] 
  \end{frame}
}


\title{Programmieren für Fortgeschrittene - \\eine Einführung in Haskell}

\author[Stephan Mielke]{\emph{Stephan Mielke}}
\institute[TU Braunschweig, IPS]{Technische Universität Braunschweig, IPS}


\begin{document}
\subtitle{Tag 4 --- ein bisschen noch} 
\date{12.01.2015}

\begin{frame}[plain]
\titlepage
\end{frame}

\section{Lazy}
\begin{frame}[fragile]
\frametitle{Lazy}
\begin{block}{Betrachte folgende Funktion}
\begin{lstlisting}
rechne :: Double -> Double -> Double
rechne a b = if a > 10 
             then a + b
             else a
\end{lstlisting}
\end{block}
\only<1>{\begin{block}{Was erwartet ihr beim Aufruf von}
\lstinline|rechne 12 6|
\end{block}}
\only<2>{\begin{block}{und bei}
\lstinline|rechne 9 (10 / 0)|
\end{block}}
\end{frame}


\begin{frame}[fragile]
\frametitle{Lazy}
\begin{block}{Wir betrachten}
\begin{lstlisting}
prims :: [Integer]-> Int -> [Integer]
prims _ 0 		= []
prims (p:xs) i 	= (:) p $prims 
                [x|x<- xs, mod x p > 0] $i + 1
\end{lstlisting}
Terminiert die Funktion?
\end{block}
\pause 
\begin{block}{Terminiert die Funktion auch bei der Eingabe von}
\lstinline|primes [2..] 1|
\end{block}
\end{frame}

\frame{\frametitle{Lazy}
\begin{block}{\vspace*{-3ex}}
\begin{itemize}
  \item Haskell verwendet die Lazy-Evaluation für Ausdrücke 
  \item Lazy $\equiv$ Call-by-Need
  \item Dadurch sind Funktionen nicht strikt
\end{itemize}
\end{block}
}

\begin{frame}[fragile]
\frametitle{Lazy --- unendliche Listen}
\begin{block}{\vspace*{-3ex}}
\begin{itemize}
  \item Es werden vom Start an eine bestimmte Anzahl an Elemente erstellt
  \item Wenn weitere Elemente benötigt werden, werden diese neu erstellt
  \item Wird immer nur ein Abschnitt benötigt, wird der Start wieder gelöscht\\
  		stellt es euch als "`Ringpuffer"' vor
  \item Wenn jedoch alle Elemente benötigt werden $\to$ bis Speicher voll 
  \item z.B.: für die Liste \lstinline![x**x | x <- [1..]]! sind wir bei Element $42$, die Elemente $32$--$52$ sind bereits berechnet
\end{itemize}
\begin{lstlisting}[basicstyle=\tiny]
drop 32 $ take 52 [x**x | x <- [1..]]
[1.2911004008776103e50,1.1756638905368616e52,1.1025074993541487e54,1.0638735892371651e56,1.0555134955777783e58,1.075911801979994e60,1.125951474620712e62,1.2089258196146292e64,1.330877630632712e66,1.5013093754529656e68,1.7343773367030268e70,2.05077382356061e72,2.4806364445134117e74,3.068034630079427e76,3.877924263464449e78,5.007020782634593e80,6.600972468621954e82,8.881784197001252e84,1.2192113050946485e87,1.7067655527413216e89]
\end{lstlisting}
\end{block}
\end{frame}

\frame{\frametitle{Lazy --- Parameter und Ausdrücke}
\begin{block}{\vspace*{-3ex}}
\begin{itemize}
  \item Haskell verwendet Call-by-Need
  \item Call-by-Need ist eine Form des Call-by-Name
\end{itemize}
\end{block}
}

\frame{\frametitle{Lazy --- Call-by-Name}
\begin{block}{\vspace*{-3ex}}
\begin{itemize}
  \item Ausdrücke werden nicht sofort ausgewertet sondern nur übergeben
  \item \lstinline|max (4 + 6) (10 / 0)|
  \only<2-5>{\item $\Rightarrow$ \lstinline|if (4 + 6) > (10 / 0) then (4 + 6) else (10 / 0)|}
  \only<3-5>{\item $\Rightarrow$ \lstinline|(4 + 6) > (10 / 0)|}
  \only<4-5>{\item $\Rightarrow$ \lstinline|10 > (10 / 0)|}
  \only<5>{\item $\Rightarrow \lightning$}
\end{itemize}
\end{block}
}


\begin{frame}[fragile]
\frametitle{Lazy --- Call-by-Need}
\begin{block}{\vspace*{-3ex}}
\begin{itemize}
  \item Call-by-Need erweitert Call-by-Name um Sharing
  \item Sharing: gleiche Ausdrücke werden nur einmal ausgewertet
\end{itemize}
\end{block}
\begin{lstlisting}
(product [1..50000]) - (product [1..49999])
...

take 9 [div 100 (10 - x) | x <- [1..]]
[11,12,14,16,20,25,33,50,100]
\end{lstlisting}
\end{frame}

\section{Funktionen höherer Ordnung (HOF)}
\subsection{Allgemeines zu HOF}
\frame{\frametitle{Funktionen höherer Ordnung (HOF)}
\begin{block}{\vspace*{-3ex}}
\begin{itemize}
	\item Funktionen können als Parameter nicht nur Ausdrücke sondern auch Funktionen erhalten
	\item Dieses wird im Funktionskopf angegeben
\end{itemize}
\begin{center}
\scalebox{0.6}{\begin{tikzpicture}[
      nonterminal/.style={
         rectangle,
         minimum size=5.0mm,
         very thick,
         draw=blue!50!black!50,
         top color=white,
         bottom color=blue!50!black!20,
         font=\itshape\scriptsize,
         %text height=1.5ex,
         %text depth=.25ex
      },
      terminal/.style={
         rounded rectangle,
         minimum size=5.0mm,
         very thick,draw=black!50,
         top color=white,bottom color=black!20,
         font=\ttfamily\scriptsize,
         %text height=1.5ex,
         %text depth=.25ex
      },
      skip loop/.style={
         to path={-- ++(0,#1) -| (\tikztotarget)}
      },
      point/.style={coordinate},>=stealth',thick,draw=black!50,
      tip/.style={->,shorten >=0.007pt},every join/.style={rounded corners},
      hv path/.style={to path={-| (\tikztotarget)}},
      vh path/.style={to path={|- (\tikztotarget)}},
      %text height=1.5ex,text depth=.25ex
    ]
    \matrix[column sep=5.0mm, row sep=3.0mm] {
 &  &  &  &  &  &  &  &  &  & \node (n111) [point] {}; &  &  &  &  &  &  &  &  & \\
 &  &  &  &  & \node (n26) [terminal] {->}; & \node (n27) [nonterminal] {Typ}; &  &  &  & \node (n211) [point] {}; &  &  &  &  & \node (n216) [terminal] {->}; & \node (n217) [nonterminal] {Typ}; &  &  & \\
\node (n31) [circle, draw, inner sep=0pt, minimum size=0.75ex] {}; & \node (n32) [nonterminal] {Funktionsname}; & \node (n33) [terminal] {::}; & \node (n34) [point] {}; & \node (n35) [point] {}; &  &  & \node (n38) [point] {}; & \node (n39) [terminal] {(}; & \node (n310) [point] {}; & \node (n311) [nonterminal] {HOF Typ}; & \node (n312) [terminal] {->}; & \node (n313) [point] {}; & \node (n314) [terminal] {)}; & \node (n315) [point] {}; &  &  & \node (n318) [point] {}; & \node (n319) [point] {}; & \node (n320) [circle, draw, inner sep=0pt, minimum size=0.75ex] {};\\
    };
  { [start chain]
       \chainin (n31);
       \chainin (n32)    [join=by tip];
  }
  { [start chain]
       \chainin (n32);
       \chainin (n33)    [join=by tip];
  }
  { [start chain]
       \chainin (n33);
       \chainin (n34)    [join];
  }
  { [start chain]
       \chainin (n34);
       \chainin (n35)    [join];
  }
  { [start chain]
       \chainin (n35);
       \chainin (n38)    [join];
  }
  { [start chain]
       \chainin (n26);
       \chainin (n27)    [join=by tip];
  }
  { [start chain]
       \chainin (n38);
       \chainin (n27)    [join=by {vh path,tip}];
  }
  { [start chain]
       \chainin (n26);
       \chainin (n35)    [join=by {hv path,tip}];
  }
  { [start chain]
       \chainin (n38);
       \chainin (n39)    [join=by tip];
  }
  { [start chain]
       \chainin (n39);
       \chainin (n310)    [join];
  }
  { [start chain]
       \chainin (n310);
       \chainin (n311)    [join=by tip];
  }
  { [start chain]
       \chainin (n311);
       \chainin (n312)    [join=by tip];
  }
  { [start chain]
       \chainin (n312);
       \chainin (n313)    [join];
  }
  { [start chain]
       \chainin (n313);
       \chainin (n314)    [join=by tip];
  }
  { [start chain]
       \chainin (n314);
       \chainin (n315)    [join];
  }
  { [start chain]
       \chainin (n315);
       \chainin (n318)    [join];
  }
  { [start chain]
       \chainin (n216);
       \chainin (n217)    [join=by tip];
  }
  { [start chain]
       \chainin (n318);
       \chainin (n217)    [join=by {vh path,tip}];
  }
  { [start chain]
       \chainin (n216);
       \chainin (n315)    [join=by {hv path,tip}];
  }
  { [start chain]
       \chainin (n318);
       \chainin (n319)    [join];
  }
  { [start chain]
       \chainin (n319);
       \chainin (n320)    [join=by tip];
  }
  { [start chain]
       \chainin (n313);
       \chainin (n211)    [join=by vh path];
       \chainin (n310)    [join=by {hv path,tip}];
  }
  { [start chain]
       \chainin (n319);
       \chainin (n111)    [join=by vh path];
       \chainin (n34)    [join=by {hv path,tip}];
  }
\end{tikzpicture}
}
\end{center}
\end{block}
}

\begin{frame}[fragile]
\frametitle{Funktionen höherer Ordnung (HOF)}
\begin{lstlisting}
filter :: (Int -> Bool) -> [Int] -> [Int]
filter do []     = []
filter do (x:xs) | do x      = x : filter do xs
                 | otherwise = filter do xs
\end{lstlisting}
\end{frame}

\subsection{Funktionskomposition}
\begin{frame}[fragile]
\frametitle{Funktionskomposition}
\begin{block}{Wie vermeiden wir am besten "`Klammerungswirrwarr"'}
\lstinline|f (f (f (f (f (f (f (f (f x))))))))|
\end{block}
\pause
\begin{block}{Mit dem "`Punkt"'-Operator können wir Funktionen verbinden}
\lstinline|(f.f.f.f.f.f.f.f.f) x|
\end{block}
\pause
\begin{block}{Oder dem \lstinline|$|-Operator die Auswertungsreihenfolge verändern}
\lstinline|f $ f $ f $ f $ f $ f $ f $ f $ f x|
\end{block}
\end{frame}

\begin{frame}[fragile]
\frametitle{Funktionskomposition}
\begin{block}{Der "`Punkt"'-Operator ist definiert mit}
\begin{lstlisting}
(.) :: (b -> c) -> (a -> b) -> a -> c
(.) outerFunc innerFunc x = outerFunc (innerFunc x)
\end{lstlisting}
Das Resultat der inneren Funktion wird auf die äußere angewandt
\end{block}
\end{frame}

\begin{frame}[fragile]
\frametitle{Funktionskomposition}
\begin{block}{Der \lstinline|$|-Operator ist definiert mit}
\begin{lstlisting}
($) :: (a -> b) -> a -> b
($) func x = func x
\end{lstlisting}
Die Funktion wird auf das Resultat von dem Ausdruck der "`rechts"' vom Operator steht angewandt
\end{block}
\end{frame}


\section{Lambda Ausdrücke}
\begin{frame}
\frametitle{Anonyme Funktionen}
\begin{block}{\vspace*{-3ex}}
\begin{itemize}
  \item Haskell unterstützt anonyme Funktionen in Form von $\lambda$-Ausdrücken
  \item Das $\lambda$-Symbol wird durch "`$\backslash$"' repräsentiert\\ $\lambda x \leadsto$ \lstinline|\x|
  \item Aufbau: \\
  %\item {} $\backslash <Parameter_1> \ldots <Parameter_n>\; ->\; <Ausdruck>$
  \item Durch das Currying können $\lambda$-Ausdrücke mehrere Argumente besitzen
  \item Es gelten alle bekannten Regeln für die $\lambda$-Notation
\end{itemize}
\begin{center}
\scalebox{0.8}{\input{images/lambda.tex}}
\end{center}
\end{block}
\end{frame}

\begin{frame}[fragile]
\frametitle{Beispiele - Lambda-Ausdrücke}
\begin{lstlisting}
plus = \x -> \y -> x + y

istKleiner = \x -> \y ->  x < y

g = \x -> \y -> (\y -> \x -> (y,x)) y x
\end{lstlisting}
\end{frame}

\begin{frame}[fragile]
\frametitle{Beispiele --- Lambda-Ausdrücke}
\begin{lstlisting}
-- y "Operator"
y f = f (y f)

fac = y (\f n -> if n > 0 then n * f (n - 1) else 1)
\end{lstlisting}
\only<2-3>{
\begin{exampleblock}{Aufruf}
\lstinline|fac 5|
\end{exampleblock}}
\only<3>{
\begin{exampleblock}{Ausgabe}
\lstinline|120|
\end{exampleblock}}
\end{frame}

\section{Beispiele für Currying, Lambda und HOF}

\begin{frame}[allowframebreaks,fragile]
\frametitle{Menge}
\begin{lstlisting}
module MySet(myPrint, myElem, mySingleSet, myBoolSet, myListSet, myAdd, myUnion, myIntersect, myDiff, myFilter, myForAll, myExists, mySubset, myEquals) where

type Set = Int -> Bool

myStart = -10000
myEnd   = 10000

myPrint :: Set -> [Int]
myPrint set = myPrint' set myStart myEnd
  where
    myPrint' s min max | min > max    = []
      | myElem s min = min : (myPrint' s (min + 1) max)
      | otherwise    = myPrint' s (min + 1) max
myElem :: Set -> Int -> Bool
myElem s e = s e

mySingleSet :: Int -> Set
mySingleSet e = (==) e

myBoolSet :: (Int -> Bool) -> Set
myBoolSet p = p

myListSet :: [Int] -> Set
myListSet list = \x -> myFind list x
  where
    myFind :: [Int] -> Int -> Bool
    myFind []     _ = False
    myFind (y:ys) x = x == y || myFind ys x

myAdd :: Set -> Int -> Set
myAdd s i = (mySingleSet i) `myUnion` s

myUnion :: Set -> Set -> Set
myUnion a b = \x -> ((myElem a x) || (myElem b x))

myIntersect :: Set -> Set -> Set
myIntersect a b = \x -> ((myElem a x) && (myElem b x))

myDiff :: Set -> Set -> Set
myDiff a b = \x -> ((myElem a x) && not (myElem b x))

myFilter :: Set -> (Int -> Bool) -> Set
myFilter s f = \x -> (f x && s x)


myForAll :: Set -> (Int -> Bool) -> Int -> Int -> Bool
myForAll s p min max 
     | min > max    = True
     | myElem s min = p min && myForAll s p (min + 1) max
     | otherwise    = myForAll s p (min + 1) max

myExists :: Set -> (Int -> Bool) -> Int -> Int -> Bool
myExists s p min max = not (myForAll s (not.p) min max)

mySubset :: Set -> Set -> Bool
mySubset a b = myForAll a (\x -> myExists b (\y -> x == y) myStart myEnd) myStart myEnd 

myEquals :: Set -> Set -> Bool
myEquals a b = (mySubset a b) && (mySubset b a)
\end{lstlisting}
\end{frame}

\section{Listen API}

\begin{frame}[fragile]{Listen API}
\begin{block}{Basis Funktionen}
\begin{itemize}
\item \lstinline|(++) :: [a] -> [a] -> [a]| \\ hängt die zweite an die erste Liste 
\item \lstinline|last :: [a] -> a| \\ letztes Element
\item \lstinline|init :: [a] -> [a]| \\ ohne das letzte Element
\item \lstinline|null :: [a] -> Bool| \\ testet, ob leer
\item \lstinline|length :: [a] -> Int| \\ Länge der Liste
\end{itemize}
\end{block}
\end{frame}

\begin{frame}[fragile]{Listen API}
\begin{block}{transformations Funktionen}
\begin{itemize}
\item \lstinline|map :: (a -> b) -> [a] -> [b]| \\ wendet die Funktion \lstinline|a -> b| auf jedes Element an
\item \lstinline|reverse :: [a] -> [a]| \\ dreht die Liste um
\item \lstinline|subsequences :: [a] -> [[a]]| \\ berechnet die Potenzmenge
\item \lstinline|permutations :: [a] -> [[a]]| \\ berechnet alle Permutationen
\end{itemize}
\end{block}
\end{frame}

\begin{frame}[fragile]{Listen API}
\begin{lstlisting}
map (^2) [1..10]
[1,4,9,16,25,36,49,64,81,100]

reverse [-x | x <- [1,,10]]
[-10,-9,-8,-7,-6,-5,-4,-3,-2,-1]

Data.List.subsequences "123"
["","1","2","12","3","13","23","123"]

Data.List.permutations "123"
["123","213","321","231","312","132"]
\end{lstlisting}
\end{frame}

\begin{frame}[fragile]{Listen API}
\begin{block}{Auflösungs-Funktionen}
\begin{itemize}
\item \lstinline|foldl :: (b -> a -> b) -> b -> [a] -> b| \\ wendet die Funktion \lstinline|(b -> a -> b)| mit dem übergebenen Element von links nach rechts an
\item \lstinline|foldr :: (a -> b -> b) -> b -> [a] -> b| \\ wendet die Funktion \lstinline|(a -> b -> b)| mit dem übergebenen Element von rechts nach links an
\end{itemize}
\end{block}
\begin{lstlisting}
foldl (-) 10 [1..5]
10-1-2-3-4-5 = -15

foldr (-) 10 [1..5]
1-(2-(3-(4-(5-10)))) = -7
\end{lstlisting}
\end{frame}

\begin{frame}[fragile]{Listen API}
\begin{block}{Abschnitts-Funktionen}
\begin{itemize}
\item \lstinline|take :: Int -> [a] -> [a]| \\ nimmt die ersten $n$ Elemente
\item \lstinline|drop :: Int -> [a] -> [a]| \\ löscht die ersten $n$ Elemente
\item \lstinline|span :: (a -> Bool) -> [a] -> ([a], [a])| \\ spaltet eine Liste in zwei Teile
\item \lstinline|group :: Eq a => [a] -> [[a]]| \\ gruppiert gleiche Elemente zu eigenen Listen in einer Liste
\item \lstinline|inits :: [a] -> [[a]]| \\ gibt alle möglichen Restlisten zurück
\item \lstinline|tails :: [a] -> [[a]]| \\ gibt alle möglichen Startlisten zurück
\end{itemize}
\end{block}
\end{frame}

\begin{frame}[fragile]{Listen API}
\begin{block}{Such-Funktionen}
\begin{itemize}
\item \lstinline|elem :: Eq a => a -> [a] -> Bool | \\ prüft, ob enthalten
\item \lstinline|lookup :: Eq a => a -> [(a, b)] -> Maybe b| \\ gibt den Value einer Key-Value-Liste
\item \lstinline|find :: (a -> Bool) -> [a] -> Maybe a| \\ gibt das erste Element zurück, das die Bedingung erfüllt
\item \lstinline|filter :: (a -> Bool) -> [a] -> [a]| \\ filtert die Liste (Positiv-Filter)
\item \lstinline|partition :: (a -> Bool) -> [a] -> ([a], [a])| \\ teilt eine Liste in zwei Listen mit "`erfüllt"' und "`erfüllt nicht"'
\end{itemize}
\end{block}
\end{frame}

\begin{frame}[fragile]{Listen API}
\begin{block}{ZIP-Funktionen}
\begin{itemize}
\item \lstinline|zip :: [a] -> [b] -> [(a, b)]| \\ bildet Tupel aus den Elementen, Anzahl ist beschränkt nach der kürzesten
\item \lstinline|zipWith :: (a -> b -> c) -> [a] -> [b] -> [c]| \\ statt Tupel zu bilden werden die Elemente an die Funktion übergeben
\item \lstinline|unzip :: [(a, b)] -> ([a], [b])| \\ statt Tupel zu bilden werden die Elemente an die Funktion übergeben
\end{itemize}
\end{block}
\end{frame}

\begin{frame}[fragile]{Listen API}
\begin{lstlisting}
zip [42..] [5..10]
[(42,5),(43,6),(44,7),(45,8),(46,9),(47,10)]

zipWith (*)  [42..] [5..10]
[210,258,308,360,414,470]

unzip [(x, x^x) | x <- [3..5]]
([3,4,5],[27,256,3125])
\end{lstlisting}
\end{frame}

\section*{Ende~}
\begin{frame}[noframenumbering]{Danke}
\centering
Vielen Dank für die Aufmerksamkeit und das Interesse!
\end{frame}

\end{document}