\subtitle{Tag zwei - etwas mehr} 
\date{25.03.2014}

\begin{frame}[plain]
\titlepage
\end{frame}
\section{Currying - allgemein}
\begin{frame}
\frametitle{Deklaration von Funktionen in $\lambda$ Notation}
\only<1-3>{
\vspace*{-2ex}
	\begin{block}{Lambda Currying}
	\begin{center}
	$f = \lambda x_1 \to \lambda x_2 \to \dots \lambda x_n \to e$
	\end{center}
	\end{block}
}
\only<2-3>{
\vspace*{-3ex}
	\begin{exampleblock}{Funkion Currying}
	\begin{center}
	\scalebox{0.6}{\input{images/FunktionCurry.tex}}
	\end{center}
	\end{exampleblock}
}
\only<3>{
\vspace*{-3ex}
	\begin{exampleblock}{Lambda Currying}
	\begin{center}
	\scalebox{0.6}{\begin{tikzpicture}[
      nonterminal/.style={
         rectangle,
         minimum size=5.0mm,
         very thick,
         draw=blue!50!black!50,
         top color=white,
         bottom color=blue!50!black!20,
         font=\itshape\scriptsize,
         %text height=1.5ex,
         %text depth=.25ex
      },
      terminal/.style={
         rounded rectangle,
         minimum size=5.0mm,
         very thick,draw=black!50,
         top color=white,bottom color=black!20,
         font=\ttfamily\scriptsize,
         %text height=1.5ex,
         %text depth=.25ex
      },
      skip loop/.style={
         to path={-- ++(0,#1) -| (\tikztotarget)}
      },
      point/.style={coordinate},>=stealth',thick,draw=black!50,
      tip/.style={->,shorten >=0.007pt},every join/.style={rounded corners},
      hv path/.style={to path={-| (\tikztotarget)}},
      vh path/.style={to path={|- (\tikztotarget)}},
      %text height=1.5ex,text depth=.25ex
    ]
    \matrix[column sep=5.0mm, row sep=3.0mm] {
 &  & \node (n13) [point] {}; &  &  &  &  & \node (n18) [point] {}; &  &  &  & \\
\node (n21) [circle, draw, inner sep=0pt, minimum size=0.75ex] {}; & \node (n22) [point] {}; & \node (n23) [nonterminal] {Bezeichner}; & \node (n24) [terminal] {=}; & \node (n25) [point] {}; & \node (n26) [point] {}; & \node (n27) [terminal] {$\lambda$}; & \node (n28) [nonterminal] {Parameter}; & \node (n29) [terminal] {->}; & \node (n210) [point] {}; & \node (n211) [nonterminal] {Ausdruck}; & \node (n212) [circle, draw, inner sep=0pt, minimum size=0.75ex] {};\\
    };
  { [start chain]
       \chainin (n21);
       \chainin (n22)    [join];
  }
  { [start chain]
       \chainin (n22);
       \chainin (n23)    [join=by tip];
  }
  { [start chain]
       \chainin (n23);
       \chainin (n24)    [join=by tip];
  }
  { [start chain]
       \chainin (n24);
       \chainin (n25)    [join];
  }
  { [start chain]
       \chainin (n25);
       \chainin (n26)    [join];
  }
  { [start chain]
       \chainin (n26);
       \chainin (n27)    [join=by tip];
  }
  { [start chain]
       \chainin (n27);
       \chainin (n28)    [join=by tip];
  }
  { [start chain]
       \chainin (n28);
       \chainin (n29)    [join=by tip];
  }
  { [start chain]
       \chainin (n29);
       \chainin (n210)    [join];
  }
  { [start chain]
       \chainin (n210);
       \chainin (n211)    [join=by tip];
  }
  { [start chain]
       \chainin (n211);
       \chainin (n212)    [join=by tip];
  }
  { [start chain]
       \chainin (n22);
       \chainin (n13)    [join=by vh path];
       \chainin (n25)    [join=by {hv path,tip}];
  }
  { [start chain]
       \chainin (n210);
       \chainin (n18)    [join=by vh path];
       \chainin (n26)    [join=by {hv path,tip}];
  }
\end{tikzpicture}
}
	\end{center}
	\end{exampleblock}
}
\end{frame}

\begin{frame}
\frametitle{Deklaration von Funktionen in $\lambda$ Notation}
\only<1-3>{
\vspace*{-2ex}
	\begin{block}{Lambda Uncurrying}
	\begin{center}
	$f = \lambda(x_1, x_2, \dots, x_n) \to e$
	\end{center}
	\end{block}
}
\only<2-3>{
\vspace*{-3ex}
	\begin{exampleblock}{Funkion Uncurrying}
	\begin{center}
	\scalebox{0.6}{\begin{tikzpicture}[
      nonterminal/.style={
         rectangle,
         minimum size=5.0mm,
         very thick,
         draw=blue!50!black!50,
         top color=white,
         bottom color=blue!50!black!20,
         font=\itshape\scriptsize,
         %text height=1.5ex,
         %text depth=.25ex
      },
      terminal/.style={
         rounded rectangle,
         minimum size=5.0mm,
         very thick,draw=black!50,
         top color=white,bottom color=black!20,
         font=\ttfamily\scriptsize,
         %text height=1.5ex,
         %text depth=.25ex
      },
      skip loop/.style={
         to path={-- ++(0,#1) -| (\tikztotarget)}
      },
      point/.style={coordinate},>=stealth',thick,draw=black!50,
      tip/.style={->,shorten >=0.007pt},every join/.style={rounded corners},
      hv path/.style={to path={-| (\tikztotarget)}},
      vh path/.style={to path={|- (\tikztotarget)}},
      %text height=1.5ex,text depth=.25ex
    ]
    \matrix[column sep=5.0mm, row sep=3.0mm] {
 &  &  &  &  & \node (n16) [nonterminal] {Parameter}; & \node (n17) [terminal] {,}; &  &  &  &  & \\
\node (n21) [circle, draw, inner sep=0pt, minimum size=0.75ex] {}; & \node (n22) [nonterminal] {Bezeichner}; & \node (n23) [terminal] {(}; & \node (n24) [nonterminal] {Parameter}; & \node (n25) [point] {}; &  &  & \node (n28) [point] {}; & \node (n29) [terminal] {)}; & \node (n210) [terminal] {=}; & \node (n211) [nonterminal] {Ausdruck}; & \node (n212) [circle, draw, inner sep=0pt, minimum size=0.75ex] {};\\
    };
  { [start chain]
       \chainin (n21);
       \chainin (n22)    [join=by tip];
  }
  { [start chain]
       \chainin (n22);
       \chainin (n23)    [join=by tip];
  }
  { [start chain]
       \chainin (n23);
       \chainin (n24)    [join=by tip];
  }
  { [start chain]
       \chainin (n24);
       \chainin (n25)    [join];
  }
  { [start chain]
       \chainin (n25);
       \chainin (n28)    [join];
  }
  { [start chain]
       \chainin (n17);
       \chainin (n16)    [join=by tip];
  }
  { [start chain]
       \chainin (n28);
       \chainin (n17)    [join=by {vh path,tip}];
  }
  { [start chain]
       \chainin (n16);
       \chainin (n25)    [join=by {hv path,tip}];
  }
  { [start chain]
       \chainin (n28);
       \chainin (n29)    [join=by tip];
  }
  { [start chain]
       \chainin (n29);
       \chainin (n210)    [join=by tip];
  }
  { [start chain]
       \chainin (n210);
       \chainin (n211)    [join=by tip];
  }
  { [start chain]
       \chainin (n211);
       \chainin (n212)    [join=by tip];
  }
\end{tikzpicture}
}
	\end{center}
	\end{exampleblock}
}
\only<2>{
	\begin{alertblock}{ABER}
	Das Tupel $(x_1, x_2, \dots, x_n)$ ist ein eigener Datentyp
	\end{alertblock}
}
\only<3>{
\vspace*{-3ex}
	\begin{exampleblock}{Lambda Uncurrying}
	\begin{center}
	\scalebox{0.6}{\begin{tikzpicture}[
      nonterminal/.style={
         rectangle,
         minimum size=5.0mm,
         very thick,
         draw=blue!50!black!50,
         top color=white,
         bottom color=blue!50!black!20,
         font=\itshape\scriptsize,
         %text height=1.5ex,
         %text depth=.25ex
      },
      terminal/.style={
         rounded rectangle,
         minimum size=5.0mm,
         very thick,draw=black!50,
         top color=white,bottom color=black!20,
         font=\ttfamily\scriptsize,
         %text height=1.5ex,
         %text depth=.25ex
      },
      skip loop/.style={
         to path={-- ++(0,#1) -| (\tikztotarget)}
      },
      point/.style={coordinate},>=stealth',thick,draw=black!50,
      tip/.style={->,shorten >=0.007pt},every join/.style={rounded corners},
      hv path/.style={to path={-| (\tikztotarget)}},
      vh path/.style={to path={|- (\tikztotarget)}},
      %text height=1.5ex,text depth=.25ex
    ]
    \matrix[column sep=5.0mm, row sep=3.0mm] {
 &  & \node (n13) [point] {}; &  &  &  &  &  &  & \node (n110) [nonterminal] {Parameter}; & \node (n111) [terminal] {,}; &  &  &  &  & \\
\node (n21) [circle, draw, inner sep=0pt, minimum size=0.75ex] {}; & \node (n22) [point] {}; & \node (n23) [nonterminal] {Bezeichner}; & \node (n24) [terminal] {=}; & \node (n25) [point] {}; & \node (n26) [terminal] {$\lambda$}; & \node (n27) [terminal] {(}; & \node (n28) [nonterminal] {Parameter}; & \node (n29) [point] {}; &  &  & \node (n212) [point] {}; & \node (n213) [terminal] {)}; & \node (n214) [terminal] {->}; & \node (n215) [nonterminal] {Ausdruck}; & \node (n216) [circle, draw, inner sep=0pt, minimum size=0.75ex] {};\\
    };
  { [start chain]
       \chainin (n21);
       \chainin (n22)    [join];
  }
  { [start chain]
       \chainin (n22);
       \chainin (n23)    [join=by tip];
  }
  { [start chain]
       \chainin (n23);
       \chainin (n24)    [join=by tip];
  }
  { [start chain]
       \chainin (n24);
       \chainin (n25)    [join];
  }
  { [start chain]
       \chainin (n25);
       \chainin (n26)    [join=by tip];
  }
  { [start chain]
       \chainin (n26);
       \chainin (n27)    [join=by tip];
  }
  { [start chain]
       \chainin (n27);
       \chainin (n28)    [join=by tip];
  }
  { [start chain]
       \chainin (n28);
       \chainin (n29)    [join];
  }
  { [start chain]
       \chainin (n29);
       \chainin (n212)    [join];
  }
  { [start chain]
       \chainin (n111);
       \chainin (n110)    [join=by tip];
  }
  { [start chain]
       \chainin (n212);
       \chainin (n111)    [join=by {vh path,tip}];
  }
  { [start chain]
       \chainin (n110);
       \chainin (n29)    [join=by {hv path,tip}];
  }
  { [start chain]
       \chainin (n212);
       \chainin (n213)    [join=by tip];
  }
  { [start chain]
       \chainin (n213);
       \chainin (n214)    [join=by tip];
  }
  { [start chain]
       \chainin (n214);
       \chainin (n215)    [join=by tip];
  }
  { [start chain]
       \chainin (n215);
       \chainin (n216)    [join=by tip];
  }
  { [start chain]
       \chainin (n22);
       \chainin (n13)    [join=by vh path];
       \chainin (n25)    [join=by {hv path,tip}];
  }
\end{tikzpicture}
}
	\end{center}
	\end{exampleblock}
}
\end{frame}

\begin{frame}[fragile]
\frametitle{Deklaration von Funktionen} 	
%\begin{block}{\vspace*{-3ex}}
\begin{lstlisting}
plus :: Int -> Int -> Int
plus a b = a + b
\end{lstlisting}
%\end{block}
\only<1>{
	\begin{exampleblock}{\vspace*{-3ex}}
	\begin{center}
	\scalebox{0.6}{\input{images/FunktionCurry.tex}}
	\end{center}
	\end{exampleblock}
}
\only<2-3>{
\begin{exampleblock}{Aufruf}
\lstinline|plus 6 7|
\end{exampleblock}}
\only<3>{
\begin{exampleblock}{Ausgabe}
\lstinline|13|
\end{exampleblock}}
\end{frame}

\begin{frame}[fragile]
\frametitle{Deklaration von Funktionen} 	
%\begin{block}{\vspace*{-3ex}}
\begin{lstlisting}
plus' :: (Int, Int) -> Int
plus' (a, b) = a + b
\end{lstlisting}
%\end{block}
\only<1>{
\vspace*{-3ex}
	\begin{exampleblock}{\vspace*{-3ex}}
	\begin{center}
	\scalebox{0.6}{\begin{tikzpicture}[
      nonterminal/.style={
         rectangle,
         minimum size=5.0mm,
         very thick,
         draw=blue!50!black!50,
         top color=white,
         bottom color=blue!50!black!20,
         font=\itshape\scriptsize,
         %text height=1.5ex,
         %text depth=.25ex
      },
      terminal/.style={
         rounded rectangle,
         minimum size=5.0mm,
         very thick,draw=black!50,
         top color=white,bottom color=black!20,
         font=\ttfamily\scriptsize,
         %text height=1.5ex,
         %text depth=.25ex
      },
      skip loop/.style={
         to path={-- ++(0,#1) -| (\tikztotarget)}
      },
      point/.style={coordinate},>=stealth',thick,draw=black!50,
      tip/.style={->,shorten >=0.007pt},every join/.style={rounded corners},
      hv path/.style={to path={-| (\tikztotarget)}},
      vh path/.style={to path={|- (\tikztotarget)}},
      %text height=1.5ex,text depth=.25ex
    ]
    \matrix[column sep=5.0mm, row sep=3.0mm] {
 &  &  &  &  & \node (n16) [nonterminal] {Parameter}; & \node (n17) [terminal] {,}; &  &  &  &  & \\
\node (n21) [circle, draw, inner sep=0pt, minimum size=0.75ex] {}; & \node (n22) [nonterminal] {Bezeichner}; & \node (n23) [terminal] {(}; & \node (n24) [nonterminal] {Parameter}; & \node (n25) [point] {}; &  &  & \node (n28) [point] {}; & \node (n29) [terminal] {)}; & \node (n210) [terminal] {=}; & \node (n211) [nonterminal] {Ausdruck}; & \node (n212) [circle, draw, inner sep=0pt, minimum size=0.75ex] {};\\
    };
  { [start chain]
       \chainin (n21);
       \chainin (n22)    [join=by tip];
  }
  { [start chain]
       \chainin (n22);
       \chainin (n23)    [join=by tip];
  }
  { [start chain]
       \chainin (n23);
       \chainin (n24)    [join=by tip];
  }
  { [start chain]
       \chainin (n24);
       \chainin (n25)    [join];
  }
  { [start chain]
       \chainin (n25);
       \chainin (n28)    [join];
  }
  { [start chain]
       \chainin (n17);
       \chainin (n16)    [join=by tip];
  }
  { [start chain]
       \chainin (n28);
       \chainin (n17)    [join=by {vh path,tip}];
  }
  { [start chain]
       \chainin (n16);
       \chainin (n25)    [join=by {hv path,tip}];
  }
  { [start chain]
       \chainin (n28);
       \chainin (n29)    [join=by tip];
  }
  { [start chain]
       \chainin (n29);
       \chainin (n210)    [join=by tip];
  }
  { [start chain]
       \chainin (n210);
       \chainin (n211)    [join=by tip];
  }
  { [start chain]
       \chainin (n211);
       \chainin (n212)    [join=by tip];
  }
\end{tikzpicture}
}
	\end{center}
	\end{exampleblock}
}
\only<2-3>{
\begin{exampleblock}{Aufruf}
\lstinline|plus' (6, 7)|
\end{exampleblock}}
\only<3>{
\begin{exampleblock}{Ausgabe}
\lstinline|13|
\end{exampleblock}}
\end{frame}

\section{Gültigkeitsbereiche}
\subsection{Block}
\begin{frame}
\frametitle{Block}
\begin{block}{\vspace*{-3ex}}
\begin{itemize}
  \item Definitionen im Block sind immer nur eine Stufe höher sichtbar
  \item Im Block ist alles Äußere sichtbar
\end{itemize}
\end{block}
\end{frame}

\begin{frame}
\frametitle{Block - Einrückungen}
\begin{block}{\vspace*{-3ex}}
In Haskell spielt das Layout des Quellcodes eine Rolle!
\begin{itemize}
  \item Blöcke werden durch gleiche Einrückungstiefe kenntlich gemacht
  \item Einzelne Deklarationen werden durch Zeilenumbrüche getrennt
  \item Beginnt eine neue Zeile gegenüber dem aktuellen Block \\
  \begin{itemize}
    \item Rechts eingerückt: aktuelle Zeile wird fortgesetzt
    \item Links eingerückt: aktueller Block wird beendet
    \item Direkt an seinem "`linken Rand darunter"', so wird der Block fortgesetzt bzw. eine neue Deklaration eingeleitet 
  \end{itemize}
\end{itemize}
\end{block}
\end{frame}

\begin{frame}[fragile]
\frametitle{Block} 
\begin{lstlisting}
outer a b =
  let inner c = a + b + c
  in inner 1
\end{lstlisting}
\end{frame}
\subsection{Module}
\begin{frame}
\frametitle{Module}
\begin{block}{\vspace*{-3ex}}
\begin{itemize}
  \item Das Programm kann in Module aufgeteilt werden
  \item Der Standard Modulname ist Main
  \item Module müssen mit einem Großbuchstaben beginnen
  \item Vorteile: \\
    \begin{itemize}
      \item Vereinfachung des Programmdesigns, Strukturierung
      \item Einfachere Isolation von Fehlern
      \item Einfaches Ändern von Teilkomponenten ohne Einfluss auf andere Teile
      \item Wiederverwendung von Code
    \end{itemize}
\end{itemize}
\end{block}
\end{frame}

\begin{frame}[fragile]
\frametitle{Module}
\begin{block}{\vspace*{-3ex}}
\begin{center}
\scalebox{1}{\input{images/Module.tex}}
\end{center}
\end{block} 
\end{frame}

\begin{frame}[fragile]
\frametitle{Module} 
\begin{lstlisting}
module Wurf where
weite :: Double -> Double -> Double
weite v0 phi = ((square v0) / 9.81) * sin (2 * phi)
square :: Double -> Double
square x = x * x
\end{lstlisting}
\begin{lstlisting}
module Foo where
import Wurf
foo ... = ... (weite v w) ...
bar ... = ... (square a) ...
\end{lstlisting}
\end{frame}

\begin{frame}
\frametitle{Module - Interfaces}
\only<1>{
\begin{block}{Import}
\begin{center}
\scalebox{1}{\begin{tikzpicture}[
      nonterminal/.style={
         rectangle,
         minimum size=5.0mm,
         very thick,
         draw=blue!50!black!50,
         top color=white,
         bottom color=blue!50!black!20,
         font=\itshape\scriptsize,
         %text height=1.5ex,
         %text depth=.25ex
      },
      terminal/.style={
         rounded rectangle,
         minimum size=5.0mm,
         very thick,draw=black!50,
         top color=white,bottom color=black!20,
         font=\ttfamily\scriptsize,
         %text height=1.5ex,
         %text depth=.25ex
      },
      skip loop/.style={
         to path={-- ++(0,#1) -| (\tikztotarget)}
      },
      point/.style={coordinate},>=stealth',thick,draw=black!50,
      tip/.style={->,shorten >=0.007pt},every join/.style={rounded corners},
      hv path/.style={to path={-| (\tikztotarget)}},
      vh path/.style={to path={|- (\tikztotarget)}},
      %text height=1.5ex,text depth=.25ex
    ]
    \matrix[column sep=5.0mm, row sep=3.0mm] {
\node (n11) [circle, draw, inner sep=0pt, minimum size=0.75ex] {}; & \node (n12) [terminal] {module}; & \node (n13) [nonterminal] {Name}; & \node (n14) [terminal] {where}; & \node (n15) [terminal] {LF}; & \node (n16) [point] {}; &  & \\
 &  &  &  &  &  &  & \\
 &  &  &  & \node (n35) [nonterminal] {Name}; & \node (n36) [terminal] {,}; &  & \\
\node (n41) [point] {}; & \node (n42) [terminal] {import}; & \node (n43) [nonterminal] {Name}; & \node (n44) [point] {}; &  &  & \node (n47) [point] {}; & \node (n48) [point] {};\\
 &  &  &  &  &  &  & \\
\node (n61) [point] {}; & \node (n62) [nonterminal] {Funktionendefinitionen}; & \node (n63) [circle, draw, inner sep=0pt, minimum size=0.75ex] {}; &  &  &  &  & \\
    };
  { [start chain]
       \chainin (n11);
       \chainin (n12)    [join=by tip];
  }
  { [start chain]
       \chainin (n12);
       \chainin (n13)    [join=by tip];
  }
  { [start chain]
       \chainin (n13);
       \chainin (n14)    [join=by tip];
  }
  { [start chain]
       \chainin (n14);
       \chainin (n15)    [join=by tip];
  }
  { [start chain]
       \chainin (n15);
       \chainin (n16)    [join=by tip];
  }
  { [start chain]
       \chainin (n41);
       \chainin (n42)    [join=by tip];
  }
  { [start chain]
       \chainin (n42);
       \chainin (n43)    [join=by tip];
  }
  { [start chain]
       \chainin (n43);
       \chainin (n44)    [join];
  }
  { [start chain]
       \chainin (n44);
       \chainin (n47)    [join];
  }
  { [start chain]
       \chainin (n36);
       \chainin (n35)    [join=by tip];
  }
  { [start chain]
       \chainin (n47);
       \chainin (n36)    [join=by {vh path,tip}];
  }
  { [start chain]
       \chainin (n35);
       \chainin (n44)    [join=by {hv path,tip}];
  }
  { [start chain]
       \chainin (n47);
       \chainin (n48)    [join=by tip];
  }
  { [start chain]
       \chainin (n61);
       \chainin (n62)    [join=by tip];
  }
  { [start chain]
       \chainin (n62);
       \chainin (n63)    [join=by tip];
  }
\end{tikzpicture}
}
\end{center}
\end{block}
}
\only<2>{
\begin{block}{Selektiver Import}
\begin{center}
\scalebox{0.7}{\begin{tikzpicture}[
      nonterminal/.style={
         rectangle,
         minimum size=5.0mm,
         very thick,
         draw=blue!50!black!50,
         top color=white,
         bottom color=blue!50!black!20,
         font=\itshape\scriptsize,
         %text height=1.5ex,
         %text depth=.25ex
      },
      terminal/.style={
         rounded rectangle,
         minimum size=5.0mm,
         very thick,draw=black!50,
         top color=white,bottom color=black!20,
         font=\ttfamily\scriptsize,
         %text height=1.5ex,
         %text depth=.25ex
      },
      skip loop/.style={
         to path={-- ++(0,#1) -| (\tikztotarget)}
      },
      point/.style={coordinate},>=stealth',thick,draw=black!50,
      tip/.style={->,shorten >=0.007pt},every join/.style={rounded corners},
      hv path/.style={to path={-| (\tikztotarget)}},
      vh path/.style={to path={|- (\tikztotarget)}},
      %text height=1.5ex,text depth=.25ex
    ]
    \matrix[column sep=5.0mm, row sep=3.0mm] {
\node (n11) [circle, draw, inner sep=0pt, minimum size=0.75ex] {}; & \node (n12) [terminal] {module}; & \node (n13) [nonterminal] {Name}; & \node (n14) [terminal] {where}; & \node (n15) [terminal] {LF}; & \node (n16) [point] {}; &  &  &  &  &  &  & \\
 &  &  &  &  &  &  &  &  &  &  &  & \\
 &  &  &  &  &  & \node (n37) [point] {}; &  &  &  &  &  & \\
 &  &  &  &  &  & \node (n47) [point] {}; &  &  &  &  &  & \\
\node (n51) [point] {}; & \node (n52) [terminal] {import}; & \node (n53) [point] {}; & \node (n54) [nonterminal] {Name}; & \node (n55) [point] {}; & \node (n56) [terminal] {(}; & \node (n57) [nonterminal] {Funktion}; & \node (n58) [terminal] {,}; & \node (n59) [terminal] {)}; & \node (n510) [point] {}; & \node (n511) [terminal] {,}; & \node (n512) [point] {}; & \node (n513) [point] {};\\
 &  &  &  &  &  &  &  &  &  &  &  & \\
\node (n71) [point] {}; & \node (n72) [nonterminal] {Funktionendefinitionen}; & \node (n73) [circle, draw, inner sep=0pt, minimum size=0.75ex] {}; &  &  &  &  &  &  &  &  &  & \\
    };
  { [start chain]
       \chainin (n11);
       \chainin (n12)    [join=by tip];
  }
  { [start chain]
       \chainin (n12);
       \chainin (n13)    [join=by tip];
  }
  { [start chain]
       \chainin (n13);
       \chainin (n14)    [join=by tip];
  }
  { [start chain]
       \chainin (n14);
       \chainin (n15)    [join=by tip];
  }
  { [start chain]
       \chainin (n15);
       \chainin (n16)    [join=by tip];
  }
  { [start chain]
       \chainin (n51);
       \chainin (n52)    [join=by tip];
  }
  { [start chain]
       \chainin (n52);
       \chainin (n53)    [join];
  }
  { [start chain]
       \chainin (n53);
       \chainin (n54)    [join=by tip];
  }
  { [start chain]
       \chainin (n54);
       \chainin (n55)    [join];
  }
  { [start chain]
       \chainin (n55);
       \chainin (n56)    [join=by tip];
  }
  { [start chain]
       \chainin (n56);
       \chainin (n57)    [join=by tip];
  }
  { [start chain]
       \chainin (n57);
       \chainin (n58)    [join=by tip];
  }
  { [start chain]
       \chainin (n58);
       \chainin (n59)    [join=by tip];
  }
  { [start chain]
       \chainin (n59);
       \chainin (n510)    [join];
  }
  { [start chain]
       \chainin (n510);
       \chainin (n511)    [join=by tip];
  }
  { [start chain]
       \chainin (n511);
       \chainin (n512)    [join];
  }
  { [start chain]
       \chainin (n512);
       \chainin (n513)    [join=by tip];
  }
  { [start chain]
       \chainin (n71);
       \chainin (n72)    [join=by tip];
  }
  { [start chain]
       \chainin (n72);
       \chainin (n73)    [join=by tip];
  }
  { [start chain]
       \chainin (n510);
       \chainin (n47)    [join=by vh path];
       \chainin (n55)    [join=by {hv path,tip}];
  }
  { [start chain]
       \chainin (n512);
       \chainin (n37)    [join=by vh path];
       \chainin (n53)    [join=by {hv path,tip}];
  }
\end{tikzpicture}
}
\end{center}
am Ende steht natürlich kein Komma
\end{block}
}
\only<3>{
\begin{block}{Negativ selektiver Import}
\begin{center}
\scalebox{0.7}{\input{images/ModuleImportNSelektiv.tex}}
\end{center}
am Ende steht natürlich kein Komma
\end{block}
}
\end{frame}

\begin{frame}[fragile]
\frametitle{Module - Interfaces} 
\begin{lstlisting}
module Wurf where
weite :: Double -> Double -> Double
weite v0 phi = ((square v0) / 9.81) * sin (2 * phi)
square :: Double -> Double
square x = x * x
\end{lstlisting}
\begin{lstlisting}
module Foo where
import Wurf(weite)
foo ... = ... (weite v w) ...
bar ... = ... (square a) ...
\end{lstlisting}
\begin{alertblock}{Achtung}
\lstinline|square| ist für \lstinline|bar| nicht definiert!
\end{alertblock}
\end{frame}

\begin{frame}[fragile]
\frametitle{Module - Interfaces} 
\begin{lstlisting}
module Wurf where
weite :: Double -> Double -> Double
weite v0 phi = ((square v0) / 9.81) * sin (2 * phi)
square :: Double -> Double
square x = x * x
\end{lstlisting}
\begin{lstlisting}
module Foo where
import Wurf hiding (weite)
foo ... = ... (weite v w) ...
bar ... = ... (square a) ...
\end{lstlisting}
\begin{alertblock}{Achtung}
\lstinline|weite| ist für \lstinline|foo| nicht definiert!
\end{alertblock}
\end{frame}

\begin{frame}
\frametitle{Module - Sichtbarkeit}
\begin{block}{\vspace*{-3ex}}
Module können festlegen was importiert werden darf
\end{block}
\begin{block}{\vspace*{-3ex}}
\begin{center}
\scalebox{0.7}{\begin{tikzpicture}[
      nonterminal/.style={
         rectangle,
         minimum size=5.0mm,
         very thick,
         draw=blue!50!black!50,
         top color=white,
         bottom color=blue!50!black!20,
         font=\itshape\scriptsize,
         %text height=1.5ex,
         %text depth=.25ex
      },
      terminal/.style={
         rounded rectangle,
         minimum size=5.0mm,
         very thick,draw=black!50,
         top color=white,bottom color=black!20,
         font=\ttfamily\scriptsize,
         %text height=1.5ex,
         %text depth=.25ex
      },
      skip loop/.style={
         to path={-- ++(0,#1) -| (\tikztotarget)}
      },
      point/.style={coordinate},>=stealth',thick,draw=black!50,
      tip/.style={->,shorten >=0.007pt},every join/.style={rounded corners},
      hv path/.style={to path={-| (\tikztotarget)}},
      vh path/.style={to path={|- (\tikztotarget)}},
      %text height=1.5ex,text depth=.25ex
    ]
    \matrix[column sep=5.0mm, row sep=3.0mm] {
 &  &  &  &  & \node (n16) [point] {}; &  &  &  &  &  & \\
\node (n21) [circle, draw, inner sep=0pt, minimum size=0.75ex] {}; & \node (n22) [terminal] {module}; & \node (n23) [nonterminal] {Name}; & \node (n24) [terminal] {(}; & \node (n25) [point] {}; & \node (n26) [nonterminal] {Funktion}; & \node (n27) [terminal] {,}; & \node (n28) [point] {}; & \node (n29) [terminal] {)}; & \node (n210) [terminal] {where}; & \node (n211) [terminal] {LF}; & \node (n212) [point] {};\\
 &  &  &  &  &  &  &  &  &  &  & \\
\node (n41) [point] {}; & \node (n42) [nonterminal] {Funktionensdefinitionen}; & \node (n43) [circle, draw, inner sep=0pt, minimum size=0.75ex] {}; &  &  &  &  &  &  &  &  & \\
    };
  { [start chain]
       \chainin (n21);
       \chainin (n22)    [join=by tip];
  }
  { [start chain]
       \chainin (n22);
       \chainin (n23)    [join=by tip];
  }
  { [start chain]
       \chainin (n23);
       \chainin (n24)    [join=by tip];
  }
  { [start chain]
       \chainin (n24);
       \chainin (n25)    [join];
  }
  { [start chain]
       \chainin (n25);
       \chainin (n26)    [join=by tip];
  }
  { [start chain]
       \chainin (n26);
       \chainin (n27)    [join=by tip];
  }
  { [start chain]
       \chainin (n27);
       \chainin (n28)    [join];
  }
  { [start chain]
       \chainin (n28);
       \chainin (n29)    [join=by tip];
  }
  { [start chain]
       \chainin (n29);
       \chainin (n210)    [join=by tip];
  }
  { [start chain]
       \chainin (n210);
       \chainin (n211)    [join=by tip];
  }
  { [start chain]
       \chainin (n211);
       \chainin (n212)    [join=by tip];
  }
  { [start chain]
       \chainin (n41);
       \chainin (n42)    [join=by tip];
  }
  { [start chain]
       \chainin (n42);
       \chainin (n43)    [join=by tip];
  }
  { [start chain]
       \chainin (n28);
       \chainin (n16)    [join=by vh path];
       \chainin (n25)    [join=by {hv path,tip}];
  }
\end{tikzpicture}
}
\end{center}
Am Ende steht natürlich kein Komma
\end{block}
\end{frame}

\begin{frame}[fragile]
\frametitle{Module - Sichtbarkeit} 
\begin{lstlisting}
module Wurf(weite) where
weite :: Double -> Double -> Double
weite v0 phi = ((square v0) / 9.81) * sin (2 * phi)
square :: Double -> Double
square x = x * x
\end{lstlisting}
\begin{lstlisting}
module Foo where
import Wurf
foo ... = ... (weite v w) ...
bar ... = ... (square a) ...
\end{lstlisting}
\begin{alertblock}{Achtung}
In \lstinline|Wurf| ist nur \lstinline|weite| sichtbar
\end{alertblock}
\end{frame}

\section[Bezeichner]{Überladung und Auflösung von Namen}
\begin{frame}
\frametitle{Überladung von Namen}
\begin{block}{\vspace*{-3ex}}
\begin{itemize}
  \item Funktionen können in Haskell nicht im selben Modul überladen werden
  \item Funktionen können nur flach in Blöcken überdeckt werden
  \item Überladene Funktionen müssen mit dem Modul Bezeichner angesprochen werden.
  \item Für Polymorphie werden Typklassen verwendet
\end{itemize}
\end{block}
\end{frame}

\begin{frame}[fragile]
\frametitle{Überladung von Namen} 
\begin{lstlisting}
maximum :: Int -> Int -> Int
maximum a b | a < b 	= b
            | otherwise = a

maximum :: Bool -> Bool -> Bool
maximum a b = a || b
\end{lstlisting}
\begin{alertblock}{Fehler}
Mehrfach-Definitionen sind unzulässig
\end{alertblock}
\end{frame}

\begin{frame}[fragile]
\frametitle{Überladung von Namen} 
\begin{lstlisting}
maximum :: Int -> Int -> Int
maximum a b | a < b 	= b
            | otherwise = a

max :: Bool -> Bool -> Bool
max a b = 
  let maximum a b = a || b
  in maximum a b
\end{lstlisting}
\begin{alertblock}{Achtung}
\lstinline|Prelude.max| für das durch \lstinline|Prelude| definierte oder \lstinline|Modulname.max| für unser \lstinline|max|
\end{alertblock}
\end{frame}


\begin{frame}
\frametitle{Auflösen von Namen}
\begin{block}{\vspace*{-3ex}}
\begin{itemize}
  \item Ohne Modul Angabe werden Funktionen nur im "`Import"' gesucht
  \item Prelude wird immer Importiert 
\end{itemize}
\end{block}
\end{frame}



\section{Ausdrücke}
\subsection{Ausdrücke allgemein}
\begin{frame}
\frametitle{Ausdrücke - Wiederholung} 
\begin{exampleblock}{Elementarer Ausdruck}
\lstinline|plus = 10 + 30|
\end{exampleblock}
\begin{exampleblock}{Ausdruck}
\lstinline|plus' a b = a + b|
\end{exampleblock}
\begin{exampleblock}{Lambda ($\lambda$)-Ausdruck}
\lstinline|plus'' =  \a -> \b -> a + b|
\end{exampleblock}
\end{frame}

\begin{frame}
\frametitle{Ausdrücke allgemein}
\begin{block}{\vspace*{-3ex}}
\begin{itemize}
  \item In Haskell besteht ein Ausdruck aus nur wenigen Grundelementen
  \begin{itemize}
    \item Konstante Werte
    \item Variablen Werte (Variablen)
    \item Funktionen
    \item Verzweigungen wie:\\Guards, If-Then-Else, Case-Of \ldots
  \end{itemize}
  \item Jeder Operator ist eine Funktion, die umdefiniert werden kann
\end{itemize}
\end{block}
\end{frame}

\begin{frame}[fragile]
\frametitle{Primitive Ausdrücke}
\begin{block}{\vspace*{-3ex}}
sind benannte bzw. unbenannte (anonyme) Funktionen mit konstanten Ergebnissen
\end{block}
\begin{lstlisting}
pi :: Double
pi = 3.14
\end{lstlisting}
\end{frame}

\begin{frame}[fragile]
\frametitle{Ausdrücke}
\begin{block}{\vspace*{-3ex}}
sind "`Berechnungen"' mit Variablen\\
besitzen "`fest"' definierte Parameter
\end{block}
\begin{lstlisting}
plus :: Int -> Int -> Int
plus a b = a + b
\end{lstlisting}
\end{frame}

\subsection{Lambda - Ausdrck}
\begin{frame}[fragile]
\frametitle{Lambda - Ausdruck}
\begin{block}{\vspace*{-3ex}}
\begin{itemize}
  \item Sind fast dasselbe wie normale Ausdrücke
  \item Parameter bzw. Variablen werden in $\lambda$-Notation angegeben
\end{itemize}
\end{block}
\begin{lstlisting}
let f = \x y -> x + y
\end{lstlisting}
\only<1-2>{
\begin{exampleblock}{Aufruf}
\lstinline|f 31 11|
\end{exampleblock}}
\only<2>{
\begin{exampleblock}{Ausgabe}
\lstinline|42|
\end{exampleblock}}
\end{frame}

\subsection{Strukturierte Anweisungen}
\begin{frame}
\frametitle{Strukturierte Anweisungen}
\begin{block}{\vspace*{-3ex}}
\begin{itemize}
  \item If-Then-Else
  \item Case-Of
  \item Pattern-Matching
  \item Guards als erweitertes Pattern-Matching
\end{itemize}
\end{block}
\end{frame}

\begin{frame}[fragile]
\frametitle{If-Then-Else}
\begin{block}{\vspace*{-3ex}}
\begin{itemize}
  \item Setzt das gewohnte If-Then-Else im Funktionsrumpf um
  \item Kann verschachtelt werden
\end{itemize}
\begin{center}
\scalebox{0.7}{\input{images/ifthenelse.tex}}
\end{center}
\end{block}
%\pause
\begin{lstlisting}
sum :: [Int] -> Int
sum []     = 0
sum (x:xs) = if mod x 2 == 1 && x > 20 
             then x + sum xs
             else sum xs
\end{lstlisting}
\end{frame}

\begin{frame}[fragile]
\frametitle{Case-Of}
\begin{block}{\vspace*{-3ex}}
\begin{itemize}
	\item Setzt das gewohnte Case-Of innerhalb von Funktionsrümpfen um
	\item Erster "`Treffer"' gewinnt
\end{itemize}
\only<1>{\begin{center}
\scalebox{0.7}{\begin{tikzpicture}[
      nonterminal/.style={
         rectangle,
         minimum size=5.0mm,
         very thick,
         draw=blue!50!black!50,
         top color=white,
         bottom color=blue!50!black!20,
         font=\itshape\scriptsize,
         %text height=1.5ex,
         %text depth=.25ex
      },
      terminal/.style={
         rounded rectangle,
         minimum size=5.0mm,
         very thick,draw=black!50,
         top color=white,bottom color=black!20,
         font=\ttfamily\scriptsize,
         %text height=1.5ex,
         %text depth=.25ex
      },
      skip loop/.style={
         to path={-- ++(0,#1) -| (\tikztotarget)}
      },
      point/.style={coordinate},>=stealth',thick,draw=black!50,
      tip/.style={->,shorten >=0.007pt},every join/.style={rounded corners},
      hv path/.style={to path={-| (\tikztotarget)}},
      vh path/.style={to path={|- (\tikztotarget)}},
      %text height=1.5ex,text depth=.25ex
    ]
    \matrix[column sep=5.0mm, row sep=3.0mm] {
\node (n11) [circle, draw, inner sep=0pt, minimum size=0.75ex] {}; & \node (n12) [terminal] {=}; & \node (n13) [terminal] {case}; & \node (n14) [nonterminal] {Variable}; & \node (n15) [terminal] {of}; & \node (n16) [terminal] {LF}; & \node (n17) [point] {}; & \\
 &  &  &  &  &  &  & \\
 &  &  & \node (n34) [point] {}; &  &  &  & \\
\node (n41) [point] {}; & \node (n42) [point] {}; & \node (n43) [nonterminal] {erwarteter Wert}; & \node (n44) [terminal] {->}; & \node (n45) [nonterminal] {Ausdruck}; & \node (n46) [terminal] {LF}; & \node (n47) [point] {}; & \node (n48) [point] {};\\
 &  &  &  &  &  &  & \\
 &  &  & \node (n64) [point] {}; &  &  &  & \\
\node (n71) [point] {}; & \node (n72) [point] {}; & \node (n73) [terminal] {\_}; & \node (n74) [terminal] {->}; & \node (n75) [nonterminal] {default Ausdruck}; & \node (n76) [point] {}; & \node (n77) [circle, draw, inner sep=0pt, minimum size=0.75ex] {}; & \\
    };
  { [start chain]
       \chainin (n11);
       \chainin (n12)    [join=by tip];
  }
  { [start chain]
       \chainin (n12);
       \chainin (n13)    [join=by tip];
  }
  { [start chain]
       \chainin (n13);
       \chainin (n14)    [join=by tip];
  }
  { [start chain]
       \chainin (n14);
       \chainin (n15)    [join=by tip];
  }
  { [start chain]
       \chainin (n15);
       \chainin (n16)    [join=by tip];
  }
  { [start chain]
       \chainin (n16);
       \chainin (n17)    [join=by tip];
  }
  { [start chain]
       \chainin (n41);
       \chainin (n42)    [join];
  }
  { [start chain]
       \chainin (n42);
       \chainin (n43)    [join=by tip];
  }
  { [start chain]
       \chainin (n43);
       \chainin (n44)    [join=by tip];
  }
  { [start chain]
       \chainin (n44);
       \chainin (n45)    [join=by tip];
  }
  { [start chain]
       \chainin (n45);
       \chainin (n46)    [join=by tip];
  }
  { [start chain]
       \chainin (n46);
       \chainin (n47)    [join];
  }
  { [start chain]
       \chainin (n47);
       \chainin (n48)    [join=by tip];
  }
  { [start chain]
       \chainin (n71);
       \chainin (n72)    [join];
  }
  { [start chain]
       \chainin (n72);
       \chainin (n73)    [join=by tip];
  }
  { [start chain]
       \chainin (n73);
       \chainin (n74)    [join=by tip];
  }
  { [start chain]
       \chainin (n74);
       \chainin (n75)    [join=by tip];
  }
  { [start chain]
       \chainin (n75);
       \chainin (n76)    [join];
  }
  { [start chain]
       \chainin (n76);
       \chainin (n77)    [join=by tip];
  }
  { [start chain]
       \chainin (n47);
       \chainin (n34)    [join=by vh path];
       \chainin (n42)    [join=by {hv path,tip}];
  }
  { [start chain]
       \chainin (n72);
       \chainin (n64)    [join=by vh path];
       \chainin (n76)    [join=by {hv path,tip}];
  }
\end{tikzpicture}
}
\end{center}}
\end{block}
\pause
\begin{lstlisting}
not :: Bool -> Bool
not a = case a of 
        True  -> False
        False -> True 
\end{lstlisting}
\end{frame}

\begin{frame}[fragile]
\frametitle{Pattern-Matching}
\begin{block}{\vspace*{-3ex}}
\begin{itemize}
  \item Testen auf erwartete Werte
  \item Erster "`Treffer"' gewinnt
\end{itemize}
\only<1>{\begin{center}
\scalebox{0.7}{\begin{tikzpicture}[
      nonterminal/.style={
         rectangle,
         minimum size=5.0mm,
         very thick,
         draw=blue!50!black!50,
         top color=white,
         bottom color=blue!50!black!20,
         font=\itshape\scriptsize,
         %text height=1.5ex,
         %text depth=.25ex
      },
      terminal/.style={
         rounded rectangle,
         minimum size=5.0mm,
         very thick,draw=black!50,
         top color=white,bottom color=black!20,
         font=\ttfamily\scriptsize,
         %text height=1.5ex,
         %text depth=.25ex
      },
      skip loop/.style={
         to path={-- ++(0,#1) -| (\tikztotarget)}
      },
      point/.style={coordinate},>=stealth',thick,draw=black!50,
      tip/.style={->,shorten >=0.007pt},every join/.style={rounded corners},
      hv path/.style={to path={-| (\tikztotarget)}},
      vh path/.style={to path={|- (\tikztotarget)}},
      %text height=1.5ex,text depth=.25ex
    ]
    \matrix[column sep=5.0mm, row sep=3.0mm] {
 &  &  & \node (n14) [point] {}; &  &  &  &  & \\
\node (n21) [circle, draw, inner sep=0pt, minimum size=0.75ex] {}; & \node (n22) [nonterminal] {Funktion}; & \node (n23) [point] {}; & \node (n24) [nonterminal] {erwarteter Wert}; & \node (n25) [terminal] { }; & \node (n26) [point] {}; & \node (n27) [terminal] {=}; & \node (n28) [nonterminal] {Ausdruck}; & \node (n29) [circle, draw, inner sep=0pt, minimum size=0.75ex] {};\\
    };
  { [start chain]
       \chainin (n21);
       \chainin (n22)    [join=by tip];
  }
  { [start chain]
       \chainin (n22);
       \chainin (n23)    [join];
  }
  { [start chain]
       \chainin (n23);
       \chainin (n24)    [join=by tip];
  }
  { [start chain]
       \chainin (n24);
       \chainin (n25)    [join=by tip];
  }
  { [start chain]
       \chainin (n25);
       \chainin (n26)    [join];
  }
  { [start chain]
       \chainin (n26);
       \chainin (n27)    [join=by tip];
  }
  { [start chain]
       \chainin (n27);
       \chainin (n28)    [join=by tip];
  }
  { [start chain]
       \chainin (n28);
       \chainin (n29)    [join=by tip];
  }
  { [start chain]
       \chainin (n26);
       \chainin (n14)    [join=by vh path];
       \chainin (n23)    [join=by {hv path,tip}];
  }
\end{tikzpicture}
}
\end{center}}
\end{block}
\pause
\begin{lstlisting}
xor :: Bool -> Bool -> Bool
xor True  True  = False
xor False False = False
xor _     _     = True
\end{lstlisting}
\end{frame}

\begin{frame}[fragile]
\frametitle{Pattern-Matching}
\begin{block}{\vspace*{-3ex}}
\begin{itemize}
  \item Testen auf erwartete Struktur
  \item Aufspalten des Datentypes
  \item Erster "`Treffer"' gewinnt
\end{itemize}
\only<1>{\begin{center}
\scalebox{0.65}{\begin{tikzpicture}[
      nonterminal/.style={
         rectangle,
         minimum size=5.0mm,
         very thick,
         draw=blue!50!black!50,
         top color=white,
         bottom color=blue!50!black!20,
         font=\itshape\scriptsize,
         %text height=1.5ex,
         %text depth=.25ex
      },
      terminal/.style={
         rounded rectangle,
         minimum size=5.0mm,
         very thick,draw=black!50,
         top color=white,bottom color=black!20,
         font=\ttfamily\scriptsize,
         %text height=1.5ex,
         %text depth=.25ex
      },
      skip loop/.style={
         to path={-- ++(0,#1) -| (\tikztotarget)}
      },
      point/.style={coordinate},>=stealth',thick,draw=black!50,
      tip/.style={->,shorten >=0.007pt},every join/.style={rounded corners},
      hv path/.style={to path={-| (\tikztotarget)}},
      vh path/.style={to path={|- (\tikztotarget)}},
      %text height=1.5ex,text depth=.25ex
    ]
    \matrix[column sep=5.0mm, row sep=3.0mm] {
 &  &  &  &  &  &  & \node (n18) [point] {}; &  &  &  &  &  &  &  &  & \\
 &  &  &  &  &  &  & \node (n28) [point] {}; &  &  &  &  &  &  &  &  & \\
 &  &  &  &  &  &  & \node (n38) [point] {}; &  &  &  &  &  &  &  &  & \\
\node (n41) [circle, draw, inner sep=0pt, minimum size=0.75ex] {}; & \node (n42) [nonterminal] {Funktion}; & \node (n43) [point] {}; & \node (n44) [point] {}; & \node (n45) [nonterminal] {Konstruktor}; & \node (n46) [terminal] {(}; & \node (n47) [point] {}; & \node (n48) [nonterminal] {erwarteter Wert}; & \node (n49) [terminal] { }; & \node (n410) [point] {}; & \node (n411) [terminal] {)}; & \node (n412) [point] {}; & \node (n413) [terminal] { }; & \node (n414) [point] {}; & \node (n415) [terminal] {=}; & \node (n416) [nonterminal] {Ausdruck}; & \node (n417) [circle, draw, inner sep=0pt, minimum size=0.75ex] {};\\
    };
  { [start chain]
       \chainin (n41);
       \chainin (n42)    [join=by tip];
  }
  { [start chain]
       \chainin (n42);
       \chainin (n43)    [join];
  }
  { [start chain]
       \chainin (n43);
       \chainin (n44)    [join];
  }
  { [start chain]
       \chainin (n44);
       \chainin (n45)    [join=by tip];
  }
  { [start chain]
       \chainin (n45);
       \chainin (n46)    [join=by tip];
  }
  { [start chain]
       \chainin (n46);
       \chainin (n47)    [join];
  }
  { [start chain]
       \chainin (n47);
       \chainin (n48)    [join=by tip];
  }
  { [start chain]
       \chainin (n48);
       \chainin (n49)    [join=by tip];
  }
  { [start chain]
       \chainin (n49);
       \chainin (n410)    [join];
  }
  { [start chain]
       \chainin (n410);
       \chainin (n411)    [join=by tip];
  }
  { [start chain]
       \chainin (n411);
       \chainin (n412)    [join];
  }
  { [start chain]
       \chainin (n412);
       \chainin (n413)    [join=by tip];
  }
  { [start chain]
       \chainin (n413);
       \chainin (n414)    [join];
  }
  { [start chain]
       \chainin (n414);
       \chainin (n415)    [join=by tip];
  }
  { [start chain]
       \chainin (n415);
       \chainin (n416)    [join=by tip];
  }
  { [start chain]
       \chainin (n416);
       \chainin (n417)    [join=by tip];
  }
  { [start chain]
       \chainin (n410);
       \chainin (n38)    [join=by vh path];
       \chainin (n47)    [join=by {hv path,tip}];
  }
  { [start chain]
       \chainin (n412);
       \chainin (n28)    [join=by vh path];
       \chainin (n44)    [join=by {hv path,tip}];
  }
  { [start chain]
       \chainin (n414);
       \chainin (n18)    [join=by vh path];
       \chainin (n43)    [join=by {hv path,tip}];
  }
\end{tikzpicture}
}
\end{center}}
\end{block}
\pause
\begin{lstlisting}
sum :: [Int] -> Int
sum []     = 0
sum (x:xs) = x + sum xs
\end{lstlisting}
\end{frame}

\begin{frame}[fragile]
\frametitle{Guards}
\begin{block}{\vspace*{-3ex}}
\begin{itemize}
  \item Erweitern das Pattern-Matching um boolesche Auswertungen in der Funktionsdefinition
  \item Erster "`Treffer"' gewinnt
\end{itemize}
\only<1>{\begin{center}
\scalebox{0.7}{\begin{tikzpicture}[
      nonterminal/.style={
         rectangle,
         minimum size=5.0mm,
         very thick,
         draw=blue!50!black!50,
         top color=white,
         bottom color=blue!50!black!20,
         font=\itshape\scriptsize,
         %text height=1.5ex,
         %text depth=.25ex
      },
      terminal/.style={
         rounded rectangle,
         minimum size=5.0mm,
         very thick,draw=black!50,
         top color=white,bottom color=black!20,
         font=\ttfamily\scriptsize,
         %text height=1.5ex,
         %text depth=.25ex
      },
      skip loop/.style={
         to path={-- ++(0,#1) -| (\tikztotarget)}
      },
      point/.style={coordinate},>=stealth',thick,draw=black!50,
      tip/.style={->,shorten >=0.007pt},every join/.style={rounded corners},
      hv path/.style={to path={-| (\tikztotarget)}},
      vh path/.style={to path={|- (\tikztotarget)}},
      %text height=1.5ex,text depth=.25ex
    ]
    \matrix[column sep=5.0mm, row sep=3.0mm] {
 &  &  &  & \node (n15) [point] {}; &  &  &  & \\
\node (n21) [circle, draw, inner sep=0pt, minimum size=0.75ex] {}; & \node (n22) [point] {}; & \node (n23) [terminal] {|}; & \node (n24) [nonterminal] {boolescher Term}; & \node (n25) [terminal] {=}; & \node (n26) [nonterminal] {Ausdruck}; & \node (n27) [terminal] {LF}; & \node (n28) [point] {}; & \node (n29) [point] {};\\
 &  &  &  &  &  &  &  & \\
\node (n41) [point] {}; & \node (n42) [terminal] {otherwise}; & \node (n43) [terminal] {=}; & \node (n44) [nonterminal] {default Ausdruck}; & \node (n45) [circle, draw, inner sep=0pt, minimum size=0.75ex] {}; &  &  &  & \\
    };
  { [start chain]
       \chainin (n21);
       \chainin (n22)    [join];
  }
  { [start chain]
       \chainin (n22);
       \chainin (n23)    [join=by tip];
  }
  { [start chain]
       \chainin (n23);
       \chainin (n24)    [join=by tip];
  }
  { [start chain]
       \chainin (n24);
       \chainin (n25)    [join=by tip];
  }
  { [start chain]
       \chainin (n25);
       \chainin (n26)    [join=by tip];
  }
  { [start chain]
       \chainin (n26);
       \chainin (n27)    [join=by tip];
  }
  { [start chain]
       \chainin (n27);
       \chainin (n28)    [join];
  }
  { [start chain]
       \chainin (n28);
       \chainin (n29)    [join=by tip];
  }
  { [start chain]
       \chainin (n41);
       \chainin (n42)    [join=by tip];
  }
  { [start chain]
       \chainin (n42);
       \chainin (n43)    [join=by tip];
  }
  { [start chain]
       \chainin (n43);
       \chainin (n44)    [join=by tip];
  }
  { [start chain]
       \chainin (n44);
       \chainin (n45)    [join=by tip];
  }
  { [start chain]
       \chainin (n28);
       \chainin (n15)    [join=by vh path];
       \chainin (n22)    [join=by {hv path,tip}];
  }
\end{tikzpicture}
}
\end{center}}
\end{block}
\pause
\begin{lstlisting}
sum :: [Int] -> Int
sum []     = 0
sum (x:xs) | mod x 2 == 1 && x > 20 = x + sum xs
           | otherwise              = sum xs
\end{lstlisting}
\end{frame}

\begin{frame}[fragile]
\frametitle{Sturkturierte Anweisungen Beispiel \lstinline|ggt a b|}
\begin{block}{If-Then-Else}
\begin{lstlisting}
ggT a b = if b == 0 then a else ggT b (mod a b)
\end{lstlisting}
\end{block}
\end{frame}

\begin{frame}[fragile]
\frametitle{Sturkturierte Anweisungen Beispiel \lstinline|ggt a b|}
\begin{block}{If-Then-Else}
\begin{lstlisting}
ggT a b = if b == 0 then a else ggT b (mod a b)
\end{lstlisting}
\end{block}
\begin{block}{Case-Of}
\begin{lstlisting}
ggT' a b = case b of 
                0 -> a
                _ -> ggT' b (mod a b)
\end{lstlisting}
\end{block}
\end{frame}

\begin{frame}[fragile]
\frametitle{Sturkturierte Anweisungen Beispiel \lstinline|ggt a b|}
\begin{block}{If-Then-Else}
\begin{lstlisting}
ggT a b = if b == 0 then a else ggT b (mod a b)
\end{lstlisting}
\end{block}
\begin{block}{Pattern-Matching}
\begin{lstlisting}
ggT'' a 0 = a
ggT'' a b = ggT'' b (mod a b)
\end{lstlisting}
\end{block}
\end{frame}

\begin{frame}[fragile]
\frametitle{Sturkturierte Anweisungen Beispiel \lstinline|ggt a b|}
\begin{block}{If-Then-Else}
\begin{lstlisting}
ggT a b = if b == 0 then a else ggT b (mod a b)
\end{lstlisting}
\end{block}
\begin{block}{Guards}
\begin{lstlisting}
ggT''' a b | b == 0    = a
           | otherwise = ggT''' b (mod a b) 
\end{lstlisting}
\end{block}
\end{frame}

\subsection{Iterationsanweisungen}
\begin{frame}[fragile]
\frametitle{Iterationsanweisungen}
\begin{block}{\vspace*{-3ex}}
\begin{itemize}
  \item In Haskell existieren keine Schleifen wie
  \item while, while do, for, repeat until \ldots
  \item Jede "`Schleife"' muss über rekursive Funktionen realisiert werden
  \item Hierzu werden Funktionen höherer Ordnung benötigt
\end{itemize}
\end{block}
\begin{lstlisting}
filter :: (Int -> Bool) -> [Int] -> [Int]
filter do []     = 0
filter do (x:xs) | do x       = x : filter do xs
                 | otherwise  = filter do xs
\end{lstlisting}
\vspace*{-3ex}
\begin{block}{\vspace*{-3ex}}
Wendet \lstinline|do| auf jedes Element der Liste an.
\end{block}
\end{frame}

\subsection{Sprunganweisungen}
\begin{frame}
\frametitle{Sprunganweisungen}
\begin{block}{\vspace*{-3ex}}
Es existieren keine Sprunganweisungen.
\end{block}
\end{frame}
